% This is part of the FinalReport document.
% Copyright (C) 2011 Piyabhum Sornpaisarn, Andrew Busse, Michael Hodgson, John Charlesworth, Paramithi Svastisinha
% See the file COPYING in FinalReport/ for copying conditions.

\section{Physical Implementation}

The physical implementation of the payload has taken many forms, reflecting 
the each of the design-prototype-test cycles that our group experienced.

\begin{itemize}
\item Initially, the system was built in two separate parts: The first was 
the modified version of the ATmega168P-based sample schematic provided by 
our customer, which handled communication to the payload port of the 
autopilot. The second was an Arduino Uno R2 with its Serial (UART) line 
connected to the camera and an SPI connection to an SD card. This system 
simply took an image from the camera and wrote it to the SD card. The SD card 
to Arduino connection was via some protoboard and an SD card reader breakout 
board on loan from Zepler Stores.
\item The next design iteration used an "Il Matto" board and daughter board, 
designed for 2011's D4 lab (a two-week Design and Build exercise for second 
year Electronic Engineers). The board itself is rather simple - an ATmega644P 
with all of its ports connected to expansion headers, an SPI line connection 
to an SD card reader, and power taken from a mini-USB connection, stepped 
down to 3V3 by a linear regulator. A daughter board (which slots into the 
expansion headers) has the transceiver to communicate with the autopilot

This implementation of the payload contains all the elements of the final 
schematic, so technically could be test-flown.
\item Whereas the above approach would technically fulfil the specification, 
implementing it would be expensive: the Il Matto boards were one-off boards 
designed specifically for a lab, so a customer would certainly need to 
manufacture and build their own, including the daughter board. Having this as 
a single PCB solution would be a lot neater, smaller, lighter and cheaper.

Therefore, a final PCB was designed (this is elaborated upon in 
\ref{sec:PCB-implementation}), specific to our application, with all the 
features we need. Only through-hole components are used where possible, 
making the task of soldering components to the PCB as easy as possible.
\end{itemize}
