
\section{Local Image Storage}
\label{sec:local_storage}

With a maximum image size of $640\times480$ pixels, an uncompressed JPEG 
image's file size would be approximately 900kB.
This is greater than the internal memory on the ATmega168, ATmega328 and 
ATmega644 (1KiB, 2KiB, and 4KiB respectively) microprocessors that were used 
during development of this project, therefore it cannot be guaranteed that a 
complete single image can be stored locally, without adding some extra memory.
\\
The advantages of being able to store a full image locally are:
\begin{itemize}
\item Would allow all original images taken during flight to be viewed once 
the UAV has been landed.
\item Would make the implementation of automatic repeat requests (resending of 
any corrupted payload -$>$ ground station packets) much easier.
\item Customer mentioned that currently, customers take images with their 
retrofitted cameras, but store the images locally instead of wirelessly 
transmitting them. Keeping the functionality to store all images taken 
on-board and then plug something into the ground station to view these once 
the UAV has been landed would be desirable, but not essential (hence why this 
is not mentioned in the specification).
\end{itemize}

The disadvantages:
\begin{itemize}
\item The material cost of the payload (in both component cost and PCB area) 
would be increased.
\item Addition of an extra component increases the project's complexity.
\end{itemize}

\subsection{Approach: Flash Chip}

One option considered was the implementation of an external flash memory chip 
(for evaluation purposes, the 4Mbit AMIC A25L040-F was purchased)

Advantages:
\begin{itemize}
\item Non-volatile memory. Allows data to be retained even if power to the payload module is cut off.
\item Excellent data integrity: ~100000 Write/Erase cycles, and $>$10 years data retention.
\item Relatively good operating current of ~20mA makes it suitable for battery-powered application.
\end{itemize}

Disadvantages:
\begin{itemize}
\item Chips that contain a significant amount of memory storage (e.g. 1Gbit 
and above), come in packages that are awkward to prototype (BGA, TSOP)
\item Interfacing this type of device with a PC would require a similar amount 
of work to the ground station software.
\item Non-standard interface. The command set for the 4Mbit device is not the 
same as for a 1Gbit device.
\end{itemize}



\subsection{Approach: SD Card}

This is essentially the same as the flash chip above, but with an additional controller chip, and packaged in a standard case.

Advantages:
\begin{itemize}
\item Non-volatile memory. Allows data to be retained even if power to the payload module is cut off.
\item Same data integrity as for the above item.
\item Getting image data from the card would be as trivial as removing the 
card from the payload and inserting it into the ground station laptop. SD card 
readers are cheap, and ubiquitous.
\item Good value for money, when compared to price-per-Mbit of the other two 
options under consideration. (For example, a 2GB microSD card (including 
microSD -$>$ SD adapter) from Amazon costs ~£4.
\end{itemize}

Disadvantages:
\begin{itemize}
\item Open source libraries can only make use of an SD card's SPI Bus
interface, not the quicker (but patent-covered) four-bit SD bus.
\item Only standard SD cards may be used, not the more recent SDHC or SDXC 
cards, which are increasingly popular.
\end{itemize}

\subsection{Approach: SRAM}

SRAM was considered as a possible option as we had all used this before.

Advantages:
\begin{itemize}
\item We have previous experience of interfacing these with AVRs from a 
second-year lab, 
\item This type of memory performs read/write operations much quicker than 
the other two options.
\item Very low operating current of ~10mA makes it suitable for this battery-powered application.
\end{itemize}

Disadvantages:
\begin{itemize}
\item Standard SRAM is volatile, so data is prone to being lost if the payload 
module power is lost, or the payload module is reset. (While non-volatile SRAM 
does exist, it is much more expensive per Mbit (~£18/Mbit) than the other 
options under consideration.)
\item Comparatively expensive
\item Communication with the device requires connection to all data, address 
and control pins (or implementation of shift registers with the same desired 
effect), this would require more board space. (The other two devices can 
communicate via an SPI bus, a four wire connection)
\end{itemize}