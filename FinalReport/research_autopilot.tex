\section{SkyCircuits Autopilot - (ab)}

Our customer kindly lent us one of their more advanced, SC2 autopilot modules 
for use in this project, as well as use of the latest version of their Ground 
Control Station software.

This module \cite{SC2} is a robust, professional unmanned avionics system, 
able to take complete control of an unmanned aircraft from take-off to landing. 
The device is low power (0.8W at 5V), lightweight (290g including aluminium 
enclosure), with a comprehensive sensor range (able to measure height, pitch, 
roll and yaw to an impressive level of accurary), capable of controlling up to 
6 servomotors, and able to operate at a number of wireless frequencies.

Wireless communication with the ground station is done with 
XBee PRO modules, with an RF data rate of 250kbps and a 
line of sight (LOS) range of up to 1.6km. However, no data 
integrity checking (Automatic Repeat reQuesting, ARQ) of 
payload data is done either by the XBee or the Ground 
Control Station software.

\subsection{Autopilot Payload Module Interface}
\label{sec:autopilot_payload_interface}
|||||| Maybe have this in background research ||||||

The SkyCircuits autopilot module allows extension modules named `payload 
modules' to be connected to the autopilot. These payload modules are connected 
via a RS485 serial connection at 38.4 kBaud, allowing several payload modules
to be connected at once in a daisy-chain configuration. All payload modules are 
connected to common TX and RX lines, where the RX line is used by the
autopilot to send commands and data to the payload modules, and the TX
line is used by all daisy-chained modules to communicate with the autopilot.

||||||| Include diagram of daisy chain configuration ||||||||||

Since the TX line is shared between all modules only one payload module can be 
transmitting at once over the link, with all other payload modules required to
leave the line tristated. This means that each payload module must know when it is
allowed to use the transmit line so as not to clobber any other payload module.
In this system this is achieved by the use of `transmit tokens' handed out 
by the autopilot over the RX line. A `transmit token' is sent to each payload 
module in turn, informing the module that it is clear to transmit data. With 
only one payload module connected to the autopilot, these tokens are sent out 
every 20 ms. |||||| CHECK THIS ||||||||

This two way communication link is used to implement a command interface to 
the autopilot. A payload module can execute commands on the autopilot, 
allowing a variety of useful and interesting possibilities for payload
module design. Of interest to us however, is the ability of the payload
module to set shared memory. Since this shared memory can be accessed 
through the ground station software this allows us to send data through the
autopilot link to the ground station. Shared memory is allocated to 
each payload module, accessible on the ground station using the command:
~\\
~\\
\begin{lstlisting}[caption={Accessing shared memory from ground station}, label=lst:gs_shared_mem_set]
payload[payload_num].mem_bytes[mem_bytes_num]
\end{lstlisting}

Where \emph{payload\_num} is the ID number identifying the payload and 
\emph{mem\_bytes\_num} is the index of the set of shared memory to be accessed.

Each shared memory set is of variable length and can be set from the payload 
module using the following function (code provided by customer):
~\\
~\\
\begin{lstlisting}[language=C, caption={Setting shared memory from payload module}, label=lst:payload_shared_mem_set]
send_set_class_indexed_item_indexed(CLASS_PAYLOAD, module_id, 
CLASS_PAYLOAD_MEM_BYTES, mem_bytes_num, message_to_send,
message_to_send_length)
\end{lstlisting}

Where \emph{CLASS\_PAYLOAD} and \emph{CLASS\_PAYLOAD\_MEM\_BYTES} are constants 
informing the autopilot that the \emph{mem\_bytes} item of the \emph{payload} 
object should be set, \emph{module\_id} is the ID of the payload module,
\emph{mem\_bytes\_num} is the same as used in listing
\ref{lst:gs_shared_mem_set} and \emph{message\_to\_send} is the message to be 
sent (consisting of an array of length \emph{message\_to\_send\_length}, the 
first element of which should be the number of bytes to set in the shared 
memory).

The ground station software can also send data directly to a payload module 
through the autopilot, where it will be sent on to the RX line of the RS485
bus as a `packet' of data. The \emph{send\_bytes} command is used to do this, as can be seen in 
listing \ref{lst:ground_station_send_bytes}. 

~\\
\begin{lstlisting}[caption={Sending bytes 1 to N from ground station to payload module}, label=lst:ground_station_send_bytes]
payload[0].send_bytes BYTE1 BYTE2 ... BYTEN 
\end{lstlisting} 
