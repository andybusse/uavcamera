\lstset{numbers=left,
numberstyle=\footnotesize, 
stepnumber=2,numbersep=5pt}

\section{MATLAB code for custom image compression}

Compression1.m
\begin{lstlisting}[language=Matlab, label = lst:cust_comp1, caption = {Whole image, transform based compression code}]
clear all
close all

% load clown
load mandrill

X = (X-min(min(X)))./max(max(X-min(min(X))));

%% fft

freqX = fft2(X);

testFreqX = zeros(size(freqX));

shiftedFreqX = fftshift(freqX);

centre = size(freqX)/2;
figure

for z = 10:-0.5:1.5
    squarewidth = round(centre(1)/z);
    squareheight = round(centre(2)/z);
    % squarewidth = 10;
    % squareheight = 16;
    testFreqX = zeros(size(freqX));
    
    for x = -squarewidth:squarewidth;
        for y = -squareheight:squareheight;
            testFreqX(centre(1)+x,centre(2)+y) = shiftedFreqX(centre(1)+x,centre(2)+y);
        end
    end
    
    testFreqX = fftshift(testFreqX);
    
    testX = abs(ifft2(testFreqX));
    
    imagesc(testX)
    colormap gray
    drawnow
    pause(0.1)
end

%% dct

cfreqX = dct2(X);
bigness = size(cfreqX);

% squarewidth = 10;
% squareheight = 16;
for z = 10:-0.5:1.5
    
    ctestFreqX = zeros(size(cfreqX));
    
    squarewidth = round(bigness(1)/z);
    squareheight = round(bigness(2)/z);
    
    for x = 1:squarewidth;
        for y = 1:squareheight;
            ctestFreqX(x,y) = cfreqX(x,y);
        end
    end
    
    testX = abs(idct2(ctestFreqX));
    
    imagesc(testX)
    colormap gray
    drawnow
    pause(0.1)
end
\end{lstlisting}

%--------------------------------------------------------------
chunkCompression.m
\begin{lstlisting}[language=Matlab, label = lst:chunk_comp, caption = {Secioned image, transform based compression code}]
clear all
close all

% load clown
load mandrill

X = (X-min(min(X)))./max(max(X-min(min(X))));


%% fft

freqX = fft(fft(X)')';

testFreqX = zeros(size(freqX));

shiftedFreqX = fftshift(freqX);

centre = size(freqX)/2;
figure

for z = 10:-0.5:1.5
    squarewidth = round(centre(1)/z);
    squareheight = round(centre(2)/z);
    % squarewidth = 10;
    % squareheight = 16;
    testFreqX = zeros(size(freqX));
    
    for x = -squarewidth:squarewidth;
        for y = -squareheight:squareheight;
            testFreqX(centre(1)+x,centre(2)+y) = shiftedFreqX(centre(1)+x,centre(2)+y);
        end
    end
    
    testFreqX = fftshift(testFreqX);
    
    %     testX = abs(ifft2(testFreqX));
    testX = abs(ifft(ifft(testFreqX)')');
    
    imagesc(testX)
    colormap gray
    drawnow
    pause(0.1)
end

%% enchunk

chunksize = 20; % must be even
imagesize = size(X);

% X2 = X;
% X = X(1:480,1:480);

for x = 1:round(imagesize(1)/chunksize)
    for y = 1:round(imagesize(2)/chunksize)
        chunk(x,y).data = X((((x-1)*chunksize)+1):(x*chunksize), (((y-1)*chunksize)+1):(y*chunksize));
    end
end

%% chunks fft

for z = 0.5:(chunksize/2)-0.5
    
    for chunkcount = 1:round(imagesize(1)/chunksize)*round(imagesize(2)/chunksize)
        
        chunk(chunkcount).freqX = fft(fft(chunk(chunkcount).data)')';
        
        chunk(chunkcount).testFreqX = zeros(size(chunk(chunkcount).freqX));
        
        chunk(chunkcount).shiftedFreqX = fftshift(chunk(chunkcount).freqX);
        
        centre = (chunksize/2)+0.5;
        
        for x = centre-z:centre+z;
            for y = centre-z:centre+z;
                chunk(chunkcount).testFreqX(x,y) = chunk(chunkcount).shiftedFreqX(x,y);
            end
        end
        
        chunk(chunkcount).testFreqX = fftshift(chunk(chunkcount).testFreqX);
        
        %     testX = abs(ifft2(testFreqX));
        chunk(chunkcount).testX = abs(ifft(ifft(chunk(chunkcount).testFreqX)')');
        
    end
    
    for x = 1:round(imagesize(1)/chunksize)
        for y = 1:round(imagesize(1)/chunksize)
            ReconTestX((((x-1)*chunksize)+1):(x*chunksize), (((y-1)*chunksize)+1):(y*chunksize)) = chunk(x,y).testX;
        end
    end
    
    figure
    
    imagesc(ReconTestX)
    colormap gray
    drawnow
    pause(0.5)
end
\end{lstlisting}

\section{Arduino code}



\begin{lstlisting}[language=C, label = lst:arduino_captureTest, caption = {Arduino code, used up until the point we started using the Il Matto. This was written in the arduino-022 IDE}]
// Camera -> SD Card Code

// PIN LAYOUT:
// Digital 13  <->  SD CLK
// Digital 12  <->  SD MISO / Data 0 / SO
// Digital 11  <->  SD MOSI / CMD / DI
// Digital 4   <->  SD D3
// Arduino RX  <->  Camera TX
// Arduino TX  <->  Camera RX

#include <SoftwareSerial.h>
#include <SD.h>
#include <stdio.h>

#define DLOG(...)  sSerial.print(__VA_ARGS__)
//#define DLOG(...)
byte SYNC[] = {0xAA,0x0D,0x00,0x00,0x00,0x00};
byte RXtest[6];

// Serial PC cable to view error messages
#define sRxPin 2
#define sTxPin 3

SoftwareSerial sSerial = SoftwareSerial(sRxPin, sTxPin);
char filePrefix[] = "test"; //Prefix of file name written to the SD card
char fileExt[] = ".jpg"; //File extension of the file being written to
char fileName[13]; //Contains full file name of file being written to
char numBuf[5]; //Number buffer, for the "find a file that doesn't exist" procedure. Could be a number up to 32767
File jpgFile;

void setup()
{
  // start serial port at 115200 bps
  Serial.begin(115200);
  sSerial.begin(9600);
  
  // start software serial library for debugging
  pinMode(sRxPin, INPUT);
  pinMode(sTxPin, OUTPUT);
  // Pin 10 = Chip Select
  pinMode(10, OUTPUT);
  
  // If SD not connected, stop execution
  if (!SD.begin(4)) {
    DLOG("SD Failed");
    return;
  }
  DLOG("SD Working"); 
  
  //DLOG("Attemting to establish contact.\n\r");
  establishContact();  // send a byte to establish contact until receiver responds
  //DLOG("Contact established, captain\n\r"); 
  delay(2000); // 2 second pause
  
  
}

void loop()
{
  sSerial.read(); //Wait for something to be sent over the PC serial line
  
  boolean fileExists = true;
  int fileNum = 0;
  while(fileExists){
    fileNum++;
    snprintf(fileName, sizeof(fileName), "%s%i%s", filePrefix, fileNum, fileExt); // prints "test{num},jpg" to fileName
    fileExists = SD.exists(fileName); // checks whether that file is on the SD card.
  }
  
  sSerial.println(fileName);
  setupSD(fileName); // creates the file on the sd for writing
  
  boolean wellness = takeSnapshot(); //Take a snapshot
  if(wellness){ //If no error is detected
   DLOG("All's well\n\r");
   delay(2000);
  }else{ //Error has been detected
    DLOG("Trouble at mill...\n\r");
  }
  jpgFile.close(); // Close JPG file
}

// setup the file on the sd card

void setupSD(char fileName[]){
  // For Optimisation of the SD card writing process.
  // See http://www.arduino.cc/cgi-bin/yabb2/YaBB.pl?num=1293975555
  jpgFile = SD.open(fileName, O_CREAT | O_WRITE);
}

// synchronise with camera
void establishContact() {
  DLOG("Sending syncs\n\r\r"); // Send SYNC to the camera.
  //This could be up to 60 times
  while(1){
    while (Serial.available() <= 0) {
      DLOG(".");
      sendSYNC();
      delay(50);
    }
    
    receiveComd();
 // Verify that the camera has sent back an ACK followed by SYNC
    if(!isACK(RXtest,0x0D,0x00,0x00))
      continue;
    DLOG("ACK received\n\r");
    receiveComd();
     
   if(!isSYNC(RXtest))
     continue;
   
   DLOG("SYNC received\n\r");
   // Send back an ACK
   sendACK(0x0D,0x00,0x00,0x00);
   
   break;
  }
}

// takes a snapshot
boolean takeSnapshot() {
  
  // setup image parameters
  DLOG("sending initial\n\r");
  delay(50);
  sendINITIAL(0x07,0x07,0x07);
  receiveComd();
  if(!isACK(RXtest,0x01,0x00,0x00))
      return false;
  DLOG("ACK received\n\r");
  
  // Package size = maximum of 512 bytes
  // Arduino serial.read() buffer size = 128 bytes
  unsigned int packageSize = 64;
  
  // sets the size of the packets to be sent from the camera
  DLOG("sending setpackagesize\n\r");
  delay(50);
  sendSETPACKAGESIZE((byte)packageSize,(byte)(packageSize>>8));
  receiveComd();
  if(!isACK(RXtest,0x06,0x00,0x00))
      return false;
  DLOG("ACK received\r");
  
  // camera stores a single frame in its buffer  
  DLOG("sending snapshot\n\r");
  delay(50);
  sendSNAPSHOT(0x00,0x00,0x00);
  receiveComd();
  if(!isACK(RXtest,0x05,0x00,0x00))
      return false;
  DLOG("ACK received\n\r");
      
  // requests the image from the camera
  DLOG("sending getpicture\n\r");
  delay(50);
  sendGETPICTURE(0x01);
  receiveComd();
  if(!isACK(RXtest,0x04,0x00,0x00))
      return false;
  DLOG("ACK received\n\r");
  receiveComd();    
  if(!isDATA(RXtest))
      return false;
  DLOG("DATA received\n\r");
  
  // Strips out the "number of packages" from the DATA command
  // then displays this over the serial link  
  unsigned long bufferSize = 0;
  bufferSize = RXtest[5] | bufferSize;
  bufferSize = (bufferSize<<8) | RXtest[4];
  bufferSize = (bufferSize<<8) | RXtest[3];
  
  unsigned int numPackages = bufferSize/(packageSize-6);
  
  DLOG("Number of packages: ");
  DLOG(numPackages,DEC);
  sSerial.println();
  
  byte dataIn[packageSize];
  int flushCount = 0;
  
  for(unsigned int package = 0;package<numPackages;package++){
    
    //DLOG("Sending ACK for package ");
    //DLOG(package,DEC);
    //DLOG("\n\r");
    sendACK(0x00,0x00,(byte)package,(byte)(package>>8));
    
      // receive package
      for(int dataPoint = 0; dataPoint<packageSize;dataPoint++){
        while (Serial.available() <= 0) {} // wait for data to be available n.b. will wait forever...
        dataIn[dataPoint] = Serial.read();
          if(dataPoint > 3 && dataPoint < (packageSize - 2)){ //strips out header data
          //sSerial.print(dataIn[dataPoint],BYTE);
          jpgFile.print(dataIn[dataPoint]);
            if(flushCount == 511){
              jpgFile.flush();
              flushCount = -1; // because it adds one straight away after
            }
          flushCount++;
          }
        //DLOG(dataPoint);
        //DLOG("\n\r");
      }
    //DLOG("Package ");
    //DLOG(package);
    //DLOG(" read successfully\n\r");

  }
  
  sendACK(0x0A,0x00,0xF0,0xF0);
  DLOG("Final ACK sent\n\r");
  return true;
}

// The following are all self-explanatory.
// Follows command structure described in the camera's datasheet

void sendSYNC() {
  sendCommand(0xAA,0x0D,0x00,0x00,0x00,0x00);
}

void sendACK(byte comID, byte ACKcounter, byte pakID1, byte pakID2) {
 sendCommand(0xAA,0x0E,comID,ACKcounter,pakID1,pakID2);
}

void sendINITIAL(byte colourType, byte rawRes, byte jpegRes) {
 sendCommand(0xAA,0x01,0x00,colourType,rawRes,jpegRes);
}

void sendSETPACKAGESIZE(byte paklb, byte pakhb) {
 sendCommand(0xAA,0x06,0x08,paklb,pakhb,0x00);
}

void sendSNAPSHOT(byte snapType, byte skip1, byte skip2) {
 sendCommand(0xAA,0x05,snapType,skip1,skip2,0x00);
}

void sendGETPICTURE(byte picType) {
 sendCommand(0xAA,0x04,picType,0x00,0x00,0x00);
}

// sends commands to the camera
void sendCommand(byte ID1, byte ID2, byte par1, byte par2, byte par3, byte par4) {
  byte Command[] = {ID1,ID2,par1,par2,par3,par4};
  Serial.write(Command, 6);
}

// receives messages from the camera
void receiveComd() {
  for(int z = 0;z<6;z++){
      while (Serial.available() <= 0) {}
      RXtest[z] = Serial.read();
  }
}

// verifies ACK
boolean isACK(byte byteundertest[],byte comID, byte pakID1, byte pakID2){
  byte testACK[] = {0xAA,0x0E,comID,0x00,pakID1,pakID2};
  for(int z = 0;z<6;z++){
    if((z != 3) && (byteundertest[z] != testACK[z]))
      return false;
  }
  return true;
}

// verifies SYNC
boolean isSYNC(byte byteundertest[]){
  for(int z = 0;z<6;z++){
    if(byteundertest[z] != SYNC[z])
      return false;
  }
  return true;
}

// verifies DATA
boolean isDATA(byte byteundertest[]){
    if((byteundertest[0] != 0xAA) || (byteundertest[1] != 0x0A)){
      return false; }
  return true;
}
\end{lstlisting}
