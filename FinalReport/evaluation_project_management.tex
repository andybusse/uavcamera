% This is part of the FinalReport document.
% Copyright (C) 2011 Piyabhum Sornpaisarn, Andrew Busse, Michael Hodgson, John Charlesworth, Paramithi Svastisinha
% See the file COPYING in FinalReport/ for copying conditions.

\section{Evaluation on Project Management (ps)}

In order to successfully meet the specifications of the project, the task needs to be prioritized so that high priority tasks are implemented before the others.
The tasks with high risk, many dependencies, and requiring the most time to complete were the first set of tasks to be completed.
Because we prioritized the tasks well, we had enough time to implement the high priority objectives of the customer's specification.
Although some low priority tasks were completed some tasks such as the progressive image display have been omitted due to time constraints.

The project management took advantage of group meetings. Each members of the group had to produce a formal documentation in order to keep track of and assign the new tasks.
Some meetings did not have enough progress done beforehand and were not as effective as hoped.
If the group agreed to and managed to implement something on every meeting, it would have made the implementation of the project much faster. 

One thing to be learned from this project is the delivery of components which can be very fast or very slow. 
Therefore, before ordering parts, the group would need to check the delivery time and make sure that the delay had a minimal effect on the progress of work.
The hardware arrived on a slightly delayed and so some members of the team who were assigned tasks related to the hardware were forced to reallocate and perform other tasks while waiting for them to be delivered.
