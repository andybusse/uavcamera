% This is part of the FinalReport document.
% Copyright (C) 2011 Piyabhum Sornpaisarn, Andrew Busse, Michael Hodgson, John Charlesworth, Paramithi Svastisinha
% See the file COPYING in FinalReport/ for copying conditions.

\section{Network and Port Connection - (ps)}

TCP/IP is suitable for device communication and it can be implemented in many software languages.
The connection can refer to an interconnect between host and router.
Host is a computer that runs a program such as a web browser, or any software on the computer \cite{davidB}.
A Router is a device that can forward communication from one machine to the another.
There are many ways to link machines using TCP.
A port connection is one way of connecting the TCP from the router to the host.
Port can refer to hardware such as a USB connection or it can refer to a processes on the machine \cite{normanM}.
A protocol is an agreement about the packet structre and what they mean between communicating programs.
Information that will sent through the port will be a sequence of data bytes. 
These byte sequences generated are called packets \cite{davidB}.
It includes user data and control information.

TCP/IP is designed to detect and recover its losses, and other errors that might occur so the application does not have to deal with this \cite{davidB} i.e. the application data does not have to break up the data on its own. It has functions to connect to the computer, send and receive commands, display flags, and set up the port values.


 
