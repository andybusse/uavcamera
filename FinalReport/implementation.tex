%% ----------------------------------------------------------------
\chapter{Implementation}
%% ----------------------------------------------------------------


% This is part of the FinalReport document.
% Copyright (C) 2011 Piyabhum Sornpaisarn, Andrew Busse, Michael Hodgson, John Charlesworth, Paramithi Svastisinha
% See the file COPYING in FinalReport/ for copying conditions.

\section{Camera Module (jc)}
\label{sec:John_Implementation}

The first step in the implementation of the camera module was to verify the communication with the camera by talking to it via a pc (milestone \ref{sec:ms_img_from_cam}). Once this step was complete the next step was to implement communications between the camera and a microprocessor, with the pc for debugging. With the microcontroller able to communicate with the camera this module was ready for integration with the payload (milestone \ref{sec:ms_img_cam_controller_implementation}).

\begin{figure}[H]
        \centering
        \includegraphics[width=1.00\textwidth]{figures/CameraModuleBlock1.png}
        \captionof{figure}{Block diagram of the camera module showing where communication takes place}
        \label{fig:camera_block}
\end{figure}

Figure \ref{fig:camera_block} shows how the camera module is broken down separate to the rest of the system. The camera and the camera controller communicate to set up the camera and get images, the camera controller then stores the image to the SD card and debug messages are sent to the computer during these processes so that the camera is less of a black-box.

During the development of the system 2 cameras (uCam \cite{ucam_datasheet}) died on us. The first very early on in development and the second after the system had been effectively finished apart from a few minor tweaks and implementation of some low priority functionality. These component malfunctions were most likely due to the uCam being sensitive to static so after the first one failed we were careful to only handle the second one on an anti-static mat whilst wearing a wrist-strap, however someone must have gotten careless towards the end of the project.

\subsection{First camera}
\label{sec:cam_prob}
The first attempt at communicating with the original camera was done using the arduino uno microcontroller board. The arduino was chosen as the platform for this first communication because it is simple to program and its programming environment provides various useful libraries, for example for serial communication \cite{arduino_serial_library}. The hardware also provides an easily accessible serial port.

The first implementation of this code managed to occasionally sync (see figure \ref{fig:syncProto}) with the camera (milestone \ref{sec:ms_basic_cam_comm}) but would generate errors when trying to send any further commands. After examining the code and the camera's datasheet it was discovered that the camera was only syncing occasionally because the arduino was using a slower baud rate than the camera could auto-detect, after increasing the baud rate on the arduino the camera would sync every time but the code still generated errors after this point.

It was at this point that the first camera broke. It was no longer syncing (or sending back anything at all, verified by oscilloscope readings), where before it had been doing so reliably and so it was checked with the 4D systems software and shown not to be working \ref{sec:existing_software_test}.

\subsection{Second camera}

Once it was established that the first camera was dead a couple of cheaper surface mount camera modules were purchased as possible replacements however making a successful physical (and therefore also communication) connection to these components proved excessively difficult so a new camera of the same type as the previous one \cite{ucam_datasheet} was ordered. The successfull connection test is seen in section \ref{sec:basic_connection_test}, with debug info observed using a serial monitor.

\subsubsection{Serial cable to computer}

The operation of the new camera module was verified using a usb to serial cable connected to a pc which was running the sample program which was provided by the camera's manufacturer \cite{ucam_test_software}. Using this set up it was shown that the camera was working and that it was possible to get images from it, see section \ref{sec:existing_software_test}.

%[] diagram of pin connections []

\subsection{Arduino implementation}
\label{sec:arduino_imp}

With the operation of the camera verified it was reconnected to the arduino board and the code run again with the same result as before: it would correctly sync (see figure \ref{fig:syncProto} and section \ref{sec:basic_connection_test}) but would generate errors when trying to send any further commands. On inspection of the code it became clear that this was because the same serial line was being used for both communication with the camera and debug messages and that these debug messages were interfering with further communications. Debug messages were therefore moved to a software serial line and sent to the computer via the usb to serial cable and observed using a serial monitor.

The code is broken down so that there are functions for each type of command that the camera can receive, functions that verify the responses from the camera and functions that combine these together in the correct sequence in order to perform a useful task with the camera.

\subsubsection{Camera synchronisation}

The first task in communicating with the camera is to synchronise the serial channel, the uCam datasheet \cite{ucam_datasheet} gives the correct protocol to do this.

\begin{figure}[H]
        \centering
        \includegraphics[width=1.00\textwidth]{figures/SyncProtocal.png}
        \captionof{figure}{Command protocol for synchronisation, sourced from \cite{ucam_datasheet} }
        \label{fig:syncProto}
\end{figure}

It should be noted that in \ref{fig:syncProto} the first sync sent from the host will be repeated until an acknowledgement is received, with everything working correctly this process usually takes 3 or 4 syncs before an ACK is received.

The inclusion of a separate debugging line allows for messages at each stage of this process to verify that it is connecting correctly and to indicate where any errors may have occurred. With all debug messages enabled this function will output "Sending syncs" followed by a "." for each sync command sent and then "ACK received" and "SYNC received", the main code will then output a message to the effect that contact has been successfully established.

\subsubsection{Taking a snapshot}

With the camera successfully synchronised the controller needs to be able to trigger the camera to take a photograph and then retrieve said photograph, again the uCam datasheet \cite{ucam_datasheet} gives an example of how to implement this.

\begin{figure}[H]
        \centering
        \includegraphics[width=1.00\textwidth]{figures/SnapshotProtocal.png}
        \captionof{figure}{Command protocol for taking and retrieving a jpeg snapshot at 640x480 resolution, sourced from \cite{ucam_datasheet} }
        \label{fig:snapProto}
\end{figure}

It should be noted in \ref{fig:snapProto} that the values in the commands are for the example and are not necessarily the values used in the code.

The "INITIAL" command has parameters that allow the image type and resolution to be set, in \ref{fig:snapProto} the image type is set to JPEG and the resolution is set to 640x480, these values are kept as the standard in the code.

The "SET PACKAGE SIZE" command is only relevant for jpeg images and set the size of the data packages that are sent when one requests an image. In \ref{fig:snapProto} this value is set to 512 bytes however in the code this is set to 64 bytes as it was originally thought that the system might have to transmit these packages as soon as receiving them so they needed to be a reasonable size to fit in with the timing between transmit packets sent by the autopilot.

The "SNAPSHOT" command tells the camera to take a single picture, its parameter sets whether the image is jpeg compressed or raw, in \ref{fig:snapProto} this is set to be a jpeg image and this is left as the default in the code.

The last command sent by the controller is the "GET PICTURE" command which tells the camera to send image data, it has one parameter which sets which image is sent. In \ref{fig:snapProto} this is set to send the snapshot image and this is also how it is set in the code. As well as sending back an ACK this function will also send back a "DATA" message which tells the controller what type of image is being returned and how large it is.

Once the controller has acknowledged the "DATA" message the camera starts to send data packages, each of which needs to be acknowledged, until the entire image is transferred. Whilst this process is going on the data packages are saved to an SD-card as described in section \ref{sec:SD_imp}.

The use of the debug communication channel again allowed for detailed analysis of this process with the maximum amount of debug messages being one before each command is sent saying what that command is and another message once and ACK is received, there is also then a message with the size of the data in and then messages with the package number and the byte number within that package, the data can also be output to the debug channel, see section \ref{sec:image_capture_test}.

\subsection{Miscellaneous Problems}
\label{sec:misc_cam_probs}

Before the SD-card was implemented the sending of the data directly via the debug channel was attempted and so was its reconstruction into a jpeg image on the computer, this could not be made to work however, see section \ref{sec:image_capture_test}. On looking at the data as it came off the arduino via the software serial debug line it became clear (particularly when looking at the counting values, whose values were already known) that the data was not being transmitted reliably. This could have been because of noise on the line, the fact that the software serial line is communicating via pins on the arduino that aren't optimised for the function or simply failings in the arduino software serial library \cite{software_serial}. For this reason this method of getting an image was abandoned, this also had to be taken into consideration whilst thinking about the integration of this part of the system: the software serial line is not a reliable method of transmitting the data so some other method would have to be used.

Another problem that was noticed after the camera control had been integrated into the payload controller, as described in section \ref{sec:int_pl}, was that every other time the system was powered up the camera would require a reset/power cycle. This was discovered to be because the Rx wire to the camera was left floating, this was fixed by the addition of a 10k$\Omega$ pull-up resistor, see appendix \ref{appendix_schematics} for schematics.

It was also noticed quite early on that the camera required 3.3V logic \cite{ucam_datasheet} levels but the arduino outputs 5V logic levels, the arduino could receive logic at 3.3V easily enough but the 5V output was causing occasional errors. In order to fix this problem the 5V output from the arduino was taken through a potential divider of a 10k$\Omega$ resistor and a 3V zener diode.


\section{Payload Controller}
\subsection{General Payload Controller Implementation OR MAYBE Overview}
||||||| Might not want this in its own section ||||||||

The payload controller module is an important part of the system, responsible
for interfacing with the camera module and communicating with the ground 
station image viewer software via the autopilot. Since this single module 
encapsulates a significant amount of the complexity of the project it was 
deemed sensible to split it up into sub-modules, which could be worked on in
parallel by different members of the team. 

With this in mind the payload controller module was split into four main 
submodules:

\begin{itemize}
	\item Camera Module Communication
	\item Communication with Ground Station via Autopilot
	\item SD Card Image Buffering
	\item Progressive JPEG Manipulation
\end{itemize}

||||||| INCLUDE REFERENCES TO EACH ||||||||||

||||||| INCLUDE SUBMODULE DIAGRAM OF THE PAYLOAD MODULE |||||||


\subsection{Communication with Ground Station via Autopilot}
Considering the overall aim of this project: to produce a system by which 
images can be downloaded over-the-air from a payload module to a ground 
station, it is fair to say that some method of communicating between the
payload module and ground station are an essential component in the system.

The specification - see chapter \ref{chap:specification} - requires the payload
module to communicate with the ground station using the autopilots payload
module interface (discussed in section \ref{sec:autopilot_payload_interface}
below).

%To better explain the protocol used we will split the explanation into two 
%sections: a \emph{Autopilot Payload Interface} section describing the
%pre-existing autopilot payload interface on which we are building the 
%protocol and a \emph{UAV Camera Communication Protocol} section describing the 
%protocol we have implemented as a part of this project.

\subsubsection{Autopilot Payload Module Interface}
\label{sec:autopilot_payload_interface}
|||||| Maybe have this in background research ||||||

The SkyCircuits autopilot module allows extension modules named `payload 
modules' to be connected to the autopilot. These payload modules are connected 
via a RS485 serial connection at 38.4 kBaud, allowing several payload modules
to be connected at once in a daisy-chain configuration. All payload modules are 
connected to common TX and RX lines, where the RX line is used by the
autopilot to send commands and data to the payload modules, and the TX
line is used by all daisy-chained modules to communicate with the autopilot.

||||||| Include diagram of daisy chain configuration ||||||||||

Since the TX line is shared between all modules only one payload module can be 
transmitting at once over the link, with all other payload modules required to
leave the line tristated. This means that each payload module must know when it is
allowed to use the transmit line so as not to clobber any other payload module.
In this system this is achieved by the use of `transmit tokens' handed out 
by the autopilot over the RX line. A `transmit token' is sent to each payload 
module in turn, informing the module that it is clear to transmit data. With 
only one payload module connected to the autopilot, these tokens are sent out 
every 20 ms. |||||| CHECK THIS ||||||||

This two way communication link is used to implement a command interface to 
the autopilot. A payload module can execute commands on the autopilot, 
allowing a variety of useful and interesting possibilities for payload
module design. Of interest to us however, is the ability of the payload
module to set shared memory. Since this shared memory can be accessed 
through the ground station software this allows us to send data through the
autopilot link to the ground station. Shared memory is allocated to 
each payload module, accessible on the ground station using the command:
~\\
\begin{lstlisting}[caption={Accessing shared memory from ground station}, label=lst:gs_shared_mem_set]
payload[payload_num].mem_bytes[mem_bytes_num]
\end{lstlisting}

Where \emph{payload\_num} is the ID number identifying the payload and 
\emph{mem\_bytes\_num} is the index of the set of shared memory to be accessed.

Each shared memory set is of variable length and can be set from the payload 
module using the following function (code provided by customer):
~\\
\begin{lstlisting}[language=C, caption={Setting shared memory from payload module}, label=lst:payload_shared_mem_set]
send_set_class_indexed_item_indexed(CLASS_PAYLOAD, module_id, 
CLASS_PAYLOAD_MEM_BYTES, mem_bytes_num, message_to_send,
message_to_send_length)
\end{lstlisting}

Where \emph{CLASS\_PAYLOAD} and \emph{CLASS\_PAYLOAD\_MEM\_BYTES} are constants 
informing the autopilot that the \emph{mem\_bytes} item of the \emph{payload} 
object should be set, \emph{module\_id} is the ID of the payload module,
\emph{mem\_bytes\_num} is the same as used in listing
\ref{lst:gs_shared_mem_set} and \emph{message\_to\_send} is the message to be 
sent (consisting of an array of length \emph{message\_to\_send\_length}, the 
first element of which should be the number of bytes to set in the shared 
memory).

The method through which this shared memory is accessed via the ground station
image viewer is discussed in section ||||||||| GS IMAGE VIEWER |||||||||.

The ground station software can also send data directly to a payload module 
through the autopilot, where it will be sent on to the RX line of the RS485
bus. The \emph{send\_bytes} command is used to do this, as can be seen in 
listing \ref{lst:ground_station_send_bytes}. 

As discussed in section |||||||| REF |||||||| it was decided that 
our communications protocol would use shared memory and \emph{send\_bytes} 
commands, allowing two way communications between the payload controller and 
ground station software to be established.

\subsubsection{UAV Camera Communication Protocol}
The Payload Module Interface discussed above (section \ref{sec:autopilot_payload_interface})
allows us to send strings of bytes in both directions. However, in order to 
communicate with the ground station image viewing software some form of 
additional communications protocol is required so that both ends of the link 
are communicating in a mutually understandable manner.

This two way communications is the interface between the payload and the 
ground station software, so some standard protocol was required. It was decided
that a message based system would be used, with the messages from the ground
station to the payload module being sent using \emph{send\_bytes} and the 
messages sent from the payload to the ground station being put into shared
memory. Each message is composed of two elements, one byte for the message ID 
- unique to each type of message - and a variable number of data bytes 
(depending on the message type.) The different message types are detailed 
below:

\paragraph{Messages sent from Ground Station To Payload}

\begin{itemize}
\item \textbf{Take Picture}
\begin{itemize}
\item \emph{Data:} None
\item Prompts payload module to capture an image and save it to the SD card.
\end{itemize}

\item \textbf{Image Download Request} 
\begin{itemize}

\item \emph{Data:} Image ID
\item Requests the payload send the image with ID \emph{Image ID} to the 
ground station. This message allows any image stored by the payload module 
to be downloaded over the connection, increasing flexibility. 
\end{itemize}

\item \textbf{Configure Camera}
\begin{itemize}
\item \emph{Data:} Colour Type, Raw Image Resolution, JPEG Image
Resolution
\item Sets the image resolution and colour mode of the camera. Only the 
JPEG mode has been tested so far.
\end{itemize}

\end{itemize}

\paragraph{Messages Sent from Payload to Ground Station}

\begin{itemize}

\item \textbf{Picture Taken}

\begin{itemize}
\item \emph{Data:} Image ID

\item Informs the ground station software that an image has been taken and 
saved to the SD card. \emph{Image ID} is the ID of the image that has been 
saved to the SD card.
\end{itemize} 

\item \textbf{Image Download Info}

\begin{itemize}

\item \emph{Data:} Number of Image Packets

\item Sent by the payload after a successful \emph{Image Download Request}
message from the ground station. Informs the ground station how many 
\emph{Image Data} packets to expect.
\end{itemize}

\item \textbf{Image Data} 
\begin{itemize}
\item \emph{Data:} Packet Number, Image Data
\item This message contains an amount of actual image data. Sent after a
\emph{Image Download Info} message which is in turn in response to an 
\emph{Image Download Request} message. The whole image is sent over
\emph{Number of Image Packets} packets (as defined by the \emph{Image Download
Info} message.) \emph{Packet Number} informs the ground station which of these
packets the message is carrying. \emph{Image Data} contains the actual image 
data for this packet and is variable size, with a maximum size of 50 bytes. 
\end{itemize}

\end{itemize}


\subsubsection{Existing Code}
\label{sec:payload_existing_code}
Our customer had provided us with some payload module communication AVR code
- written for a ATMega168 - for communicating with the autopilot. This code
was the basis on which the payload controllers communication link was built.

The code provided a number of useful utilities:
\begin{itemize}
\item Ability to set shared memory on the autopilot.

\item Ability to receive messages sent from the Ground Station to the 
autopilot.

\item Example code for setting shared memory on the autopilot.
\end{itemize}

This base code was modified slightly after a bug was found in its handling of 
the transmit enable signal. The RS485 communication protocol used for the 
autopilot-payload link (as described in section |||||||| SEC ||||||||) 
requires a `transmit enable' signal to be asserted when the payload is 
transmitting. This signal should be asserted just before data is to be sent 
and cleared just after. However, the original payload base code cleared this 
signal in an interrupt service routine (ISR) which fired after the transmit 
buffer of the UART was ready to accept new data. Since this transmit buffer 
would be ready to accept new data before the data was actually sent over the
physical connection this lead to the transmit enable signal being cleared
before all data had been sent, causing strange behaviour on the RS485 link.
This bug did not seem to cause any problems, and the odd behaviour was only
noticed when testing the system with an oscilloscope. ||||| INC TRACES ||||||
It was considered sensible to fix the bug in case it did cause problems later.

The fix for this problem was reasonably simple: a new ISR was set up which 
fired only when the current transmission had actually completed, and the 
command to clear the transmit enable signal was moved into this ISR.
||||| INC AFTER TRACE ||||||


||||||||| FLOWCHART OF INTERACTIONS |||||||

\subsection{Payload Debug Interface}
\label{sec:payload_debug_interface}
While the payload module was being developed debugging was a major concern. 
Having access to an oscilloscope very useful when debugging some particularly 
difficult problems, as described elsewhere in this report ||||||| WHERE |||||.
However, while building the payload module it was clear more complex details 
about the internal state of the program as it was running would be useful.

While prototyping the camera module on the Arduino platform this problem was 
solved by using the \emph{SoftwareSerial} library (for information regarding 
the library see \cite{software_serial}) to send textual debug information via
a serial to USB cable to PC. This proved very useful when prototyping camera
communications.

A slightly different approach was taken when using the ATMega644P. Initial 
testing of the software serial library on this platform suggested that it 
would not work without a significant investment in time fixing it. It was
therefore decided that it would be faster to send debug messages over 
`bit-banged' SPI to a spare Arduino board which then forwarded this data to 
the connected PC over serial. Again, having the use of this debug interface 
proved very useful while implementing the system, and was well worth the 
small amount of extra time taken to develop it.

||||| INC EXAMPLE OF DEBUG TEXT (MAYBE SCREENSHOT) ||||||

\subsection{Integration of Payload Sub-Modules}

\subsubsection{REMINDER: DISCUSSION OF ARDUINO MULTIPLEXED SERIAL}

\subsubsection{Implementation on ATMega644P}
Initially we planned to implement the system on the Arduino prototyping 
platform with an expansion `shiled' board providing any extra hardware needed -
The reasons for this are discussed in section |||||| REF |||||||. 
This plan relied on the implementation of the multiplexed serial solution 
discussed above in section |||||||| REF ||||||||. 

Unfortunately, due to time constraints alongside setbacks such as those
discussed in sections ||||||| REF AUTOPILOT BUG |||||||| and |||||||| REF 
CAMERA FAILURE |||||||| it was decided that implementing the system on the 
Arduino platform may not the best way to proceed, since it would almost 
certainly be non-trivial to implement and debug.

Several options were considered alongside the multiplexing option. One 
alternative workaround discussed was to use a separate AVR as an interface
to the autopilot and connect it to the Arduino using bit-banged SPI or the
inbuilt two-wire interface (TWI, sometimes known as I2C). This approach was 
attractive as it meant we could use the existing Arduino code for the camera 
and SD card and use the existing ATMega168 code for the autopilot communication.
However, this solution would add another complex component to the system along
with another communications link, increasing complexity significantly.

Another option considered was using an AVR device with multiple UARTs so one 
could be used for the camera and another for the autopilot link. However, 
the code we had written for interacting with the SD card buffer relied upon 
some Arduino specific libraries. Finding new libraries and rewriting this code 
would be time consuming and inefficient, especially since we had already 
invested time in creating a working SD card solution.

Further investigation into the Arduino SD card library code (for more info
regarding the library see reference \cite{arduino_sd_library}) suggested that
it was coded to work with the ATMega644P chip as well as the ATMega168 and
ATMega328P. Both the ATMega168 and ATMega328P feature only one UART, but the 
ATMega644P features two, making it attractive for use in this project (see
\cite{atmega644p} for more info about the ATMega644P).

Since the ATMega644P provides two UARTs and the SD card library we were using
supported this chip it was decided that using the 644P would be a sensible 
way forward. 

It is important to note that frequent progress reviews meant that this problem
was caught early before it could become more damaging.

As mentioned above in section \ref{sec:payload_existing_code} the code provided
to us for communication with the autopilot was written for an ATMega168 chip. 
Porting this code and the existing Arduino camera code to the 644P was not a 
trivial task, although it was simplified by the use of the Sanguino library -
a port of the Arduino core libraries to the 644P (see \cite{sanguino}).

\section{REMINDER: PROBLEMS CHALLENGES}
\section{REMINDER: HOW DID WE SOLVE PROBLEMS, DEBUGGING, TESTING, ETC}
\section{REMINDER: JOHN COULD TALK ABOUT HOW BROKEN CAMERAS SLOWED DOWN DEVELOPMENT BUT GOOD PLANNING AND CONTINGENCY MINIMISED RISK OR IN MANAGEMENT SECTION}
\section{REMINDER: FUTURE WORK}


\section{Ground Station Image Viewer}
The implementation of the program needs to start from many very basic programs such as connect to the port, change byte data to image, and change image to byte. 
So the first prototype was implemented by using console application in order to save time while debugging and we can see the output from the UAV. In order to see that the DataStream port stream data, the UAV set to send some data on the customer’s program, and then the data will be taken through the port and so we can see from the software.  

\subsection{Get Picture Function}
The 'Get Picture' functionality is the most important part of the GUI. The plan is when the user click on the Get Picture button, the program sends a string command to the ground station software with correct command bytes. Then the ground station software will generate a command byte and transmitted by TCP to the payload. Then the payload sends a ''Picture Taken'' command back through the data stream port. The GUI will then automatically send a download request command to the payload. The payload will then send image data back to the data stream port. The small programs that have been implemented in the early stage of development are reused here to send and receive data from the port. 

\subsection{File Dialogue}
For a .NET C\# Programming, there is a previous tools that we can use to change directory, open and save files. The OpenFileDialog,  SaveFileDialog, and FolderBrowser dialogue are existing classes that is suitable to do this job. However, to open any file dialogue, the application needs to have a STAThread in order to handle multiple objects. STAThread is an attribute which applied to the main method, indicating that the application should communicate with unmanaged COM code using the Single Threading Apartment. It allows main thread or the main form to run in the background while the file dialogue is executing. 

These file dialogues are not stated in the image viewer specification. But it has been implemented because the user may install the program in many computers and the initial folder might not exist and this might cause the system to give an error. It is also make the users feel that they can use the program more freely.  It also give the advantage of limiting the access to only an existing file, so there is no error when trying to save the image.

\subsubsection*{GUI connection}
 When the program start running, the program initialized the port and commands the customer’s application to tell the UAV to stream data to the data stream port. Figure~\ref{GCS_connect_command} shows the connection between the UAV data stream port and the ground station. The UAV has two ports, console port, and data stream port and it can send and receive any length of data. SEND\_ ZERO\_ TOKEN is designed by the customer’s so when the DataStream port receives this handshaking token, it will start streaming data. When any data transmission and receive, the software transfer the data through pipes. There are some global initializations that the software needs to perform only once when it is loaded for the first time. Note that the console port is manually connect by the ground control station software. The image viewer will give an error when there is no connection.
 The method to stream data is by writing this code to the user program:
 
\begin{center}
\texttt{da 20 ht}
\end{center}

''da'' defines which graph on the customer's program to display the data. ''20'' define the time period to send an amount of data in millisecond. ''ht'' is the customer's defined code to show height on da graph. 



\begin{figure}[!hbtp]
\begin{center}
\includegraphics[scale=0.5]{connect_command.png} 
\end{center}
\caption{The connection of data stream port\label{GCS_connect_command}}
\end{figure}


\subsubsection*{GUI data flow diagram}

\begin{figure}[!hbtp]
\begin{center}
\includegraphics[scale=0.6]{GCS_Payload_communication.PNG} 
\caption{The connection of data stream port\label{GCS_Payload_comm}}
\end{center}
\end{figure}

Table~\ref{command_table} shows how the command receive and send at ground station.SEND\_ZERO\_TOKEN is use for toggle the data stream port in order to make it send data to the image viewer. TAKE\_ PICTURE command sends 0 byte to command port by sending a string command to the UAV as shown in Figure~\ref{GCS_Payload_comm}. The PICTURE\_ TAKEN signal act in a similar way to ACK (acknowledgement) command which send back to the ground with a value of number of packet of the byte data. To ensure that the number of packet is the same in ground station and the UAV, the SEND\_ DOWNLOAD\_ REQUEST command sends back the information of number of packet. The number of packet shows how many cycle does the image view to do to receive all the data.  The payload will then send the DOWNLOAD\_ INFO to the ground station which follows by the IMAGE\_ DATA. The image data consist of 3 information bytes the first byte is always 4, this number tell the image view program that it is an image data. The next 2 bytes is saying which number of packet it is. If there is any packet skipped, the image viewer program will noticed and either send error or ask the UAV for the old set of packet. The image data at each cycle have a variable length. The program is designed to take the variable length of data.

\begin{table}[!htbp]

\begin{center}
\begin{tabular}{l l @{.} l}
 Command&
\multicolumn{2}{l}{Address Byte } \\

\hline
\underline{Command from Ground} & \\
SEND\_ZERO\_TOKEN & 0 \\
TAKE\_PICTURE & 0 \\
SEND\_DOWNLOAD\_REQUEST & 2 [MSB] [LSB]  \\
\\
\underline{Command received at Ground}\\
PICTURE\_TAKEN & 1 [MSB] [LSB]\\
DOWNLOAD\_INFO & 3 [MSB] [LSB]\\
IMAGE\_DATA & 4 $\overbrace{ [packet number]}^{2bytes} \overbrace{[image data]}^{data length}$ \\
\end{tabular}
\caption{Command table\label{command_table}}
\end{center}
\end{table}
 
 
\subsection{Code Highlight}

This section describe the code that is important for the program. The entire code will not be described but it will be in the appendix. In the image viewer program. It needs a class that can do these to port: connect, receive and send. FileStream class is a class in the .NET C\# which can create a file. BinaryWriter class used for writing the byte data into the generated file made by the FileStream class. The file directory will introduce a thread. While the open or save a file, the main application must be running, so the application need to deal with multi threads at the same time. Also when the get picture button got clicked, the main application is frozen because of the thread time organize do one thing at a time. The special code of thread need for handle this thread. 

\subsubsection*{Connect to UAV}
The Socket class has functions to send and receive byte and strings data. The handshaking protocol is using this code:

\begin{lstlisting}[caption={connect to port},label=lst:connectT]
public void ConnectToPort(Int32 portNumber, string portName)
{
        try            
        {
             Port.Connect(portName, portNumber);                
        }            
        catch 
        {            
            MessageBox.Show("Error code
                
             \n Unable to connect to dataStreamPort:
             \n Please check that:                
             \n1.The dataStreamPort is connected                 
             \n2.The program gcs has opened                 
             \n3.The program has connect to the dataStreamPort                
             \n4.The program has send stream data                
             \n5. The program as to run testbyte on da");                
        }            
}
        \end{lstlisting}

        
	The design of the .NET Socket class simply connect to a Port by a single command without any hesitation of changing the baud rate, stop bits, and parity bits. This advantage makes the Socket class a more useful class to work with the Port with existing and static set up. 

\subsubsection*{Start of the program}

\begin{lstlisting}[caption={Start of the program}, label=lst:payload_shared_mem_set]
	statusLabel.Text = ''Starting'';   
	progressBar.Value = 1;   
	string fileName = string.Format(''uavPictureAt { 0:yyyy-MM-dd_hh-mm-ss-tt}.jpg'', DateTime.Now);
	FileStream fileStream;
	fileStream = new FileStream(filePathTextBox.Text+''\\"+fileName, FileMode.Create);      
	BinaryWriter opFile = new BinaryWriter(fileStream);
	uavConn.SendTextToUAV("da 20 payload[0].mem_bytes[0]");      
 \end{lstlisting}
             In order to make the file name different and meaningful, the name of the picture will be the time and date of the time taken the picture. This has been done using the date and time class. fileStream was initiate to be in FileMode.Create, so it can create file. The BinaryWriter write the binary byte into a file in the directory of the fileStream. 
            
\texttt{string fileName = string.Format("uavPictureAt{ 0 : yyyy-MM-dd\_ hh-mm-ss-tt}
 .jpg", DateTime.Now);   }  
        
            This code time setting is valid for a file name. It will display year, month,date, and time in this order. The BinaryWriter class can create a binary file using specific data layout for its bytes. 

\subsubsection*{Change File Directory}
If the user wish to change the directory of the image taken, the image viewer program must support it. the mnuOpen\_click() method introduce an OpenFileDialog class. It begin with open file dialog box. This allow the user to open an inital image to the program. If the user took a picture, it will be saved in the same directory as the open file. This file dialog is limited to only the jpg picture in order to avoid the error of displaying the image. 
This file dialog introduce an extra thread. Without thread handler, the program will automatically organize the thread in the same timeline. This means it have to finish one event before it start another. The more detail about thread will be discussed in the thread section. 
\begin{lstlisting}[caption={change file directory},label=lst:changeFD]

        private void mnuOpen_Click(object sender, EventArgs e)        
        {        
             OpenFileDialog fileOpen = new OpenFileDialog();      
            
             fileOpen.Title = ''Select file to open:'';   
             fileOpen.Filter =''(* .JPG)| *.JPG; |(*.* )| *.* '';           

             if (fileOpen.ShowDialog() == DialogResult.OK)    
             {
    
                 updateDirectory(fileOpen.FileName);     
                 if (jpegList.Length != 0)     
                 {                    
                     pictureBox.Image = Image.FromFile(jpegList[filePathCount]);       
                 }         
                else       
                {       
                     pictureBox.Image = Image.FromFile(jpegList[0]);        
                }        
             }        
             pictureBox.SizeMode = PictureBoxSizeMode.StretchImage;       
             fileOpen.Dispose();        
        }       
        \end{lstlisting}
\subsubsection*{Threading}

A Scheduler that is reponsible for time-slicing threads controls the thread execution, manaing blocking of I/O message and signal handling\cite{keithC}. In any program there is always a main or initial thread running in a recursive loop that repond to the client application. In image viewer program, the main thread is the window application. When the get picture button got clicked, it will generate another less fundamental thread which will be execute after the loop have done. Therefore, in order to deal with this application while the picture is loading we need

\begin{center}
\texttt{Application.DoEvents();.}
\end{center}

During a process of taking picture and loading from the sky, the program wait for the loop to be finished before the user can do any action on the program. This is because while the code is processing, all other events wait in the queue, and it makes the program stop working. This can be fixed by using Application.DoEvents(). This code processes all of the Windows event messages that have queued up \cite{davidW}. When this code has been applied, the application can deal with other event at the same time as the code is running.


\begin{lstlisting}[caption={thread handling in the main},label=lst:threadH]
        [STAThread]        
        static void Main(string[] args)        
        {        
            mainForm.Show();
            while (true)            
            {            
                Application.DoEvents();                
            }
        	...
        \end{lstlisting}
        Application.DoEvents() in the Main function delegate that wraps the method that indicate where to start execution. The thread begin to run when the Application.DoEvents(); get called\cite{xieX}.When the file dialog try to run, the main application must be close. The [STAthread] has to be introduced in order to run multi thread at the same time. But, this doesn't mean the processor will be faster, but thread can make use of the resources that would go unused. A background thread can continue to run, while a foreground thread waits for the user input. This is called an apartment thread. 
        
\subsubsection*{Text Command}
TCP/IP protocols transfer data without modifying them. This allow the application to freely encode the data.\cite{davidB}.The Ground station software allow the program to send a stream of string in bytes and it will read the command bytes and send it to the payload on the UAV. The code has shown the way to implement the string and send a byte array to the payload.
	
	

\begin{lstlisting}[caption={send text in byte array},label=lst:sendT]
	 public void SendTextToUAV(string textToUAV)	        
         {        
             char[] toUAVChar = new char[512];       
             byte[] toUAVByte = new byte[512];        
             byte[] fromUAVByte = new byte[1000];       
             byte[] oneByteArray = new byte[1];       
             textToUAV += '' $\backslash$ n'';        
             toUAVChar = textToUAV.ToCharArray();        
             toUAVByte = System.Text.Encoding.ASCII.GetBytes(toUAVChar);       
             try       
             {      
                 int sendByte = consolePort.Send(toUAVByte, toUAVChar.Length, SocketFlags.None);     
             }     
            catch (SocketException ex)      
            {       
                Console.WriteLine(''ERROR\: '' + ex.Message);       
            }       
         }
\end{lstlisting}
                

To send text, the string of characters is translated into an array of bytes. American Standard Code for Information Interchange(ASCII) is use for translating English into a binary code. In the System.Text classes provide converting mechanism between each character sets. The ASCIIEncoding.GetBytes() is used for convert character array into a byte array.  
The Socket.Send() Method allow the user to send byte stream to the port connected.The customer's program will read from the port and display it onto the command line. The ''@ '' sign indicate that the command correctly sent from the application to the ground station software. The advantage of linking to the ground station software is that our customers can understand what is going on in the console line. For example:

\begin{center}
uavConn.SendTextToUAV(''da 20 payload[0].mem\_ bytes[0]'')\;
\end{center}

The text ''da 20 payload[0].mem\_ bytes[0]'' will be converted to char array and then to byte array. The byte array will then send the command to the console port to the payload by \texttt{consolePort.Send(toUAVByte, toUAVChar.Length)}. The customer is familiar with the ground station software program, so this method of sending data is satisfy. 

\subsubsection*{I/O Streams}
    The .NET framework's stream class is use as a powerful tools for encoding and decoding\cite{davidB}. In the image viewer program, we use FileStream to create a directory and to save the byte data into image. However, the program might have to deal with a corrupted image data. Framing is the problem of the receiver failed to find the beginning and the end of the message. The solution is the data packet give the information of how many loop must the program do in order to receive all the image data. The packet address locate in the second and third byte of the image data stream. 
     

\begin{lstlisting}[caption={writing binary file},label=lst:writingb]          
	for (int i = 3; i < packetSize; i++)
	{
          opFile.Write(packet[i]);
          numBytes++;
    	}
\end{lstlisting}         

       At every cycle of the data being received, opFile.Write() method will write the packet data into the file in the directory. After the cycle finished, the file will be saved and the image will be displayed in the picture box. 
\subsubsection*{Update the Directory}
The update directory method update itself every time the file directory is changing. This method will update the string of all the possible JPEG file in order to make it display in the picture box correctly. So when the left or right button got clocked, the string of JPEG file name will get plus or minus. This array of string have to update because it changes when the picture got deleted, or the new picture got taken. The array called \texttt{jpegList} is use as a string file storage for all the JPEG possible file
\begin{lstlisting}[caption=update directory class highlight, label=updateD]
            bool fileNameIsDirectory = false;
            if (fileName.Substring(fileName.Length - 4, 4).Contains("."))
            {
                fileDirectory = fileName.Substring(0, fileName.LastIndexOf("\\"));
                fileNameIsDirectory = false;

            }
            else
            {
                fileDirectory = fileName;
                fileNameIsDirectory = true;
            }
            filePathTextBox.Text = fileDirectory;
            try
            {
                jpegList = Directory.GetFiles(fileDirectory, "*.jpg");
            }
            catch
            {
            }
\end{lstlisting}
\subsubsection*{Other functions}
The user might want to delete some unwanted photo, so the delete button have been implemented. The delete button work like a normal file deleting button. But it has been complicated because the photo to delete must be the photo on the pictureBox. There is an error because the photo is using by the pictureBox. To fix this problem, the picture have to shifted left or right first and then delete the photo. This will solve almost all the problem. But where there is only one picture in the file, the program can not delete because even the pictureBox is set to null, it's last memory is still point at the deleting file. However, the disadvantage of the delete button is that if the wanted photo got deleted accidentally, it might take a long time to launch the UAV again and take the same photo.

\begin{figure}[!hbtp]
\begin{center}
\includegraphics[scale=1]{resolutionOption.PNG} 
\end{center}
\caption{The resolution in combo box\label{resolutionOption}}
\end{figure}
The camera has options of resolution as shown in Figure\ref{resolutionOption}. This can be useful when the speed is important. The lower the resolution, the faster the data transmitted to the ground. GUI has the combo box for the user to choose any wanted resolution in the options. The resolution allow user to have more accessible to the camera. However, this mean there is more on the programmer side to program the application.


The baud rate that we can transmit data from the UAV is 38.4kbaud. Bandwidth limitations severely restrict the volume of data to transfer over the wireless link. It takes around 8 to 20 seconds to transmit an image of resolution 640x480. This progress bar tells the user how much percentage of data received. The progress bar update at each cycle of the data receives. At the end of each image downloaded, the progress bar reset to its normal state.  The status text tells the user what signal have been sent or received. This status text ensures that the picture is downloading and it is good for testing that the tasking in process. These two nice application allocate on the bottom left of the GUI as shown in Figure\ref{progressBar}. However, these have to implement every cycle of the data collection and it might cause the cycle to run slower. The actual loop is much faster than the 38.4kbaud, therefore there is no problem implementing these in.
\begin{figure}[!hbtp]
\begin{center}
\includegraphics[scale=1]{progressBar.PNG} 
\end{center}
\caption{The resolution in combo box\label{progressBar}}
\end{figure}


\subsection{Work Flow Diagram}
Describe Work Flow

\begin{figure}[!hbtp]
\begin{center}
\includegraphics[scale=1]{finalWorkFlow.PNG} 
\end{center}
\caption{Final work flow diagram of the GUI\label{GUI_finalWorkFlow}}
\end{figure}
\section{Progressive JPEG Manipulation}

\section{Physical Implementation}
\label{sec:PCB-implementation}

The final, delivered module is a $90mm\times62mm$ PCB. The PCB manufactured 
for the project was done so for free, using Spirit Circuits' "Go Naked"
service \cite{go-naked}. The PCB itself is a "tracks and holes" only service 
- no soldermask or silkscreen is applied. The schematic of the circuit 
delivered is available in Appendix ???, and of the PCB layout in Appendix 
???. A waterproof lacquer will be applied to the PCB to prevent condensation 
from shorting tracks together, and the module itself presented in a 
waterproof container before flight testing.

\begin{figure}[H]
        \centering
        \includegraphics[width=1.00\textwidth]{figures/PayloadImplementation.jpg}
        \captionof{figure}{Image of the final payload, under test, before lacquer is applied. R11 can be seen between the Camera header and a via near the SD card Vcc}
        \label{fig:PayloadImplementation}
\end{figure}

Due to an issue discovered between ordering and receiving the PCB, an 
additional 10k$\Omega$ resistor has been placed between Camera RX and 3V3 (R11 
on the schematic). Also, the Single In Line header holes (for the Port A 
expansion and camera headers) have been widened from 0.40mm to 0.80mm. An 
update to the PCB layout is provided in the delivered repository.

The camera will also be presented in a sealed, weatherproof container.

\section{System Integration}

