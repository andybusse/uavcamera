% This is part of the FinalReport document.
% Copyright (C) 2011 Piyabhum Sornpaisarn, Andrew Busse, Michael Hodgson, John Charlesworth, Paramithi Svastisinha
% See the file COPYING in FinalReport/ for copying conditions.

%% ----------------------------------------------------------------
\chapter{Implementation - Payload}
%% ----------------------------------------------------------------
\label{chap:implementation}

\section{Overview (mh)}

The payload controller module is an important part of the system, responsible
for interfacing with the camera module and communicating with the ground 
station image viewer software via the autopilot. Since this single module 
encapsulates a significant amount of the complexity of the project it was 
deemed sensible to split it up into sub-modules, which could be worked on in
parallel by different members of the team. 

With this in mind the payload controller module was split into four main 
submodules:

\begin{itemize}
	\item Camera Module Communication
	\item Communication with Ground Station via Autopilot
	\item SD Card Image Buffering
	\item Progressive JPEG Manipulation
\end{itemize}

%||||||| INCLUDE REFERENCES TO EACH ||||||||||

%||||||| INCLUDE SUBMODULE DIAGRAM OF THE PAYLOAD MODULE |||||||

% This is part of the FinalReport document.
% Copyright (C) 2011 Piyabhum Sornpaisarn, Andrew Busse, Michael Hodgson, John Charlesworth, Paramithi Svastisinha
% See the file COPYING in FinalReport/ for copying conditions.

\section{Camera Module (jc)}
\label{sec:John_Implementation}

The first step in the implementation of the camera module was to verify the communication with the camera by talking to it via a pc (milestone \ref{sec:ms_img_from_cam}). Once this step was complete the next step was to implement communications between the camera and a microprocessor, with the pc for debugging. With the microcontroller able to communicate with the camera this module was ready for integration with the payload (milestone \ref{sec:ms_img_cam_controller_implementation}).

\begin{figure}[H]
        \centering
        \includegraphics[width=1.00\textwidth]{figures/CameraModuleBlock1.png}
        \captionof{figure}{Block diagram of the camera module showing where communication takes place}
        \label{fig:camera_block}
\end{figure}

Figure \ref{fig:camera_block} shows how the camera module is broken down separate to the rest of the system. The camera and the camera controller communicate to set up the camera and get images, the camera controller then stores the image to the SD card and debug messages are sent to the computer during these processes so that the camera is less of a black-box.

During the development of the system 2 cameras (uCam \cite{ucam_datasheet}) died on us. The first very early on in development and the second after the system had been effectively finished apart from a few minor tweaks and implementation of some low priority functionality. These component malfunctions were most likely due to the uCam being sensitive to static so after the first one failed we were careful to only handle the second one on an anti-static mat whilst wearing a wrist-strap, however someone must have gotten careless towards the end of the project.

\subsection{First camera}
\label{sec:cam_prob}
The first attempt at communicating with the original camera was done using the arduino uno microcontroller board. The arduino was chosen as the platform for this first communication because it is simple to program and its programming environment provides various useful libraries, for example for serial communication \cite{arduino_serial_library}. The hardware also provides an easily accessible serial port.

The first implementation of this code managed to occasionally sync (see figure \ref{fig:syncProto}) with the camera (milestone \ref{sec:ms_basic_cam_comm}) but would generate errors when trying to send any further commands. After examining the code and the camera's datasheet it was discovered that the camera was only syncing occasionally because the arduino was using a slower baud rate than the camera could auto-detect, after increasing the baud rate on the arduino the camera would sync every time but the code still generated errors after this point.

It was at this point that the first camera broke. It was no longer syncing (or sending back anything at all, verified by oscilloscope readings), where before it had been doing so reliably and so it was checked with the 4D systems software and shown not to be working \ref{sec:existing_software_test}.

\subsection{Second camera}

Once it was established that the first camera was dead a couple of cheaper surface mount camera modules were purchased as possible replacements however making a successful physical (and therefore also communication) connection to these components proved excessively difficult so a new camera of the same type as the previous one \cite{ucam_datasheet} was ordered. The successfull connection test is seen in section \ref{sec:basic_connection_test}, with debug info observed using a serial monitor.

\subsubsection{Serial cable to computer}

The operation of the new camera module was verified using a usb to serial cable connected to a pc which was running the sample program which was provided by the camera's manufacturer \cite{ucam_test_software}. Using this set up it was shown that the camera was working and that it was possible to get images from it, see section \ref{sec:existing_software_test}.

%[] diagram of pin connections []

\subsection{Arduino implementation}
\label{sec:arduino_imp}

With the operation of the camera verified it was reconnected to the arduino board and the code run again with the same result as before: it would correctly sync (see figure \ref{fig:syncProto} and section \ref{sec:basic_connection_test}) but would generate errors when trying to send any further commands. On inspection of the code it became clear that this was because the same serial line was being used for both communication with the camera and debug messages and that these debug messages were interfering with further communications. Debug messages were therefore moved to a software serial line and sent to the computer via the usb to serial cable and observed using a serial monitor.

The code is broken down so that there are functions for each type of command that the camera can receive, functions that verify the responses from the camera and functions that combine these together in the correct sequence in order to perform a useful task with the camera.

\subsubsection{Camera synchronisation}

The first task in communicating with the camera is to synchronise the serial channel, the uCam datasheet \cite{ucam_datasheet} gives the correct protocol to do this.

\begin{figure}[H]
        \centering
        \includegraphics[width=1.00\textwidth]{figures/SyncProtocal.png}
        \captionof{figure}{Command protocol for synchronisation, sourced from \cite{ucam_datasheet} }
        \label{fig:syncProto}
\end{figure}

It should be noted that in \ref{fig:syncProto} the first sync sent from the host will be repeated until an acknowledgement is received, with everything working correctly this process usually takes 3 or 4 syncs before an ACK is received.

The inclusion of a separate debugging line allows for messages at each stage of this process to verify that it is connecting correctly and to indicate where any errors may have occurred. With all debug messages enabled this function will output "Sending syncs" followed by a "." for each sync command sent and then "ACK received" and "SYNC received", the main code will then output a message to the effect that contact has been successfully established.

\subsubsection{Taking a snapshot}

With the camera successfully synchronised the controller needs to be able to trigger the camera to take a photograph and then retrieve said photograph, again the uCam datasheet \cite{ucam_datasheet} gives an example of how to implement this.

\begin{figure}[H]
        \centering
        \includegraphics[width=1.00\textwidth]{figures/SnapshotProtocal.png}
        \captionof{figure}{Command protocol for taking and retrieving a jpeg snapshot at 640x480 resolution, sourced from \cite{ucam_datasheet} }
        \label{fig:snapProto}
\end{figure}

It should be noted in \ref{fig:snapProto} that the values in the commands are for the example and are not necessarily the values used in the code.

The "INITIAL" command has parameters that allow the image type and resolution to be set, in \ref{fig:snapProto} the image type is set to JPEG and the resolution is set to 640x480, these values are kept as the standard in the code.

The "SET PACKAGE SIZE" command is only relevant for jpeg images and set the size of the data packages that are sent when one requests an image. In \ref{fig:snapProto} this value is set to 512 bytes however in the code this is set to 64 bytes as it was originally thought that the system might have to transmit these packages as soon as receiving them so they needed to be a reasonable size to fit in with the timing between transmit packets sent by the autopilot.

The "SNAPSHOT" command tells the camera to take a single picture, its parameter sets whether the image is jpeg compressed or raw, in \ref{fig:snapProto} this is set to be a jpeg image and this is left as the default in the code.

The last command sent by the controller is the "GET PICTURE" command which tells the camera to send image data, it has one parameter which sets which image is sent. In \ref{fig:snapProto} this is set to send the snapshot image and this is also how it is set in the code. As well as sending back an ACK this function will also send back a "DATA" message which tells the controller what type of image is being returned and how large it is.

Once the controller has acknowledged the "DATA" message the camera starts to send data packages, each of which needs to be acknowledged, until the entire image is transferred. Whilst this process is going on the data packages are saved to an SD-card as described in section \ref{sec:SD_imp}.

The use of the debug communication channel again allowed for detailed analysis of this process with the maximum amount of debug messages being one before each command is sent saying what that command is and another message once and ACK is received, there is also then a message with the size of the data in and then messages with the package number and the byte number within that package, the data can also be output to the debug channel, see section \ref{sec:image_capture_test}.

\subsection{Miscellaneous Problems}
\label{sec:misc_cam_probs}

Before the SD-card was implemented the sending of the data directly via the debug channel was attempted and so was its reconstruction into a jpeg image on the computer, this could not be made to work however, see section \ref{sec:image_capture_test}. On looking at the data as it came off the arduino via the software serial debug line it became clear (particularly when looking at the counting values, whose values were already known) that the data was not being transmitted reliably. This could have been because of noise on the line, the fact that the software serial line is communicating via pins on the arduino that aren't optimised for the function or simply failings in the arduino software serial library \cite{software_serial}. For this reason this method of getting an image was abandoned, this also had to be taken into consideration whilst thinking about the integration of this part of the system: the software serial line is not a reliable method of transmitting the data so some other method would have to be used.

Another problem that was noticed after the camera control had been integrated into the payload controller, as described in section \ref{sec:int_pl}, was that every other time the system was powered up the camera would require a reset/power cycle. This was discovered to be because the Rx wire to the camera was left floating, this was fixed by the addition of a 10k$\Omega$ pull-up resistor, see appendix \ref{appendix_schematics} for schematics.

It was also noticed quite early on that the camera required 3.3V logic \cite{ucam_datasheet} levels but the arduino outputs 5V logic levels, the arduino could receive logic at 3.3V easily enough but the 5V output was causing occasional errors. In order to fix this problem the 5V output from the arduino was taken through a potential divider of a 10k$\Omega$ resistor and a 3V zener diode.


\section{Interfacing the Arduino with an SD card (ab)}
\label{sec:SD_imp}

Interfacing the arduino with an SD card is a rather trivial process. Using an Arduino Uno, a multiple-size SD card socket from Zepler stores, and the 
"ReadWrite" test program provided in the arduino-022 IDE, we connected the following Arduino pins to the following SD card pins:

\begin{itemize}
\item Arduino GND - SD GND (pins 3 and 6)
\item Arduino 3V3 - SD Vdd (pin 4)
\item Arduino Digital pin 11 - SD MOSI (pin 2)
\item Arduino Digital pin 12 - SD MISO (pin 7)
\item Arduino Digital pin 13 - SD SCLK (pin 5)
\item Arduino Digital pin 4 - SD CS (pin 1)
\end{itemize}

Communication with the SD card only works in SPI mode, unfortunately the built-in 
SD card library does not support 1-wire or 4-wire SD mode.

\subsection{File Naming System}

This example program is useful, but only allows us to write to one file name at 
a time. Therefore, a method was written \ref{arduino_captureTest}, lines 62-68 that detects all of files present on the SD card, increments the filename, and 
writes the new data to that filename.


\section{Payload Controller}
\subsection{General Payload Controller Implementation OR MAYBE Overview}
||||||| Might not want this in its own section ||||||||

The payload controller module is an important part of the system, responsible
for interfacing with the camera module and communicating with the ground 
station image viewer software via the autopilot. Since this single module 
encapsulates a significant amount of the complexity of the project it was 
deemed sensible to split it up into sub-modules, which could be worked on in
parallel by different members of the team. 

With this in mind the payload controller module was split into four main 
submodules:

\begin{itemize}
	\item Camera Module Communication
	\item Communication with Ground Station via Autopilot
	\item SD Card Image Buffering
	\item Progressive JPEG Manipulation
\end{itemize}

||||||| INCLUDE REFERENCES TO EACH ||||||||||

||||||| INCLUDE SUBMODULE DIAGRAM OF THE PAYLOAD MODULE |||||||


\subsection{Communication with Ground Station via Autopilot}
Considering the overall aim of this project: to produce a system by which 
images can be downloaded over-the-air from a payload module to a ground 
station, it is fair to say that some method of communicating between the
payload module and ground station are an essential component in the system.

The specification - see chapter \ref{chap:specification} - requires the payload
module to communicate with the ground station using the autopilots payload
module interface (discussed in section \ref{sec:autopilot_payload_interface}
below).

%To better explain the protocol used we will split the explanation into two 
%sections: a \emph{Autopilot Payload Interface} section describing the
%pre-existing autopilot payload interface on which we are building the 
%protocol and a \emph{UAV Camera Communication Protocol} section describing the 
%protocol we have implemented as a part of this project.

\subsubsection{Autopilot Payload Module Interface}
\label{sec:autopilot_payload_interface}
|||||| Maybe have this in background research ||||||

The SkyCircuits autopilot module allows extension modules named `payload 
modules' to be connected to the autopilot. These payload modules are connected 
via a RS485 serial connection at 38.4 kBaud, allowing several payload modules
to be connected at once in a daisy-chain configuration. All payload modules are 
connected to common TX and RX lines, where the RX line is used by the
autopilot to send commands and data to the payload modules, and the TX
line is used by all daisy-chained modules to communicate with the autopilot.

||||||| Include diagram of daisy chain configuration ||||||||||

Since the TX line is shared between all modules only one payload module can be 
transmitting at once over the link, with all other payload modules required to
leave the line tristated. This means that each payload module must know when it is
allowed to use the transmit line so as not to clobber any other payload module.
In this system this is achieved by the use of `transmit tokens' handed out 
by the autopilot over the RX line. A `transmit token' is sent to each payload 
module in turn, informing the module that it is clear to transmit data. With 
only one payload module connected to the autopilot, these tokens are sent out 
every 20 ms. |||||| CHECK THIS ||||||||

This two way communication link is used to implement a command interface to 
the autopilot. A payload module can execute commands on the autopilot, 
allowing a variety of useful and interesting possibilities for payload
module design. Of interest to us however, is the ability of the payload
module to set shared memory. Since this shared memory can be accessed 
through the ground station software this allows us to send data through the
autopilot link to the ground station. Shared memory is allocated to 
each payload module, accessible on the ground station using the command:
~\\
\begin{lstlisting}[caption={Accessing shared memory from ground station}, label=lst:gs_shared_mem_set]
payload[payload_num].mem_bytes[mem_bytes_num]
\end{lstlisting}

Where \emph{payload\_num} is the ID number identifying the payload and 
\emph{mem\_bytes\_num} is the index of the set of shared memory to be accessed.

Each shared memory set is of variable length and can be set from the payload 
module using the following function (code provided by customer):
~\\
\begin{lstlisting}[language=C, caption={Setting shared memory from payload module}, label=lst:payload_shared_mem_set]
send_set_class_indexed_item_indexed(CLASS_PAYLOAD, module_id, 
CLASS_PAYLOAD_MEM_BYTES, mem_bytes_num, message_to_send,
message_to_send_length)
\end{lstlisting}

Where \emph{CLASS\_PAYLOAD} and \emph{CLASS\_PAYLOAD\_MEM\_BYTES} are constants 
informing the autopilot that the \emph{mem\_bytes} item of the \emph{payload} 
object should be set, \emph{module\_id} is the ID of the payload module,
\emph{mem\_bytes\_num} is the same as used in listing
\ref{lst:gs_shared_mem_set} and \emph{message\_to\_send} is the message to be 
sent (consisting of an array of length \emph{message\_to\_send\_length}, the 
first element of which should be the number of bytes to set in the shared 
memory).

The method through which this shared memory is accessed via the ground station
image viewer is discussed in section ||||||||| GS IMAGE VIEWER |||||||||.

The ground station software can also send data directly to a payload module 
through the autopilot, where it will be sent on to the RX line of the RS485
bus. The \emph{send\_bytes} command is used to do this, as can be seen in 
listing \ref{lst:ground_station_send_bytes}. 

As discussed in section |||||||| REF |||||||| it was decided that 
our communications protocol would use shared memory and \emph{send\_bytes} 
commands, allowing two way communications between the payload controller and 
ground station software to be established.

\subsubsection{UAV Camera Communication Protocol}
The Payload Module Interface discussed above (section \ref{sec:autopilot_payload_interface})
allows us to send strings of bytes in both directions. However, in order to 
communicate with the ground station image viewing software some form of 
additional communications protocol is required so that both ends of the link 
are communicating in a mutually understandable manner.

This two way communications is the interface between the payload and the 
ground station software, so some standard protocol was required. It was decided
that a message based system would be used, with the messages from the ground
station to the payload module being sent using \emph{send\_bytes} and the 
messages sent from the payload to the ground station being put into shared
memory. Each message is composed of two elements, one byte for the message ID 
- unique to each type of message - and a variable number of data bytes 
(depending on the message type.) The different message types are detailed 
below:

\paragraph{Messages sent from Ground Station To Payload}

\begin{itemize}
\item \textbf{Take Picture}
\begin{itemize}
\item \emph{Data:} None
\item Prompts payload module to capture an image and save it to the SD card.
\end{itemize}

\item \textbf{Image Download Request} 
\begin{itemize}

\item \emph{Data:} Image ID
\item Requests the payload send the image with ID \emph{Image ID} to the 
ground station. This message allows any image stored by the payload module 
to be downloaded over the connection, increasing flexibility. 
\end{itemize}

\item \textbf{Configure Camera}
\begin{itemize}
\item \emph{Data:} Colour Type, Raw Image Resolution, JPEG Image
Resolution
\item Sets the image resolution and colour mode of the camera. Only the 
JPEG mode has been tested so far.
\end{itemize}

\end{itemize}

\paragraph{Messages Sent from Payload to Ground Station}

\begin{itemize}

\item \textbf{Picture Taken}

\begin{itemize}
\item \emph{Data:} Image ID

\item Informs the ground station software that an image has been taken and 
saved to the SD card. \emph{Image ID} is the ID of the image that has been 
saved to the SD card.
\end{itemize} 

\item \textbf{Image Download Info}

\begin{itemize}

\item \emph{Data:} Number of Image Packets

\item Sent by the payload after a successful \emph{Image Download Request}
message from the ground station. Informs the ground station how many 
\emph{Image Data} packets to expect.
\end{itemize}

\item \textbf{Image Data} 
\begin{itemize}
\item \emph{Data:} Packet Number, Image Data
\item This message contains an amount of actual image data. Sent after a
\emph{Image Download Info} message which is in turn in response to an 
\emph{Image Download Request} message. The whole image is sent over
\emph{Number of Image Packets} packets (as defined by the \emph{Image Download
Info} message.) \emph{Packet Number} informs the ground station which of these
packets the message is carrying. \emph{Image Data} contains the actual image 
data for this packet and is variable size, with a maximum size of 50 bytes. 
\end{itemize}

\end{itemize}


\subsubsection{Existing Code}
\label{sec:payload_existing_code}
Our customer had provided us with some payload module communication AVR code
- written for a ATMega168 - for communicating with the autopilot. This code
was the basis on which the payload controllers communication link was built.

The code provided a number of useful utilities:
\begin{itemize}
\item Ability to set shared memory on the autopilot.

\item Ability to receive messages sent from the Ground Station to the 
autopilot.

\item Example code for setting shared memory on the autopilot.
\end{itemize}

This base code was modified slightly after a bug was found in its handling of 
the transmit enable signal. The RS485 communication protocol used for the 
autopilot-payload link (as described in section |||||||| SEC ||||||||) 
requires a `transmit enable' signal to be asserted when the payload is 
transmitting. This signal should be asserted just before data is to be sent 
and cleared just after. However, the original payload base code cleared this 
signal in an interrupt service routine (ISR) which fired after the transmit 
buffer of the UART was ready to accept new data. Since this transmit buffer 
would be ready to accept new data before the data was actually sent over the
physical connection this lead to the transmit enable signal being cleared
before all data had been sent, causing strange behaviour on the RS485 link.
This bug did not seem to cause any problems, and the odd behaviour was only
noticed when testing the system with an oscilloscope. ||||| INC TRACES ||||||
It was considered sensible to fix the bug in case it did cause problems later.

The fix for this problem was reasonably simple: a new ISR was set up which 
fired only when the current transmission had actually completed, and the 
command to clear the transmit enable signal was moved into this ISR.
||||| INC AFTER TRACE ||||||


||||||||| FLOWCHART OF INTERACTIONS |||||||

\subsection{Payload Debug Interface}
\label{sec:payload_debug_interface}
While the payload module was being developed debugging was a major concern. 
Having access to an oscilloscope very useful when debugging some particularly 
difficult problems, as described elsewhere in this report ||||||| WHERE |||||.
However, while building the payload module it was clear more complex details 
about the internal state of the program as it was running would be useful.

While prototyping the camera module on the Arduino platform this problem was 
solved by using the \emph{SoftwareSerial} library (for information regarding 
the library see \cite{software_serial}) to send textual debug information via
a serial to USB cable to PC. This proved very useful when prototyping camera
communications.

A slightly different approach was taken when using the ATMega644P. Initial 
testing of the software serial library on this platform suggested that it 
would not work without a significant investment in time fixing it. It was
therefore decided that it would be faster to send debug messages over 
`bit-banged' SPI to a spare Arduino board which then forwarded this data to 
the connected PC over serial. Again, having the use of this debug interface 
proved very useful while implementing the system, and was well worth the 
small amount of extra time taken to develop it.

||||| INC EXAMPLE OF DEBUG TEXT (MAYBE SCREENSHOT) ||||||

\subsection{Integration of Payload Sub-Modules}

\subsubsection{REMINDER: DISCUSSION OF ARDUINO MULTIPLEXED SERIAL}

\subsubsection{Implementation on ATMega644P}
Initially we planned to implement the system on the Arduino prototyping 
platform with an expansion `shiled' board providing any extra hardware needed -
The reasons for this are discussed in section |||||| REF |||||||. 
This plan relied on the implementation of the multiplexed serial solution 
discussed above in section |||||||| REF ||||||||. 

Unfortunately, due to time constraints alongside setbacks such as those
discussed in sections ||||||| REF AUTOPILOT BUG |||||||| and |||||||| REF 
CAMERA FAILURE |||||||| it was decided that implementing the system on the 
Arduino platform may not the best way to proceed, since it would almost 
certainly be non-trivial to implement and debug.

Several options were considered alongside the multiplexing option. One 
alternative workaround discussed was to use a separate AVR as an interface
to the autopilot and connect it to the Arduino using bit-banged SPI or the
inbuilt two-wire interface (TWI, sometimes known as I2C). This approach was 
attractive as it meant we could use the existing Arduino code for the camera 
and SD card and use the existing ATMega168 code for the autopilot communication.
However, this solution would add another complex component to the system along
with another communications link, increasing complexity significantly.

Another option considered was using an AVR device with multiple UARTs so one 
could be used for the camera and another for the autopilot link. However, 
the code we had written for interacting with the SD card buffer relied upon 
some Arduino specific libraries. Finding new libraries and rewriting this code 
would be time consuming and inefficient, especially since we had already 
invested time in creating a working SD card solution.

Further investigation into the Arduino SD card library code (for more info
regarding the library see reference \cite{arduino_sd_library}) suggested that
it was coded to work with the ATMega644P chip as well as the ATMega168 and
ATMega328P. Both the ATMega168 and ATMega328P feature only one UART, but the 
ATMega644P features two, making it attractive for use in this project (see
\cite{atmega644p} for more info about the ATMega644P).

Since the ATMega644P provides two UARTs and the SD card library we were using
supported this chip it was decided that using the 644P would be a sensible 
way forward. 

It is important to note that frequent progress reviews meant that this problem
was caught early before it could become more damaging.

As mentioned above in section \ref{sec:payload_existing_code} the code provided
to us for communication with the autopilot was written for an ATMega168 chip. 
Porting this code and the existing Arduino camera code to the 644P was not a 
trivial task, although it was simplified by the use of the Sanguino library -
a port of the Arduino core libraries to the 644P (see \cite{sanguino}).

\section{REMINDER: PROBLEMS CHALLENGES}
\section{REMINDER: HOW DID WE SOLVE PROBLEMS, DEBUGGING, TESTING, ETC}
\section{REMINDER: JOHN COULD TALK ABOUT HOW BROKEN CAMERAS SLOWED DOWN DEVELOPMENT BUT GOOD PLANNING AND CONTINGENCY MINIMISED RISK OR IN MANAGEMENT SECTION}
\section{REMINDER: FUTURE WORK}


\section{JPEG Information Extractor}

\subsection{Progressive Scan of JPEG (commandCheck)}

After receiving the size of the JPEG file (in bytes), 
the JPEG Information Extractor starts reading the bytes of the JPEG stored in the SD card. 
Because the information will not all be available instantly, 
the extractor uses a class to read the next byte in the input stream, one at a time. This class \textbf{read\_byte()} is 
how the JPEG extractor gains access to the JPEG input stream. 

As the extractor reads in the bytes from the input stream, it checks to see if the byte indicates the start of a header (0xff). 
If this is the case, the next byte identifying the header is read and the \textbf{JPEGMethod} class is 
called which uses a switch command to determine the actions to take. 
If not, the data is discarded and the extractor continues to read the next byte is read from the input stream. 

\subsection{Image Data Extraction (JPEGMethod)}

The JPEG Information Extraction class does not need all the possible information that can be extracted from a JPEG image. 
The few image headers containing important information store the important information from the input stream into 
variables which will be sent to the custom encoding algorithm. 

If the header is not recognized by this class, 
it is assumed to not have any important information and is ignored; 
the code continues to scan through the input stream for the next header byte. 
The following headers have  information to be stored (all information is considered to be stored in one byte unless stated otherwise):

\paragraph*{SOF0: Start Of Frame (0xc0)}
\begin{description}
	\item[Data precision] indicates how many bits compose one sample (generally 8 to indicate a byte).
	\item[Image height] indicates the height of the image in pixels (2 bytes long).
	\item[Image width] indicates the width of the image in pixels (2 bytes long).
	\item[Number of components] indicates how many components define one pixel. 
		This will always be three for our cases to indicate the use of the YCbCr colour model. 
		The component information is read in in the order Y, Cb, Cr.
	\item[Component ID] indicates the ID number used by the JPEG file for this component.\footnote{One for each component}
	\item[Sampling factor] both horizontal and vertical are stored within a single byte. 
		The least significant 4 bits indicate teh vertical sampling factor, 
		while the most significant 4 bits indicate the horizontal sampling factor.\footnotemark[1] 
	\item[Quantization Table Number] associates the component to a quantization table.\footnotemark[1] 
\end{description}

\paragraph*{APP0: JFIF application segment (0xe0)}
\begin{description}
	\item[File Identifier Mark] indicates that this is a JFIF standard image.
	\item[Major Revision Number] must be checked to be 1.
	\item[Minor Revision Number] 
	\item[Units for x/y Density] 
	\item[X Density]2 bytes
	\item[Y Density]2 bytes
	\item[Thumbnail Width] Not used.
	\item[Thumbnail Height] Not used.
	\item[Remaining bytes] contains information regarding the image's thumbnail. This is not used in our extractor.
\end{description}

\paragraph*{DHT: Define Huffman Table(s) (0xc4)}
\begin{description}
	\item[HT Information] contains the following information concerning the indicated Huffman Table in one byte:
		\begin{description}
			\item[Bits 0\ldots3] ID number of the HT (cannot exceed 3)
			\item[Bit 4] Indicates whether the HT is DC (0) or AC (1)
			\item[Bits 5\ldots7] Unused.
		\end{description}
	\item[Number of Symbols] is a string of 16 bytes, each byte containing the number of symbols with 
		codes of length 1\ldots16. For example, if the third byte of the string is 7, 
		there are 7 signals of code length 3 in this Huffman Table.
	\item[Symbols] is a string of \emph{n} bytes, where \emph{n} is 
		the sum of all the values in the previous 16-byte string. 
		The extractor stores the Huffman tables in a linked list of struct objects representing the Huffman Tables.
\end{description}

\paragraph*{DQT: Define Quantization Table(s) (0xdb)}
\begin{description}
	\item[QT Information] contains the following information concerning the indicated Quantization Table in one byte:
		\begin{itemize}
			\item[Bits 0\ldots3] ID number of the QT (cannot exceed 3).
			\item[Bits 4\ldots7] Precision of the QT. Either 8-bit (0) or 16-bit (all other values).
		\end{itemize}
	\item[QT values] is a string of \emph{n} bytes, where \emph{n} is $64*$precision$+1$.
\end{description}

\paragraph*{SOS: Start Of Scan (0xda)}
\begin{description}
	\item[Number of components in scan] indicates how many components define one pixel in the scan. 
		This is usually identical to  the number of components obtained from the SOF segment.
	\item[Component ID] indicates the ID number used by the JPEG file for this component.\footnotemark[1] 
	\item[Huffman Table Indicator] stores the AC and DC Huffman Table ID number associated with this component in one byte. 
		The AC Table number is stored in bits 0\ldots3. The DC Table number is stored in bits 4\ldots7.\footnotemark[1]
\end{description}

\subsection{Image Data Treatment}

The SOS Header information extraction segment is followed by 
an algorithm which sorts the image data into chrominance pixels before 
sending the chrominance pixel to the custom JPEG encoding class. 
If we consider \emph{n} to be the number of image pixels in a chrominance pixel, then 
we associate the first $n - 2$ bytes with the Y (luma) Huffman tables, the $n - 1$ byte with the Cb Huffman tables, and 
the $n^{th}$ byte with the Cr Huffman Table. 
This continues until the EOI header is found, or until there is no more bytes coming in from the input stream.


\section{Integration of Payload Sub-Modules (mh)}
\label{sec:int_pl}
%|||||||| REMINDER: DISCUSSION OF ARDUINO MULTIPLEXED SERIAL

\subsection{Integration of Camera Module with SD Card}
Both the camera module and SD card module were initially prototyped and 
partially tested separately,
%|||||| REFS ||||||||
but were integrated at
a fairly early stage. This integration was the easiest way of testing the 
camera module, since the images from the camera could be written to the SD
card, allowing for easy inspection.

\subsection{Integration of Camera Module and SD Card with Payload-Ground Station Communication}
Integrating the camera module with the payload to ground station communications
was a major milestone for the project, since completing it would mean we would
have a basic working system. It was also one of the most challenging parts of
the project. This step was performed iteratively over time, with the most basic
features being implemented, integrated and tested before more advanced features
were attempted

Integration of the Payload-Ground Station communications with the Camera module
was completed on the ATMega644P hardware, please see section \ref{sec:atmega644p_impl}
for more info about this aspect and an idea of the challenges and problems
faced.

\subsection{Integration of Progressive JPEG Manipulation}
Unfortunately due to time constraints we were not able to integrate the
progressive JPEG manipulation code into the payload module, please see section \ref{sec:implementation_progressive_jpeg} for more information.

\subsection{Implementation on Arduino}
\label{sec:implementation_arduino}
Initially we planned to implement the system on the Arduino prototyping 
platform with an expansion `shield' board providing any extra hardware needed -
the reasons for this are discussed in section \ref{sec:appr_considered_arduino}.
The camera module connection code had also already been developed and tested on
the Arduino. 

This plan relied on the implementation of a multiplexed serial solution, where
the single serial port of the Arduino would be used to connect to two different
devices (the camera and the autopilot) at different baud rates and message
formats. It was planned that this would be achieved by the use of tristate
buffers to form a bus with the two devices TX and RX lines. The payload software
could then take care of altering the message format and speed when needed.

Unfortunately, due to time constraints alongside setbacks such as those
discussed in sections \ref{sec:probls_pl} and \ref{sec:cam_prob}
 it was decided that implementing the system on the 
Arduino platform may not the best way to proceed. The main reason for this 
decision was the number of unknowns in the planned multiplexed implementation
- since it was not really how the hardware is designed to be used, it was not
certain it would provide satisfactory results in the limited time that we had.

\subsection{Implementation on ATMega644P}
\label{sec:atmega644p_impl}
Several options were considered alongside the multiplexing option discussed above. One alternative workaround discussed was to use a separate AVR as an interface to the autopilot and connect it to the Arduino using bit-banged SPI or the inbuilt two-wire interface (TWI, sometimes known as I2C). This approach was 
attractive as it meant we could use the existing Arduino code for the camera 
and SD card and use the existing ATMega168 code for the autopilot communication.
However, this solution would add another complex component to the system along
with another communications link, increasing complexity significantly.

Another option considered was using an AVR device with multiple UARTs so one 
could be used for the camera and another for the autopilot link. However, 
the code we had written for interacting with the SD card buffer relied upon 
some Arduino specific libraries. Finding new libraries and rewriting this code 
would be time consuming and inefficient, especially since we had already 
invested time in creating a working SD card solution.

Further investigation into the Arduino SD card library code (for more info
regarding the library see reference \cite{arduino_sd_library}) suggested that
it was coded to work with the ATMega644P chip as well as the ATMega168 and
ATMega328P. Both the ATMega168 and ATMega328P feature only one UART, but the 
ATMega644P features two, making it attractive for use in this project (see
\cite{atmega644p} for more info about the ATMega644P).

Since the ATMega644P provides two UARTs and the SD card library we were using
supported this chip it was decided that using the 644P would be a sensible 
way forward. 

It is important to note that frequent progress reviews meant that the problem of how to implement this integration and the need for 2 serial lines was caught early and controlled before it could become more damaging.

As mentioned above in section \ref{sec:payload_existing_code} the code provided
to us for communication with the autopilot was written for an ATMega168 chip. 
Porting this code and the existing Arduino camera code to the 644P was not a 
trivial task, although it was simplified by the use of the Sanguino library -
a port of the Arduino core libraries to the 644P (see \cite{sanguino}). Some 
changes were made the the Sanguino libraries for use in our project, for 
example the Sanguino handling code for one of the UARTs was disabled to allow
it to be configured manually for autopilot communications. Some small pieces
of code that were causing compilation issues were also modified.

\section{Debug Interface (mh)}
\label{sec:payload_debug_interface}
While the payload module was being developed debugging was a major concern. 
Having access to an oscilloscope very useful when debugging some particularly 
difficult problems.
However, while building the payload module it was clear more complex details 
about the internal state of the program as it was running would be useful.

While prototyping the camera module on the Arduino platform this problem was 
solved by using the \emph{SoftwareSerial} library (for information regarding 
the library see \cite{software_serial}) to send textual debug information via
a serial to USB cable to PC. This proved very useful when prototyping camera
communications.

A slightly different approach was taken when using the ATMega644P. Initial 
testing of the software serial library on this platform suggested that it 
would not work without a significant investment in time fixing it. It was
therefore decided that it would be faster to send debug messages over 
`bit-banged' SPI to a spare Arduino board which then forwarded this data to 
the connected PC over serial. Again, having the use of this debug interface 
proved very useful while implementing the system, and was well worth the 
small amount of extra time taken to develop it.

Figure \ref{fig:debud_conn} shows an example of this debug interface seen on
a serial terminal on a PC.

\begin{figure}[H]
        \centering
        \includegraphics[width=1.0\textwidth]{testing_screenshots/camera_image_saving_sd_card_test.png}
        \captionof{figure}{Debug text sent from the payload over the debug interface.}
        \label{fig:debud_conn}
\end{figure}


%\section{Development Testing and Debugging (mh)}
%The payload module and submodules were tested throughout development in order 
%to ensure the latest changes and features added were working as expected. 
%This allowed bugs to be caught and fixed as quickly as possible, as well as
%increasing our knowledge of the current state and progress of the system and
%project as a whole.

% This is part of the FinalReport document.
% Copyright (C) 2011 Piyabhum Sornpaisarn, Andrew Busse, Michael Hodgson, John Charlesworth, Paramithi Svastisinha
% See the file COPYING in FinalReport/ for copying conditions.

\section{Physical Implementation (ab)}
\label{sec:PCB-implementation}

The final, delivered module is a $88mm\times62mm$ PCB. The PCB manufactured 
for the project was done so for free, using Spirit Circuits' "Go Naked"
service \cite{go-naked}. The PCB itself is a "tracks and holes" only service 
- no soldermask or silkscreen is applied. The schematic of the circuit 
delivered is available in Appendix \ref{Payload_Schematic}, and of the PCB layout in Appendix 
\ref{appendix-layout}. A waterproof lacquer will be applied to the PCB to prevent condensation 
from shorting tracks together, and the module itself presented in a 
waterproof container before flight testing.

A quote of \pounds 56.39 has been obtained from PCB Train \footnote{\url{http://www.pcbtrain.co.uk/quote-and-order-pcb/}} 
for a single PCB, with electrical testing, on a 15 day lead time. 
Please note that Spirit Circuits have only manufactured our prototype 
free of charge on a one-off basis. Being impressed with this service, 
we may recommend that clients contact them for a competitive quote.

\begin{figure}[H]
        \centering
        \includegraphics[width=1.00\textwidth]{figures/PayloadImplementation.png}
        \captionof{figure}{Image of the final payload, under test, before lacquer is applied. R11 can be seen between the Camera header and a via near the SD card Vcc.}. 
        \label{fig:PayloadImplementation}
\end{figure}

In figure \ref{fig:PayloadImplementation}, the letters refer to the following 
features of the payload module:

\begin{itemize}
\item \textbf{A:} microSD Card slot
\item \textbf{B:} RJ45 socket, more popularly known as an ethernet socket. 
(Actually implements the RS485 protocol)
\item \textbf{C:} Power LED. Shows whether 3V3 from the ethernet socket is connected.
\item \textbf{D:} ISP programming header socket. As seen, this is not 
soldered on, as we have been using the JTAG header for testing, but will be soldered 
before delivery to the customer.
\item \textbf{E:} Camera header. The camera can be connected to our PCB 
simply by attaching the hook-up wire in the correct order. (L-R: 3V3, GND, 
Camera TX, Camera RX)
\item \textbf{F:} MAX489 (RS485 Transceiver) \footnote{\url{http://uk.farnell.com/9725148}}. 
Cheaper version of the MAX3070 used in the sample peripheral.
\item \textbf{G:} Power header. L: 3V3, R: GND.
\item \textbf{H:} ATmega644P \cite{atmega644p}
\item \textbf{I:} ATmega644P Port A expansion header. (In the image, this is 
being used as a connection to an Arduino Uno so that we may view the debug 
information)
\item \textbf{J:} JTAG programming header
\item \textbf{K:} Reset button
\item \textbf{L:} Debug LEDs
\end{itemize}


%% ----------------------------------------------------------------
\chapter{Implementation - Ground Station (ps)}
%% ----------------------------------------------------------------

% This is part of the FinalReport document.
% Copyright (C) 2011 Piyabhum Sornpaisarn, Andrew Busse, Michael Hodgson, John Charlesworth, Paramithi Svastisinha
% See the file COPYING in FinalReport/ for copying conditions.

\label{chap:implementation_ground_station }

On the ground station the user only has access to the image viewer program.
The programmer makes use of both the data stream port and the console port to send commands to and receive JPEG photos from the UAV.

\section{The development process} 
Because the GUI includes many functions on the application and links to the TCP/IP ports, the GUI part consider as a very big program. 
Therefore, in order to complete the GUI step by step, a plan for the development is required.
The connection to the datastream can be done by using a console application described in section\ref{sec:testing_connection_send_to_stream}.
When the connection is established, the appliction can listen to the data stream port as in section\ref{sec:testing_receive_stream}.
These console applications will be incorporated into a larger program in later section \ref{completeSystem}. 

These are the development stages that the GUI has went through:

\flushleft
\begin{enumerate}

\item	Understanding what the customer wants. 

\item	Determining the hardware and software specifications.
 
\item	Designing the GUI.

\item  Developing a smaller program which simulates the UAV camera's signals in order to make the testing easier and less time consuming. 

\item	Learning the .NET class that can support the connection to TCP/IP, and communicate between the host and device.

\item	Using GUI to link the ground station software to access the UAV.

\item	Distributing the GUI.
\end{enumerate}

\section{Initial Use Case Diagram}
The use case diagram shows what functions the user can use in the program.
It includes all the specification that the customer wants for a complete system.
Figure~\ref{GUI_useCase} shows all possible actions the user can perform on the program, such as saving, opening and deleting any JPEG image from the computer. 
The user can also connect to the UAV if it is not already connected.
He can use the image viewer program to prompt the UAV payload to return an image which will be displayed on the picture box.
\begin{figure}[H]
\begin{center}
\includegraphics[scale=0.6]{figures/userCase.png} 
\end{center}
\caption{Use case diagram of the GUI\label{GUI_useCase}}
\end{figure}

\begin{figure}[!hbtp]
\begin{center}
\includegraphics[scale=0.7]{figures/FinaluserCase.png} 
\end{center}
\caption{Final use case diagram of the GUI\label{GUI_finalUseCase}}
\end{figure}

\section{The Design}

The user of our application is assumed to have limited programming experience, so the program will need to be simple to understand. Figure~\ref{ini_GUI} show the first prototype view of the GUI. 
During the downloading process, the application should stay active so that the cancel button can be used. 
The \texttt{Gallery} button will link to another page which will show a gallery of images taken. 
The \texttt{Left} and \texttt{Right} buttons can navigate the picture box to view an earlier picture or a later picture. 
The \texttt{Cancel} button will cancel the download of the receiving image, so that corrupted pictures can be cancel. 
The user mode of the application can access only the main features, such as taking picture, changing directory, and cancelling the download of a picture.  
It allows the user to choose the resolution and picture type (raw or JPEG) to be transmitted from the UAV to the ground station. However, the user doesn't have access to changing the command sent, changing the receiving data, and any interaction with the UAV because to avoid of any errors. 

\begin{figure}[H]
\begin{center}
\includegraphics[width=1.0\textwidth]{figures/initialGUI.png} 
\end{center}
\caption{The initial design of GUI\label{ini_GUI}}
\end{figure}
The GUI has been planned to have functions such as auto triggering, selecting the image type, the resolution type, and file path chosen, a progress bar, a help button, and stop and delete buttons. Figure \ref{finalGUI} is the screen shot of the final GUI.

\begin{figure}[H]
\begin{center}
\includegraphics[width=1.0\textwidth]{figures/finalGUI.png} 
\end{center}
\caption{final GUI\label{finalGUI}}
\end{figure}

\section{Final Use Case Diagram}
Figure \ref{GUI_finalUseCase} shows the final use case diagram.
The connect button was not implemented because the ground station will connect automatically to the UAV when the program is opened.
The raw image is too large and takes to long to send down the connection so it was not implemented.
Therefore, the Milestone\ref{sec:ms_pl_img_gs_cam_colour_type} will not be implemented.

\section{Class Diagram}
\subsection*{The Initial Class Diagram}
Figure~\ref{ini_Class} shows initial classes and methods of the image viewer program.
The \texttt{JPEGFileReader} Class has functions for decoding and encoding the JPEG file.
There will be a decoding/encoding algorithm because the image will take a long time to download to the ground station. 

\begin{center}
\begin{figure}[!hbtp]
\includegraphics[width=150mm,height=100mm]{figures/initialClassDiagram.png} 
\caption{The initial design of GUI classes\label{ini_Class}}
\end{figure}
\end{center}

The \texttt{DCT} class has many arithmetic operations and equations which have to be implemented on the image viewer program. 

The \texttt{Painting} class is supported by the \texttt{DCT} class. The intention of this class is to display an encoded image point by point on the \texttt{pictureBox}.
Using this method, the \texttt{pictureBox} can display an image from the first pixel transmitted.

The \texttt{CameraCommand} class is designed to send the data from the ground station to the camera. The idea is to make the camera sync with the payload by using ground station commands. The \texttt{SetAndSendCommand} class is used to set the byte command and then send it to the payload via the console port. The \texttt{SetGetPicture()},\texttt{SetInitial()}, \texttt{SetPhoto()}, and \texttt{SetTakePicture()} methods are used for setting the correct byte to send through the \texttt{SetAndSendCommand} class.

\subsection*{The Class Diagram}
The class diagram has been implemented very differently from the planned one.
This is because the new plan is to decode the image on board the UAV and then transmit that image to the ground station in a compressed JPEG file. 
The \texttt{DCT} class is not needed anymore because all the DCT calculation will be done onboard the UAV. 
The \texttt{CameraCommand} has been taken away because the payload will receive an image viewer command from the ground station and then the payload will send another different signal to the camera.
Therefore, the command sent to the camera from the payload does not have to be the same as the command sent from the ground station to the payload. 
The \texttt{Painting} class is used to draw each pixel onto the \texttt{pictureBox}, but it has not been implemented in the final program because the JPEG encoding will be done onboard the UAV.
Therefore, the milestone\ref{sec:ms_pl_img_gs_progressive_dl} will not be implemented.
This is also due to the time limitations.
More detail about the progressive image can be found in the section\ref{sec:implementation_progressive_jpeg}.
 
\begin{figure}[!hbtp]
\begin{center}
\includegraphics[scale=0.7]{figures/finalClassDiagram.png} 
\end{center}
\caption{Final class diagram of the GUI\label{GUI_finalClassDiagram}}
\end{figure}

\section{Before Connection to the Image Viewer Program}
Figure \ref{schemetic_clipA} shows a diagram of how the connection of the hardware should be. 
The UAV ground receiver is a USB-compatible device that uses Zigbee to communicate. 
The USB device driver has been developed by the customer so the hardware can be accessed by the ground station software, and other applications. 
The USB is active when the host asks for a data. 
A host is the computer network which the UAV connects to. 
The data is in queue until the host asks for the data. 
\begin{figure}[!hbtp]
\begin{center}
\includegraphics[scale=0.4]{figures/clipArt.png} 
\end{center}
\caption{The connection of the hardware\label{schemetic_clipA}}
\end{figure}


\section{GUI data flow diagram}

Table~\ref{command_table} shows how the command is sent and received to and from the ground station.  Figure~\ref{GCS_Payload_comm} gives a brief detail of how the data communicates between the payload and the image viewer program. 
A more detailed diagram on the data communication can be found in the figure \ref{sequence diagram}.
This has been tested in section\ref{sec:send_console}.

\begin{figure}[H]
\begin{center}
\includegraphics[scale=0.6]{figures/GCS_Payload_communication.png} 
\caption{The connection of data stream port\label{GCS_Payload_comm}}
\end{center}
\end{figure}

\begin{table}[H]

\begin{center}
\begin{tabular}{l l @{.} l}
 Command&
\multicolumn{2}{l}{Address Byte } \\

\hline
\underline{Command from Ground} & \\
SEND\_ZERO\_TOKEN & 0 \\
TAKE\_PICTURE & 0 \\
SEND\_DOWNLOAD\_REQUEST & 2 [MSB] [LSB]  \\
\\
\underline{Command received at Ground}\\
PICTURE\_TAKEN & 1 [MSB] [LSB]\\
DOWNLOAD\_INFO & 3 [MSB] [LSB]\\
IMAGE\_DATA & 4 $\overbrace{ [packet number]}^{2bytes} \overbrace{[image data]}^{data length}$ \\
\end{tabular}
\caption{Command table\label{command_table}}
\end{center}
\end{table}

\section{Get Image Algorithm}
\label{get image algorithm}
When the user clicks on the \texttt{Get Picture} button, the program should send a \textbf{string command} to the ground station software.
Then, the ground station software generates a \textbf{byte command} to be transmitted by TCP to the payload. 
Afterwards, the payload sends a ''Picture Taken'' command back through the data stream port to the Image Viewer Program.

When the program starts running, it initializes the port and commands the customer's application to tell the UAV to stream data to the data stream port. Figure~\ref{GCS_connect_command} shows the connection between the UAV data stream port and the ground station.

\begin{figure}[!hbtp]
\begin{center}
\includegraphics[scale=0.5]{figures/connect_command.png} 
\end{center}
\caption{The connection of data stream port\label{GCS_connect_command}}
\end{figure}

 
\section{Important Code}

This section describes the code that is important for the program. The entire code will not be described but it will be in the appendices. The program must be able to perform these actions to the port: connect to, receive data from and send data to. 

The following .NET C\# classes will be used in the ground station program:
\begin{itemize}
	\item \texttt{FileStream} creates a file. 
	\item \texttt{BinaryWriter} writes the byte data into a specific file made by the \texttt{FileStream} class. 
\end{itemize}

\subsubsection*{UAV Connection}
This section references appendix \ref{appen:UAVConnector} lines 18-38.

The \texttt{Socket} class has functions to send and receive byte and string data. The connection protocol is using the \texttt{PortConnect()} method.  
        
The design of the .NET \texttt{Socket} class simply connects to the port by a \texttt{PortConnect()} command regardless of changes to the baud rate, stop bits, and parity bits. 
This advantage makes the \texttt{Socket} class a more compatible class than the \texttt{SerialPort} class to work with the UAV.
This class has tested using a console application before implementing it in a final GUI. 
This will complete milestone\ref{sec:ms_pl_tx_token_resp}.

\subsection{Start of the program}
This section references appendix \ref{appen:main_form} line 78-87

The \texttt{FileStream} class is initialized by using the method \texttt{FileMode.Create()} in order to generate files. The \texttt{BinaryWriter} writes binary bytes into a file in the directory of the \texttt{FileStream}.
The \texttt{BinaryWriter} class creates a binary file using specific data layout for its bytes. 


\subsection{Text Command}
This section references appendix \ref{appen:UAVConnector} line 68-85.

The TCP/IP protocol transfers data without modifying them. 
It allows the application to freely encode the data \cite{davidB}.
The ground station software allows the Image Viewer program to send a stream composed of a string in bytes. It will then read the command bytes and send it to the payload on the UAV. The codes have shown a correct way to implement the string and send a byte array to the payload.

\texttt{The Socket.Send()} method sends bytes to the ground station software. The ''@ '' sign indicates that the command is correct. For example, consider the bit of code: uavConn.SendTextToUAV(''da 20 payload[0].mem\_ bytes[0]'');
The text \texttt{''da 20 payload[0].mem\_ bytes[0]''} will be converted into a char array and then into a byte array. The byte array will then send the command to the console port and then to the payload through the method \texttt{consolePort.Send(toUAVByte, toUAVChar.Length)}. This will complete Milestone\ref{sec:ms_pl_rx_msg_gs}.

In order to test that the payload receives the same data, we use the oscilloscope to see the signal. The byte displayed on the payload is the same as the byte sent from the ground station. Therefore, the Milestone\ref{sec:ms_pl_img_gs_trigger} is completed. The different resolutions send different byte commands to the payload. The byte commands change if the comboBox options change. Therefore, we can say that Milestone\ref{sec:ms_pl_img_gs_cam_res} has completed.

\begin{lstlisting}[caption={writing binary file},label=lst:writingb]          
	for (int i = 3; i < packetSize; i++)
	{
          opFile.Write(packet[i]);
          numBytes++;
    	}
\end{lstlisting}         

At every cycle of the data being received, the \texttt{opFile.Write()} method will write the packet data into the file in the directory. After the cycle finished, the file will be saved and the image will be displayed in the picture box. 

\subsection{Get Picture Button}
This \texttt{Get Picture} button will use both the send and receive functions of the program. 
The work flow of the \texttt{Get Picture} signal is shown in figure \ref{GUI_finalWorkFlow}.
If the sequence of signal is sent and received correctly, the photo received from the UAV will be displayed on the photo box in the program.
This will complete Milestone\ref{sec:ms_pl_img_sending_gs}.

\begin{figure}[H]
\begin{center}
\includegraphics[scale=1]{figures/finalWorkFlow.png} 
\end{center}
\caption{Final work flow diagram of the GUI\label{GUI_finalWorkFlow}}
\end{figure}

\subsection{Implementation - Way-point Triggering (ms)}
\label{sec:waypoint_triggering}

\subsubsection{Description}

The ground station is capable of assigning way-points to
the payload which will allow the camera to take pictures 
at a given location.

This is achieved by sending a simple command script from
the ground station to be uploaded by the payload 
controller. The way-point is designated by the user
through the ground station software. The script 
tells the UAV controller to continuously check the
distance separating itself from the way-point.
When the UAV reaches is within 200 metres of the
way-point, a 0 byte is sent to the camera to prompt
the camera to take a picture.

After taking a picture at the way-point, the camera
is delayed for 10 seconds to avoid taking another picture
at the same way-point. When the 10 second delay is over,
the camera repeats the operation and continuously checks
its distance from the next waypoint.

\subsubsection{Pseudo-code description}

Below is a brief pseudo-code description of the script
sent by the ground station to the payload to take an
image at designated way-points:

\begin{itemize}
	\item while !(UAV distance from next way-point $le$ 200 metres)
		\begin{itemize}
			\item Do nothing.
		\end{itemize}
	\item end while
	\item Prompt camera to take an image.
	\item wait 10 seconds
	\item Re-enter while loop
\end{itemize}

\subsection{Other functions}
The \texttt{Delete} button works like a normal file deleting button. However, when there is only one picture in the file, the program is not allowed to delete that file. This is because even if the pictureBox is set to null, the last image displayed is still pointing to the deleting file. However, the disadvantage of the \texttt{Delete} button is that if the wanted photo was deleted accidentally, it may take a long time to launch the UAV again and take the same photo. 

\begin{figure}[H]
\begin{center}
\includegraphics[scale=0.5]{figures/resolutionOption.png} 
\end{center}
\caption{The resolution in combo box\label{resolutionOption}}
\end{figure}

The camera has resolution options as shown in Figure\ref{resolutionOption}. This can be useful when the downloading speed must be fast. The lower the resolution, the faster the data is transmitted to the ground. The GUI's combo box allows the user to choose any wanted resolution in the options. The resolution option gives the user more control over the camera. 

\begin{figure}[H]
\begin{center}
\includegraphics[width=0.3\textwidth]{figures/progressBar.png} 
\end{center}
\caption{The resolution in combo box\label{progressBar}}
\end{figure}

\section{Complete System}
\label{completeSystem}
After implementation has been completed, the different functions will be tested together. When all the parts come together, the ground station program fulfills the requirements of the specification. Figure \ref{completeSystem} shows a final working GUI of our program. 

\begin{figure}[H]
\begin{center}
\includegraphics[width=1.0\textwidth]{testing_screenshots/ui.png} 
\end{center}
\caption{The complete system\label{completeSystem}}
\end{figure}




%% ----------------------------------------------------------------
%\chapter{Implementation - Whole System}
%% ----------------------------------------------------------------

%\section{System Integration}

