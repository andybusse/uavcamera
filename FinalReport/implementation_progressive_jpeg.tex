\subsection{JPEG Information Extractor}

\subsubsection{Progressive Scan of JPEG (commandCheck)}

After receiving the size of the JPEG file (in bytes), 
the JPEG Information Extractor starts reading the bytes of the JPEG stored in the SD card. 
Because the information will not all be available instantly, 
the extractor uses a class to read the next byte in the input stream, one at a time. This class \textbf{read\_byte()} is 
how the JPEG extractor gains access to the JPEG input stream. 

As the extractor reads in the bytes from the input stream, it checks to see if the byte indicates the start of a header (0xff). 
If this is the case, the next byte identifying the header is read and the \textbf{JPEGMethod} class is 
called which uses a switch command to determine the actions to take. 
If not, the data is discarded and the extractor continues to read the next byte is read from the input stream. 

\subsubsection{Image Data Extraction (JPEGMethod)}

The JPEG Information Extraction class does not need all the possible information that can be extracted from a JPEG image. 
The few image headers containing important information store the important information from the input stream into 
variables which will be sent to the custom encoding algorithm. 

If the header is not recognized by this class, 
it is assumed to not have any important information and is ignored; 
the code continues to scan through the input stream for the next header byte. 
The following headers have  information to be stored (all information is considered to be stored in one byte unless stated otherwise):

\paragraph*{SOF0: Start Of Frame (0xc0)}
\begin{description}
	\item[Data precision] indicates how many bits compose one sample (generally 8 to indicate a byte).
	\item[Image height] indicates the height of the image in pixels (2 bytes long).
	\item[Image width] indicates the width of the image in pixels (2 bytes long).
	\item[Number of components] indicates how many components define one pixel. 
		This will always be three for our cases to indicate the use of the YCbCr colour model. 
		The component information is read in in the order Y, Cb, Cr.
	\item[Component ID] indicates the ID number used by the JPEG file for this component.\footnote{One for each component}
	\item[Sampling factor] both horizontal and vertical are stored within a single byte. 
		The least significant 4 bits indicate teh vertical sampling factor, 
		while the most significant 4 bits indicate the horizontal sampling factor.\footnotemark[1] 
	\item[Quantization Table Number] associates the component to a quantization table.\footnotemark[1] 
\end{description}

\paragraph*{APP0: JFIF application segment (0xe0)}
\begin{description}
	\item[File Identifier Mark] indicates that this is a JFIF standard image.
	\item[Major Revision Number] must be checked to be 1.
	\item[Minor Revision Number] 
	\item[Units for x/y Density] 
	\item[X Density]2 bytes
	\item[Y Density]2 bytes
	\item[Thumbnail Width] Not used.
	\item[Thumbnail Height] Not used.
	\item[Remaining bytes] contains information regarding the image's thumbnail. This is not used in our extractor.
\end{description}

\paragraph*{DHT: Define Huffman Table(s) (0xc4)}
\begin{description}
	\item[HT Information] contains the following information concerning the indicated Huffman Table in one byte:
		\begin{description}
			\item[Bits 0\ldots3] ID number of the HT (cannot exceed 3)
			\item[Bit 4] Indicates whether the HT is DC (0) or AC (1)
			\item[Bits 5\ldots7] Unused.
		\end{description}
	\item[Number of Symbols] is a string of 16 bytes, each byte containing the number of symbols with 
		codes of length 1\ldots16. For example, if the third byte of the string is 7, 
		there are 7 signals of code length 3 in this Huffman Table.
	\item[Symbols] is a string of \emph{n} bytes, where \emph{n} is 
		the sum of all the values in the previous 16-byte string. 
		The extractor stores the Huffman tables in a linked list of struct objects representing the Huffman Tables.
\end{description}

\paragraph*{DQT: Define Quantization Table(s) (0xdb)}
\begin{description}
	\item[QT Information] contains the following information concerning the indicated Quantization Table in one byte:
		\begin{itemize}
			\item[Bits 0\ldots3] ID number of the QT (cannot exceed 3).
			\item[Bits 4\ldots7] Precision of the QT. Either 8-bit (0) or 16-bit (all other values).
		\end{itemize}
	\item[QT values] is a string of \emph{n} bytes, where \emph{n} is $64*$precision$+1$.
\end{description}

\paragraph*{SOS: Start Of Scan (0xda)}
\begin{description}
	\item[Number of components in scan] indicates how many components define one pixel in the scan. 
		This is usually identical to  the number of components obtained from the SOF segment.
	\item[Component ID] indicates the ID number used by the JPEG file for this component.\footnotemark[1] 
	\item[Huffman Table Indicator] stores the AC and DC Huffman Table ID number associated with this component in one byte. 
		The AC Table number is stored in bits 0\ldots3. The DC Table number is stored in bits 4\ldots7.\footnotemark[1]
\end{description}

\subsubsection{Image Data Treatment}

The SOS Header information extraction segment is followed by 
an algorithm which sorts the image data into chrominance pixels before 
sending the chrominance pixel to the custom JPEG encoding class. 
If we consider \emph{n} to be the number of image pixels in a chrominance pixel, then 
we associate the first $n - 2$ bytes with the Y (luma) Huffman tables, the $n - 1$ byte with the Cb Huffman tables, and 
the $n^{th}$ byte with the Cr Huffman Table. 
This continues until the EOI header is found, or until there is no more bytes coming in from the input stream.
