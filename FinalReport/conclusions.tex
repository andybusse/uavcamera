%% ----------------------------------------------------------------
\chapter{Conclusions}
%% ----------------------------------------------------------------

This section will discuss to what extent we have fulfilled the requirements 
of this project, mainly in terms of our initial specification, including a 
justification for why certain aspects were not met, and an analysis of 
whether the project, as it currently stands, could be reasonably presented to our customer as a working solution.

\section{Commented Specification}

The initial specification is available as Appendix \ref{initial_spec}. 

This is then analysed and justified in Section \ref{sec:specification}

\subsection{Fully Completed Parts}

\begin{itemize}
\item \ref{sec:spec_b} : The mean time to download an image, with no packet 
loss, is X seconds ??? reference ???. This figure is obtained from testing 
with the serial link only.
\item \ref{sec:spec_c} : The combined weight of the electronic module and 
board camera is 58g. Combined with the camera enclosure and ethernet cable 
(i.e. the entire additional weight to the UAV to add our module), this is Xg.
?? insert pictures of scales proving this ??
\item \ref{sec:spec_d} : We are able to get an image resolution of up to 
$640\times480$ pixels. Reference the test.
\item \ref{sec:spec_e} : We provide software that prompts the UAV to capture 
an image, download it and then displays it on screen. ?Reference??
\item \ref{sec:spec_f} : Our software is able to cancel the download of an 
image mid-transmission. ?reference test?
\item \ref{sec:spec_i} : Our software allows you to select any resolution that 
is featured on the camera, and download an image at that resolution. ?show test?
\item \ref{sec:spec_m} : We are able to transmit colour images as opposed to 
black and white due to our choice of camera - our camera is only able to 
provide colour images. Therefore it is implemented, but only through our 
choice of camera, not through design.
\end{itemize}

\subsection{Partially Completed Parts}

\begin{itemize}
\item \ref{sec:spec_a} : This task was lowered in priority when we discovered 
that our camera compressed JPEG images a lot more severely than we were 
expecting (image files are approximately 30kB in size, we were expecting 
these files to be 900kB in size). To reflect the time invested in this, we 
have produced some code that partially implements this part of the 
specification, available ???here?
\item \ref{sec:spec_j} : A progress indictor does exist ???ref???, but the 
display only indicates the number of packets received - it does not 
indicate the time remaining to download as this includes the time taken to 
capture an image from the camera and to store it on the SD card.
\end{itemize}

\subsection{Incomplete Parts}
\label{sec:incomplete}

\begin{itemize}
\item \ref{sec:spec_g} : Resending of the same image is not implemented. 
This is, however, a low priority task, which could have been implemented 
if we had more time. In theory, this would be rather trivial to implement.
\item \ref{sec:spec_h} : Interruption of an image mid-transmission and 
saving the incomplete image was again, a low priority task that was not 
implemented due to a lack of time. Again, it should be rather trivial to 
implement.
\item \ref{sec:spec_l} : Automatic image capture at user-specified time 
intervals was another low priority task which we were unable to implement, 
but which should be rather trivial to implement due to the SkyCircuits 
software's scripting interface, but slightly more difficult than the above
two.
\item \ref{sec:spec_n} : Selecting between colour and black-and-white images 
was included in the specification as we considered that this would reduce 
the image transmission time. However, this was not a feature supported 
by our board camera, implementing it on our AVR would be a non-trivial 
process, and the priority to reduce transmission time was reduced as the 
colour image files themselves were much smaller than expected.
\end{itemize}

\section{Deliverables}

This will discuss to what extent we have delivered everything we specified 
in our specification \ref{initial_spec}

\subsection{Hardware}

\ref{sec:deliv_a} : We have delivered a suitable board camera, and an 
electronic module on PCB which interfaces the camera to the autopilot's 
payload port. All schematics, layouts, gerber files etc. are provided in 
our central Github repository \ref{github}, and as Appendices \ref
{Appendix_schematics} and ???

\subsection{Software}

\ref{sec:deliv_b} : The electronic module delivered to the customer is 
presented with firmware loaded on to it, and the user interface software 
presented as an executable file (requiring only that the .NET framework 
is installed on the PC, however this is the same dependancy as for the 
SkyCircuits software). All firmware for the electronic module, as well as 
source code for our executable Ground Station Software, is presented in 

\subsection{Documentation}

\ref{sec:deliv_c} : Documentation for our Ground Station Software is 
delivered with this project ???reference???. It is assumed that our customers' clients are technically competent, therefore documentation on 
how to order a PCB with our gerber files, how to construct the PCB, and how 
to flash an ATmega644P with our source code was deemed unnecessary.

\subsection{Public Repository}

\ref{sec:deliv_d} : All files related to our project are available in our 
central GitHub repository \ref{github}. Most things in there are open-source: source code is licensed under GPLv3 (GNU Public License version 3), 
\ref{gpl}
images (including PCB layouts and schematic designs) are licensed under 
CC BY-SA 2.0 UK (Creative Commons Attribution ShareAlike) \ref{ccbysa}, 
and all documentation under the FDL (Free Documentation License), \ref{fdl}. This is as-requested, and should encourage sharing this information 
amongst our customers' clients.

\section{General Conclusions}

In conclusion, we have presented the customer with exactly what was requested: 
an electronic module that interfaces 
