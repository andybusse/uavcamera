\section{Risk Management}

\subsection{Risks Planned}
A risk assessment was performed at the beginning of the project to make sure that the appropriate actions could be taken when confronted with a problem. 
The following table shows the risks the group has prepared for, the likelihood and 
impact of the risk which determines the priority of the risk, and the appropriate action in response to the risk.

\begin{center}
	\begin{tabular}{ | p{4cm} | p{2cm} | p{2cm} | p{5cm} | }
	\hline
	\textbf{Risk} & \textbf{Likelihood (1: Low, 5: High)} & 
	\textbf{Impact (1: Low, 5: High)} & \textbf{Action} \\ \hline
	Faulty Components & 4 & 4 & Order spares where feasible.
	Source new/replace faulty components as soon as possible otherwise. \\ \hline
	Team member becoming unavailable & 2 & 4 & At least two people per task.
	Reallocating people to different tasks as required. \\ \hline
	Team member facing difficulties & 5 & 2 & At least two people per task.
	Good team communication. Prioritise necessary tasks first. \\ \hline
	Tasks overrunning & 4 & 3 & Plan redundancies into Gantt Charts.
	Reallocate resources when necessary. Communicate between team members often. \\ \hline
	Loss of files/source code & 1 & 5 & Proper use of source control. Back up all source code. \\ \hline
	Loss of access to facilities & 1 & 4 & Work from home laboratory. 
	Buy missing components if necessary \\
	\hline
	\end{tabular}
\end{center}

\subsection{Risks Encountered}

\subsubsection{Faulty Components}
Two major components were found to be faulty very early on. 
The first camera module that was ordered arrived in an unresponsive state. 
Due to the price of the device, the group preferred attempting to debug the device before ordering a new camera module. 
Eventually however, another device needed to be ordered. 
The second camera module lasted for the majority of the development process, but also 
become unresponsive during the final weeks of the project. A third camera was ordered for testing purposes. 
This vulnerable camera made up the majority of the project's budget (SEE BUDGET).

The other faulty component was the autopilot module which was given to the group by the customer. 
Due to the nature of the autopilot, being a custom component from the customer, it was not possible to simply order a new autopilot module. 
The group decided to contact the customer for help, and worked with the customer to decode the autopilot software. 
It is important to note that the autopilot module was also in a prototype state, and this is the first practical implementation of the module.

\subsubsection{Team Members Unavailable}
Some members of the team were too occupied to attend all the weekly group meetings and the weekly meetings with the supervisor. 
However, this was not a sever problem and individual members were always willing to allocate meeting times when asked directly. 
The group members were willing to warn the others when they would be unavailable for a large period of time. 
This issue did not cause large set-backs in the project.

\subsubsection{Difficulties}
Whenever difficulties were faced, the group member facing the difficulty would inform all the other group members about the problem. 
If the difficulty concerns a high priority task, the individual group members would reassign tasks to make sure that all the high priority 
tasks were completed as soon as possible. Thanks to the skills audit, this made sure that 
any difficulties faced in the high priority tasks were taken care of.

\subsubsection{Tasks Overrunning}
In order to prevent important tasks from overrunning, the group decided to meet up together on a 
weekly basis as well as organizing weekly meetings with the supervisor. 
The group meetings would be used to monitor all the tasks that had been accomplished during the week and 
assign each member with tasks to be performed for the next week. 
These meetings were very useful and it successfully monitored the tasks that would require extra time to be completed. 
The meetings with the supervisor informed him about the progress the group has made. 
It gave the group a more objective idea of how they were progressing and if what tasks should be prioritized.
