\section{Risk Management (ms)}

A risk assessment was performed at the beginning of the project to make sure that the appropriate actions could be taken when confronted with a problem. 
The following table shows the risks the group has prepared for, the likelihood and 
impact of the risk which determines the priority of the risk, and the appropriate action in response to the risk.

\begin{center}
	\begin{tabular}{ | p{4cm} | p{2cm} | p{2cm} | p{5cm} | }
	\hline
	\textbf{Risk} & \textbf{Likelihood (1: Low, 5: High)} & 
	\textbf{Impact (1: Low, 5: High)} & \textbf{Action} \\ \hline
	Faulty Components & 4 & 4 & Order spares where feasible.
	Source new/replace faulty components as soon as possible otherwise. \\ \hline
	Team member becoming unavailable & 2 & 4 & At least two people per task.
	Reallocating people to different tasks as required. \\ \hline
	Team member facing difficulties & 5 & 2 & At least two people per task.
	Good team communication. Prioritise necessary tasks first. \\ \hline
	Tasks overrunning & 4 & 3 & Plan redundancies into Gantt Charts.
	Reallocate resources when necessary. Communicate between team members often. \\ \hline
	Loss of files/source code & 1 & 5 & Proper use of source control. Back up all source code. \\ \hline
	Loss of access to facilities & 1 & 4 & Work from home laboratory. 
	Buy missing components if necessary \\
	\hline
	\end{tabular}
\end{center}

