% This is part of the FinalReport document.
% Copyright (C) 2011 Piyabhum Sornpaisarn, Andrew Busse, Michael Hodgson, John Charlesworth, Paramithi Svastisinha
% See the file COPYING in FinalReport/ for copying conditions.

\section{SOF0: Start Of Frame (0xc0):}

\begin{table}[!hbtp]
	\label{sof-content}
	\caption{SOF0 Marker Content}
	\centering
	\begin{tabular}{ | p{2cm} | p{1.5cm} | p{4cm} | }
	\hline
	\textbf{Field} & \textbf{Size} & \textbf{Comments} \\ \hline
	Length & 2 bytes & Equal to 8 + the number of components used in the colour scheme * 3\\ \hline
	Data precision & 1 byte & Value shown in bits/sample. Usually 8 signifying bytes.\\ \hline
	Image height  & 2 bytes & ----\\ \hline
	Image width  & 2 bytes & ----\\ \hline
	Number of components & 1 byte & indicates the number of components used  in the colour scheme\\ \hline
	Component information & 3 bytes / component & See component information table below.\\ \hline
	\end{tabular}
\end{table}

\begin{table}[!hbtp]
	\label{sof-comp-content}
	\caption{SOF Component Information}
	\centering
	\begin{tabular}{ | p{2cm} | p{4cm} | }
	\hline
	\textbf{Byte Number} &  \textbf{Information} \\ \hline
	1 & Component ID (1 = Y, 2 = Cb, 3 = Cr, 4 = I, 5 = Q)\\ \hline
	2 & Sampling factors (bits 0\ldots3 : vertical, bits 4\ldots7 horizontal)\\ \hline
	3 & Quantization table number\\ \hline
	\end{tabular}
\end{table}

\newpage

\section{DHT: Define Huffman Table(s) (0xc4):}

\begin{table}[!hbtp]
	\label{dht-content}
	\caption{DHT Marker Content}
	\centering
	\begin{tabular}{ | p{2cm} | p{1.5cm} | p{4cm} | }
	\hline
	\textbf{Field} & \textbf{Size} & \textbf{Comments} \\ \hline
	Length & 2 bytes & ----\\ \hline
	HT Information & 1 byte & See HT Information table below\\ \hline
	Number of Symbols  & 16 bytes & Number of symbols with codes of length 1\ldots16.
	The sum of these values (\emph{n})must be $\le$ 256\\ \hline
	Symbols  & \emph{n} bytes & Symbols of the HT in order of increasing code length\\ \hline
	\end{tabular}
\end{table}

\begin{table}[!hbtp]
	\label{dht-info-content}
	\caption{HT Information Byte Content}
	\centering
	\begin{tabular}{ | p{2cm} | p{4cm} | }
	\hline
	\textbf{Bit} &  \textbf{Information} \\ \hline
	0\ldots3 & ID number of HT\\ \hline
	4 & type of HT, 0 = DC table, 1 = AC table\\ \hline
	5\ldots7 & Unused\\ \hline
	\end{tabular}
\end{table}

\section{DQT: Define Quantization Table(s) (0xdb):}

\begin{table}[!hbtp]
	\label{dqt-content}
	\caption{DQT Marker Content}
	\centering
	\begin{tabular}{ | p{2cm} | p{1.5cm} | p{4cm} | }
	\hline
	\textbf{Field} & \textbf{Size} & \textbf{Comments} \\ \hline
	Length & 2 bytes & ----\\ \hline
	QT Information & 1 byte & See QT Information table below\\ \hline
	Bytes  & \emph{n} bytes & QT values. \emph{n} = 64 * (QT precision + 1) \\ \hline
	\end{tabular}
\end{table}

\begin{table}[!hbtp]
	\label{dqt-info-content}
	\caption{QT Information Byte Content}
	\centering
	\begin{tabular}{ | p{2cm} | p{4cm} | }
	\hline
	\textbf{Bit} &  \textbf{Information} \\ \hline
	0\ldots3 & ID number of QT\\ \hline
	4\ldots7 & precision of QT (0 = 8 bit, else 16 bit)\\ \hline
	\end{tabular}
\end{table}

\newpage

\section{SOS: Start Of Scan (0xda):}

\begin{table}[!hbtp]
	\label{sos-content}
	\caption{SOS Marker Content}
	\centering
	\begin{tabular}{ | p{2cm} | p{1.5cm} | p{4cm} | }
	\hline
	\textbf{Field} & \textbf{Size} & \textbf{Comments} \\ \hline
	Length & 2 bytes & Equal to 6 + 2 * (number of components in scan)\\ \hline
	Number of components in scan & 1 byte & Within range 1 to 4\\ \hline
	Component information & 2 bytes / component & See component information table below.\\ \hline
	Ignorable Bytes & 3 bytes & Separates header information from entropy-encoded image data\\ \hline
	\end{tabular}
\end{table}

\begin{table}[!hbtp]
	\label{sof-comp-content}
	\caption{SOS Component Information Content}
	\centering
	\begin{tabular}{ | p{2cm} | p{4cm} | }
	\hline
	\textbf{Byte Number} &  \textbf{Information} \\ \hline
	1 & Component ID (1 = Y, 2 = Cb, 3 = Cr, 4 = I, 5 = Q)\\ \hline
	2 & HT to use (bits 0\ldots3 : AC Table ID, bits 4\ldots7 DC Table ID)\\ \hline
	\end{tabular}
\end{table}
