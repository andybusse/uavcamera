\label{chap:implementation_ground_station }

The user can only see the image viewer program.
The console application were used for testing because it load faster than the window application.
The programmer make use of both data stream port and the console port in order to send command and receive JPEG photo from the UAV.

\section{The development process} 
Because GUI include many functions on the application and also it links to TCP/IP ports, the GUI part consider as a very big program. 
Therefore, in order to complete the GUI step by step, it needs a plan for the development.
The connection to datastream can be done by using a console application describe in section\ref{sec:testing_connection_send_to_stream}.
If the connection is correct the appliction can listen to the data stream port as in section\ref{sec:testing_receive_stream}.
These console application will be combined to a bigger and more complete program in later section.

\flushleft
\begin{enumerate}


\item	The most important process of GUI development is to understand what does the customer wants. 

\item	The hardware and software specification have to be determined.
 
\item	Make a GUI design decision

\item Develop a smaller program in order to make the testing easy and less time consuming. 

\item	Learn the .NET class that can support the connection to TCP/IP, and communication between host and device

\item	Use GUI to link to the customer’s software to access the software

\item	Distribute the GUI
\end{enumerate}

\section{Use Case Diagram}
The use case diagram show what functionality the user can use in the program.
It has to include all the specification that the customer want for a complete system.
Figure~\ref{GUI_useCase} shows a possible user action on the program which the user can save, open and delete any jpeg image from the computer. 
The user has an option to connect to the UAV in case of the UAV is not connected properly.
He can command the image viewer program to send an image to the software, and it will display the image onto the picture box.
\begin{figure}[H]
\begin{center}
\includegraphics[scale=0.6]{figures/userCase.png} 
\end{center}
\caption{Use case diagram of the GUI\label{GUI_useCase}}
\end{figure}

The connect button was not implemented because it will do automatically when the program is open.
The raw image is too big so we did not implement it.
Therefore, the Milestone\ref{sec:ms_pl_img_gs_cam_colour_type} will not be implemented.
Figure\ref{GUI_finalUseCase} shows an agreed use case diagram. 

\begin{figure}[!hbtp]
\begin{center}
\includegraphics[scale=0.7]{figures/FinaluserCase.png} 
\end{center}
\caption{Final use case diagram of the GUI\label{GUI_finalUseCase}}
\end{figure}



\section{The Design}

The user for our application is assumed to have a limited programming experience, so the program will be implemented so it is simple understand. Figure~\ref{ini_GUI} show the first brainstorm view of the GUI. 
During the downloading process, the application should keep running so the cancel button can be used. 
Gallery button will link to another page which will be the collection of images taken. 
Left and right button can navigate the picture box to view an earlier picture or later picture. 
Cancel button cancel the receiving image, therefore the corrupted picture will not be downloaded. 
The user mode of the application can access only the main feature such as take picture, change directory, and cancel download picture.  
It allows the user to choose the resolution and picture type (RAW or JPEG) to transmitted from the UAV to the ground station. But the user doesn't have access to changing the command, changing the receiving data, and any interaction with the UAV because of safety and avoid of any errors. 
\begin{figure}[H]
\begin{center}
\includegraphics[width=1.0\textwidth]{figures/initialGUI.png} 
\end{center}
\caption{The initial design of GUI\label{ini_GUI}}
\end{figure}
The GUI has been planned to have functions such as auto triggering, image type, resolution type, file path chosen, progress bar, help button, stop and delete. Figure \ref{finalGUI} is the screen shot of the final GUI.

\begin{figure}[H]
\begin{center}
\includegraphics[width=1.0\textwidth]{figures/finalGUI.png} 
\end{center}
\caption{final GUI\label{finalGUI}}
\end{figure}

\section{Class Diagram}
\subsection*{The Initial Class Diagram}
Figure~\ref{ini_Class} shows initial classes and methods of the image viewer program.
The \texttt{JPEGFileReader} Class has functions for decoding/encoding the JPEG file.
There will be decoding/encoding algorithm because the image will take long time to download to the ground station. 
\begin{center}
\begin{figure}[!hbtp]
\includegraphics[width=150mm,height=100mm]{figures/initialClassDiagram.png} 
\caption{The initial design of GUI classes\label{ini_Class}}
\end{figure}
\end{center}
\texttt{DCT} class has many math operation and equation which have to be implemented on the image viewer program. 

The \texttt{Painting} class is supported by the \texttt{DCT} class. The intention of this class is to display an encoded image point by point on the \texttt{pictureBox}.
By this method, the \texttt{pictureBox} can display an image from the first pixel transmitted.

The \texttt{CameraCommand} class design for send the data from the ground station to the camera. The idea is that make the camera sync with the payload by using the ground station command. \texttt{SetAndSendCommand()} use for set the byte command and then send it to the payload via the Console port. \texttt{SetGetPicture(),SetInitial(), SetPhoto(), and SetTakePicture()} methods use for setting the correct byte in order to send the byte by using \texttt{SetAndSendCommand} class.

\subsection*{The Class Diagram}
The class diagram has been implemented very differently from the planned one.
This is because the new plan is to decode the image on board and then transmitted to the ground station in JPEG further compressed file. 
Therefore, the class \texttt{JpegFileReader} has changed to C code and to be implemented on board. 
The \texttt{DCT} class is not needed anymore because all the calculation will be on board. 
The \texttt{CameraCommand} has taken away because the payload will receive an image viewer command to the payload and then the payload will send another different signal to the camera.
Therefore, the command send to the camera from the payload does not have to be the same as the command sent from ground station to the payload. 
The \texttt{Painting} class use to draw each pixel onto the \texttt{pictureBox}, but it has not been implemented in the final program because of the stated reason.
Therefore, the milestone\ref{sec:ms_pl_img_gs_progressive_dl} will not be implemented.
This is also because of the time limitation.
More detail about the progressive image is in the section\ref{sec:implementation_progressive_jpeg}.
 
\begin{figure}[!hbtp]
\begin{center}
\includegraphics[scale=0.7]{figures/finalClassDiagram.png} 
\end{center}
\caption{Final class diagram of the GUI\label{GUI_finalClassDiagram}}
\end{figure}

\section{Before Connect to the Image Viewer Program}
Figure \ref{schemetic_clipA} shows a diagram of how the connection of the hardware should be. 
The UAV ground receiver is a USB-compatible device which uses Zigbee to communicate. 
USB device driver has been developed by the customer’s so the hardware can be accessed by ground station software, and other applications. 
USB is active when the host ask for a data. 
A host is the computer network which the UAV connect to. 
The data in its queue until the host asks for the data. 
\begin{figure}[!hbtp]
\begin{center}
\includegraphics[scale=0.4]{figures/clipArt.png} 
\end{center}
\caption{The connection of the hardware\label{schemetic_clipA}}
\end{figure}


\section{GUI data flow diagram}

Table~\ref{command_table} shows how the command send and receive to/from ground station.  Figure~\ref{GCS_Payload_comm} give a brief detail of how the data communicate between the payload and the image viewer program. 
A more detailed diagram on the data communication is in figure \ref{sequence diagram}.
This has been tested in section\ref{sec:send_console}.

\begin{figure}[H]
\begin{center}
\includegraphics[scale=0.6]{figures/GCS_Payload_communication.png} 
\caption{The connection of data stream port\label{GCS_Payload_comm}}
\end{center}
\end{figure}

\begin{table}[H]

\begin{center}
\begin{tabular}{l l @{.} l}
 Command&
\multicolumn{2}{l}{Address Byte } \\

\hline
\underline{Command from Ground} & \\
SEND\_ZERO\_TOKEN & 0 \\
TAKE\_PICTURE & 0 \\
SEND\_DOWNLOAD\_REQUEST & 2 [MSB] [LSB]  \\
\\
\underline{Command received at Ground}\\
PICTURE\_TAKEN & 1 [MSB] [LSB]\\
DOWNLOAD\_INFO & 3 [MSB] [LSB]\\
IMAGE\_DATA & 4 $\overbrace{ [packet number]}^{2bytes} \overbrace{[image data]}^{data length}$ \\
\end{tabular}
\caption{Command table\label{command_table}}
\end{center}
\end{table}

\section{Get Image Algorithm}
\label{get image algorithm}
The plan is of getting image button is when the user click on the Get Picture button, the program sends a \textbf{string command} to the ground station software.
Then the ground station software generates a \textbf{byte command} to transmitted by TCP to the payload. 
Then the payload sends a ''Picture Taken'' command back through the data stream port to the Image Viewer Program. 
Picture taken button has been tested in section\ref{sec:test_get_image_button}
The Image Viewer Program will then automatically send a download request command to the payload. 
The payload will then send image data back to the data stream port. 

When the program start running, the program initialized the port and commands the customer’s application to tell the UAV to stream data to the data stream port. Figure~\ref{GCS_connect_command} shows the connection between the UAV data stream port and the ground station.

\begin{figure}[!hbtp]
\begin{center}
\includegraphics[scale=0.5]{figures/connect_command.png} 
\end{center}
\caption{The connection of data stream port\label{GCS_connect_command}}
\end{figure}


 
\section{Code Highlight}

This section describes the code that is important for the program. The entire code will not be described but it will be in the appendices. It needs a class that can do these to port: connect, receive and send. \texttt{FileStream} class is a class in the .NET C\# which can create a file. \texttt{BinaryWriter} class is use for writing the byte data into a specific file made by the \texttt{FileStream} class. 

\subsubsection*{Connect to UAV:Appendix \ref{appen:UAVConnector} line 18-38.}
The \texttt{Socket} class has functions to send and receive byte and strings data. The handshaking protocol is using \texttt{PortConnect()} method.  
        
The design of the .NET \texttt{Socket} class simply connects to a Port by a single command without any hesitation of changing the baud rate, stop bits, and parity bits. 
This advantage makes the \texttt{Socket} class a more useful class then the \texttt{SerialPort} Class to work with the Port with existing and static set up. 
This class has been tested using a console application before implementing it in a final GUI. 
This will complete milestone\ref{sec:ms_pl_tx_token_resp}.

\subsection{Start of the program: Appendix \ref{appen:main_form} line 78-87}%
\texttt{FileStream} was initialize to be in \texttt{FileMode.Create()}, so it can create file. The BinaryWriter write the binary byte into a file in the directory of the fileStream.
The \texttt{BinaryWriter} class can create a binary file using specific data layout for its bytes. 


\subsection{Text Command:Appendix \ref{appen:UAVConnector} line 68-85}
TCP/IP protocols transfer data without modifying them. 
This allow the application to freely encode the data \cite{davidB}.
The Ground station software allows the program to send a stream of string in bytes and it will read the command bytes and send it to the payload on the UAV. The code has shown the way to implement the string and send a byte array to the payload.

\texttt{The Socket.Send()} method can send bytes to the ground station software. The ''@ '' sign indicate that the command is correct. For example:uavConn.SendTextToUAV(''da 20 payload[0].mem\_ bytes[0]'');
The text \texttt{''da 20 payload[0].mem\_ bytes[0]''} will be converted to char array and then to byte array. The byte array will then send the command to the console port to the payload by \texttt{consolePort.Send(toUAVByte, toUAVChar.Length)}. This will complete Milestone\ref{sec:ms_pl_rx_msg_gs}.

In order to test that the payload receives the same data, we use the oscilloscope to see the signal. The byte display on the payload is the same as the byte data send from the ground station. Therefore, the Milesone\ref{sec:ms_pl_img_gs_trigger} is completed. The different resolutions send different byte command to the payload. The byte command changes if the comboBox options change. Therefore, we can say that Milestone\ref{sec:ms_pl_img_gs_cam_res} has completed.


\begin{lstlisting}[caption={writing binary file},label=lst:writingb]          
	for (int i = 3; i < packetSize; i++)
	{
          opFile.Write(packet[i]);
          numBytes++;
    	}
\end{lstlisting}         

At every cycle of the data being received, \texttt{opFile.Write()} method will write the packet data into the file in the directory. After the cycle finished, the file will be saved and the image will be displayed in the picture box. 

\subsection{Get Picture Button}
This get picture button will use both send and receive function of the program. 
The work flow of the get picture signal show in figure \ref{GUI_finalWorkFlow}.
If the sequence of signal send/receive correctly, the photo that receive from the sky will be display on the photo box in the program.
This will complete Milestone\ref{sec:ms_pl_img_sending_gs}.

\begin{figure}[H]
\begin{center}
\includegraphics[scale=1]{figures/finalWorkFlow.png} 
\end{center}
\caption{Final work flow diagram of the GUI\label{GUI_finalWorkFlow}}
\end{figure}

\subsection{Implementation - Way-point Triggering (ms)}
\label{sec:waypoint_triggering}

\subsubsection{Description}

The ground station is capable of assigning way-points to
the payload which will allow the camera to take pictures 
at a given location.

This is achieved by sending a simple command script from
the ground station to be uploaded by the payload 
controller. The way-point is designated by the user
through the ground station software. The script 
tells the UAV controller to continuously check the
distance separating itself from the way-point.
When the UAV reaches is within 200 metres of the
way-point, a 0 byte is sent to the camera to prompt
the camera to take a picture.

After taking a picture at the way-point, the camera
is delayed for 10 seconds to avoid taking another picture
at the same way-point. When the 10 second delay is over,
the camera repeats the operation and continuously checks
its distance from the next waypoint.

\subsubsection{Pseudo-code description}

Below is a brief pseudo-code description of the script
sent by the ground station to the payload to take an
image at designated way-points:

\begin{itemize}
	\item while !(UAV distance from next way-point $le$ 200 metres)
		\begin{itemize}
			\item Do nothing.
		\end{itemize}
	\item end while
	\item Prompt camera to take an image.
	\item wait 10 seconds
	\item Re-enter while loop
\end{itemize}

\subsection{Other functions}
The delete button work like a normal file deleting button. But where there is only one picture in the file, the program is not allow to delete because even the pictureBox is set to null, its last memory is still point at the deleting file. However, the disadvantage of the delete button is that if the wanted photo got deleted accidentally, it might take a long time to launch the UAV again and take the same photo. 

\begin{figure}[H]
\begin{center}
\includegraphics[scale=0.5]{figures/resolutionOption.png} 
\end{center}
\caption{The resolution in combo box\label{resolutionOption}}
\end{figure}
The camera has options of resolution as shown in Figure\ref{resolutionOption}. This can be useful when the speed is important. The lower the resolution, the faster the data transmitted to the ground. GUI has the combo box for the user to choose any wanted resolution in the options. The resolution allow user to have more accessible to the camera. However, this mean there is more on the programmer side to program the application.

\begin{figure}[H]
\begin{center}
\includegraphics[width=0.3\textwidth]{figures/progressBar.png} 
\end{center}
\caption{The resolution in combo box\label{progressBar}}
\end{figure}

\section{Complete System}
After implementations has completed, functions will be tested together. The display of image data will implement as a final display of the image.  Figure \ref{completeSystem} shows a final working GUI of our program. 
\begin{figure}[H]
\begin{center}
\includegraphics[width=1.0\textwidth]{testing_screenshots/ui.png} 
\end{center}
\caption{The complete system\label{completeSystem}}
\end{figure}


