\subsection{Ground Station Image Viewer}
The GUI needs a long winded testing. This is because all the button has to be click and test one by one separately. Because of this, some function of the UAV that does not need a button to test will have it own console application to test it in order to save time. The programmer has to detect all the possible bug that might cause during the real operation in user mode. This part is very important because for any program if it caught an error, the program will end unexpectedly.
\subsubsection{Connect to UAV, Send and Receive Data}

The UAV connector use .NET class called System.Net.Sockets. The method that will be using from this class is Sockets.Connect(), Sockets.Send(), Sockets.Receive(). To test the function of this class the programmer develop a separate console application in order to debug only the specific part of the program. Sockets.Connect can be tested by using the Console to display the error code of the running application. The problem could be wrong host name and port number. Therefore, the user are not allow to change these values. The other problem found in the program is that the programmer forgot to connect the UAV, tell the GCS to enable the enable the network server, and tell the GCS to stream data. Listing \ref{lst:connectionT} show a Conector class that use for testing the Socket connect, read and write method.
\begin{lstlisting}[caption=the class that use to test the connection to the UAV,label=lst:connectionT]
    class Connector
    {
        static int requestCounter;
        static ArrayList hostData = new ArrayList();
        static StringCollection hostNames = new StringCollection();
        Socket UAV = new Socket(AddressFamily.InterNetwork,
         SocketType.Stream, ProtocolType.Tcp);

        public byte[] dataFromUAV()
        {
            byte[] uavData = new byte[1];
            byte[] testMessage = {0}; //send zero token to 
            start the datastream port to receive something
            try
            {
                int sendByte = UAV.Send(testMessage, testMessage.Length, 
                SocketFlags.None); //send zero token to the UAV
                Console.WriteLine("Sent {0} bytes.", sendByte); //print on the console
                 so we know what we know
                int size = 8;

                while (true)
                {
                    int byteCount = UAV.Receive(uavData, size, SocketFlags.None);
                     //receive data from the uav of fixed length size
                    for(int countIn=0; countIn<uavData.Length;countIn++)
                    Console.WriteLine("{0}", uavData[countIn]); //
                }
            }
            catch(SocketException uavMessage)
            {
                Console.WriteLine("{0} Error code:{1}.",
                 uavMessage.Message, uavMessage.ErrorCode);
            }
            return uavData;
        }

        public void ConnectToPort()
        {
            const Int32 portNumber = 8802;
            const string portName = "localhost";
            try
            {
                UAV.Connect("localhost", portNumber);
                Console.WriteLine("Connection to Port Name:{0}; Port Number{1} is complete!"
                ,portName,portNumber);
            }
            catch (SocketException uavMessage)
            {
                Console.WriteLine("{0} Error code:{1}.", uavMessage.Message
                , uavMessage.ErrorCode);
            }
        }
    }
\end{lstlisting}
\subsubsection{Save and Open Photo from FileDialog Class}
In order to test the save and open FileDialog Class, we need a window application to test it. The  Get Picture button and the resolution option is commented out in order to test only the FileDialog. The only proper way to test the FileDialog class is to run the program and see the File directory from the file path TextBox. The problem founded by this long winded testing is that when the other file that is not JPEG file got picked in the user mode, the pictureBox will give an error. This problem can be fixed by not allowing the user to choose a file that is not a JPEG file. The FileDialog need a thread handler because it has to run an extra window application for the user to choose the file. This however can not change by the user. So this problem have already solved.
\subsubsection{Image Data}
The connection between the aircraft and the ground station is using TCP port, it allows the data to send from the aircraft in any form of data. In order to test this relationship, a camera will take a picture that the developer know what the picture look like. If the picture display an image the same as we expected, the image data is correct. 
\subsubsection{Send command to console port}
The ground station program allow the programmer to use the console port to send a string command to it. This string command can be test that it is correct by the ground station program will produce a line with ''@'' sign on the front of the command sent from the outside of the program.
If the data is correct, the ground station program will detect it and send the data to UAV. So if we receive the correct data from the data stream, then we know that the the string command is correct.
\subsubsection{Get Picture Button}
The get picture button is a combination of many function of the program. When the button got clicked, it must send the correct resolution signal, receive signal from data stream port, save the byte into a file, and display the image. Therefore, this will be the last button that will be tested. However, all the previous testing work properly so the testing of this button is very fast.

