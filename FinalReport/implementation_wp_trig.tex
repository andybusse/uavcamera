\section{Implementation - Way-point Triggering (ms)}
\label{sec:waypoint_triggering}

\subsection{Description}

The ground station is capable of assigning way-points to
the payload which will allow the camera to take pictures 
at a given location.

This is achieved by sending a simple command script from
the ground station to be uploaded by the payload 
controller. The way-point is designated by the user
through the ground station software. The script 
tells the UAV controller to continuously check the
distance separating itself from the way-point.
When the UAV reaches is within 200 metres of the
way-point, a 0 byte is sent to the camera to prompt
the camera to take a picture.

After taking a picture at the way-point, the camera
is delayed for 10 seconds to avoid taking another picture
at the same way-point. When the 10 second delay is over,
the camera repeats the operation and continuously checks
its distance from the next waypoint.

\subsection{Pseudo-code description}

Below is a brief pseudo-code description of the script
sent by the ground station to the payload to take an
image at designated way-points:

\begin{itemize}
	\item while !(UAV distance from next way-point $le$ 200 metres)
		\begin{itemize}
			\item Do nothing.
		\end{itemize}
	\item end while
	\item Prompt camera to take an image.
	\item wait 10 seconds
	\item Re-enter while loop
\end{itemize}

See section \ref{sec