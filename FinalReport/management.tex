%% ----------------------------------------------------------------
\chapter{Project Management}
%% ----------------------------------------------------------------
Project management goes here...

\section{Gantt Charts}
Michael
Place Gantt charts here.

\section{Risk Management}
Mitch
Place risk management considerations and compare with risks encountered.

\section{Work Allocation}
Peak
Talk about planned work allocation to be compared with actual work allocation. Reflect on efficiency.

\subsection{Planned Work Allocation}
Talk about the tasks planned and the skills audit.

\subsection{Actual Task Allocation}
Talk about who did what in the end and how tasks were passed around. How did people work together?

\section{Team Resources}
John
What we used to get the job done...

\subsection{Budget}
What did we need to buy in the end?

\subsection{Electronic Material}
Did we make use of anything electronic that we had before the project?

\section{Group Communication}
Andy
Talk about how the group contacted each other (email, meetings, mobiles)...



\subsection{Formal Meetings}
Talk about the usefulness and benefit of the formal Tuesday meetings...

For the duration of the project, we held weekly meetings on most Tuesday 
Afternoons from 4pm onwards in the Hartley Library. This allowed us to 
review progress made against progress expected, and to modify our project 
plan and allocation of work for the following week.
\\
Minutes of these meetings are available both in our repository \cite{github} 
(documents/minutes), and as an appendix ?????
\\
The timing of these (the afternoon before our weekly meetings with our 
supervisor) was also quite useful, as it allowed us to present a clear outline 
of new developments and our plans week-by-week.

\subsection{Methods of Communication}
Talk about which methods were used more often, which were useful...

The group has used various methods of communication during the project: 
email, 

\subsection{Source Control}
How we used e-mails, SVN Tortoise, and Github to keep our progress safe...

We decided to use version control software throughout the project, to manage 
everything from minutes to source code and payload schematics. As well as 
being a very useful tool to facilitate group work, it also helps us to meet 
our customer's request to deliver this as an open source project - at the end 
of the project, a repository could easily be gardened and presented, with the 
addition of appropriate open-source licenses at the end of the project.
\\
Initially, we used a central Subversion (SVN) repository hosted on ECS' 
UGForge service. It was decided to use this as most of the group had 
experience using SVN. This worked well for most of the project, but 
unfortunately, something was committed to this repository that could not be 
open sourced (the customer's Ground Control Station software). This presented 
us with a problem, as although it would be trivial to revert this commit with 
the "\$ svn merge" command, this creates a new commit, and the file in 
question would remain in the repository's history and could still be accessed.
\\
In theory, it would be possible to remove this commit from SVN history by 
using the "\$ svnadmin dump" command, filtering the offending commit out of 
the dumpfile, and regenerate the repository with the "\$ svnadmin load" 
command. However, in practice, this was not possible as our repository 
contains binary information above 64kB, therefore the ASCII editor used to 
modify the dumpfile would cut off data in the binary file after the 64kB 
limit, rendering any following data useless.
\\
It would be possible to start a new SVN repository and check in our 
repository up to but not including the offending commit, but we would lose 
all metadata (i.e. committer, commit time, etc.). Therefore, it was decided to 
switch to git, a distributed version control system (which would allow us to 
modify project history). Converting the repository using git-svn was a trivial 
process. Although not all of the group was confident using this tool, git 
lets a committer commit on behalf of another user. Moving to git also 
allowed us to use github, a hosting service, which is free to open-source 
projects (such as ours), which also provides us with some additional 
project statistics. ??? Reference ???? Appendix ???
