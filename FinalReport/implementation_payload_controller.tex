%\section{Payload Controller}


\section{Communication Between Payload Controller and Ground Station}
\subsection{Existing Code}
Our customer had provided us with some payload module AVR code, written for a 
ATMega168, for communicating with the autopilot. This code was the basis on 
which the payload controllers communication link was built.

The code provided a number of useful utilities:
\begin{itemize}
\item Ability to set shared memory on the autopilot.

\item Ability to receive messages sent from the Ground Station to the 
autopilot.

\item Example code for setting shared memory on the autopilot.
\end{itemize}

This base code was modified slightly after a bug was found in its handling of 
the transmit enable signal. The RS485 communication protocol used for the 
autopilot-payload link (as described in section |||||||| SEC ||||||||) 
requires a `transmit enable' signal to be asserted when the payload is 
transmitting. This signal should be asserted just before data is to be sent 
and cleared just after. However, the original payload base code cleared this 
signal in an interrupt service routine (ISR) which fired after the transmit 
buffer of the UART was ready to accept new data. Since this transmit buffer 
would be ready to accept new data before the data was actually sent over the
physical connection this lead to the transmit enable signal being cleared
before all data had been sent, causing strange behaviour on the RS485 link.
This bug did not seem to cause any problems, and the odd behaviour was only
noticed when testing the system with an oscilloscope. ||||| INC TRACES ||||||
It was considered sensible to fix the bug in case it did cause problems later.

The fix for this problem was reasonably simple: a new ISR was set up which 
fired only when the current transmission had actually completed, and the 
command to clear the transmit enable signal was moved into this ISR.
||||| INC AFTER TRACE ||||||


\subsection{Communication Protocol}
In order to communicate with the ground station image viewing software some
form of communications protocol is required so that both ends of the link 
are communicating in a mutually understandable manner. The specification and 
design decisions taken require this protocol to be built upon the existing
autopilot payload interface.

To better explain the protocol used we will split the explanation into two 
sections: a \emph{Autopilot Payload Interface} section describing the
pre-existing autopilot payload interface on which we are building the 
protocol and a \emph{UAV Camera Control Protocol} section describing the 
protocol we have implemented as a part of this project.

\subsubsection{Autopilot Payload Interface}
|||||| Maybe have this in background research ||||||

The SkyCircuits autopilot module allows extension modules named `payload 
modules' to be connected to the autopilot. These payload modules are connected 
via a RS485 serial connection, allowing several payload modules to be 
connected at once in a daisy-chain configuration. All payload modules are 
connected to common TX and RX lines, where the RX line is used by the
autopilot to send commands and data to the payload modules, and the TX
line is used by all daisy-chained modules to communicate with the autopilot.

||||||| Include diagram of daisy chain configuration ||||||||||

Since the TX line is shared between all modules only one payload module can be 
transmitting at once over the link, with all other payload modules required to
leave the line tristated. This means that each payload must know when it is
allowed to use the transmit line so as not to clobber any other payload module.
In this system this is achieved by the use of `transmit tokens' handed out 
by the autopilot over the RX line. A `transmit token' is sent to each payload 
module in turn, informing the module that it is clear to transmit data.

This two way communication link is used to implement a command interface to 
the autopilot. A payload module can execute commands on the autopilot, 
allowing a variety of useful and interesting possibilities for payload
module design. Of interest to us however, is the ability of the payload
module to set shared memory. Since this shared memory can be accessed 
through the ground station software this allows us to send data through the
autopilot link to the ground station. A set of shared memory is allocated to 
each payload module, accessible on the ground station using the command:


\begin{lstlisting}[caption={Accessing shared memory from Ground Station}, label=lst:gs_shared_mem_set]
payload[payload_num].mem_bytes[mem_bytes_num]
\end{lstlisting}

Where \emph{payload\_num} is the ID number identifying the payload and 
\emph{mem\_bytes\_num} is the index of the set of shared memory to be accessed.

Each shared memory set is of variable length and can be set from the payload 
module using the following function (code provided by customer):

\begin{lstlisting}[caption={Setting shared memory from Payload}, label=lst:payload_shared_mem_set]
send_set_class_indexed_item_indexed(CLASS_PAYLOAD, module_id, 
CLASS_PAYLOAD_MEM_BYTES, mem_bytes_num, message_to_send,
message_to_send_length)
\end{lstlisting}

Where \emph{CLASS\_PAYLOAD} and \emph{CLASS\_PAYLOAD\_MEM\_BYTES} are constants 
informing the autopilot that the \emph{mem\_bytes} item of the \emph{payload} 
object should be set, \emph{module\_id} is the ID of the payload module,
\emph{mem\_bytes\_num} is the same as used in listing \ref{lst:gs_shared_mem_set},
\emph{message\_to\_send} is the message to be sent (consisting of an array of 
length \emph{message_to_send_length}, the first element of which should be 
the number of bytes to set in the shared memory).

The method through which this shared memory is accessed via the ground station
image viewer is discussed in section ||||||||| GS IMAGE VIEWER |||||||||.

The ground station software can also send data directly to a payload module 
through the autopilot, where it will be sent onto the RX line of the RS485
bus. 

\subsubsection{UAV Camera Control Protocol}