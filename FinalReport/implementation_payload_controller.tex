


\section{Communication with Ground Station via Autopilot}
\label{sec:payload controller}
Considering the overall aim of this project: to produce a system by which 
images can be downloaded over-the-air from a payload module to a ground 
station, it is fair to say that some method of communicating between the
payload module and ground station are an essential component in the system.

The specification - see chapter \ref{chap:specification} - requires the payload
module to communicate with the ground station using the autopilots payload
module interface (discussed in section \ref{sec:autopilot_payload_interface}
below).

%To better explain the protocol used we will split the explanation into two 
%sections: a \emph{Autopilot Payload Interface} section describing the
%pre-existing autopilot payload interface on which we are building the 
%protocol and a \emph{UAV Camera Communication Protocol} section describing the 
%protocol we have implemented as a part of this project.

\subsection{Autopilot Payload Module Interface}
\label{sec:autopilot_payload_interface}
|||||| Maybe have this in background research ||||||

The SkyCircuits autopilot module allows extension modules named `payload 
modules' to be connected to the autopilot. These payload modules are connected 
via a RS485 serial connection at 38.4 kBaud, allowing several payload modules
to be connected at once in a daisy-chain configuration. All payload modules are 
connected to common TX and RX lines, where the RX line is used by the
autopilot to send commands and data to the payload modules, and the TX
line is used by all daisy-chained modules to communicate with the autopilot.

||||||| Include diagram of daisy chain configuration ||||||||||

Since the TX line is shared between all modules only one payload module can be 
transmitting at once over the link, with all other payload modules required to
leave the line tristated. This means that each payload module must know when it is
allowed to use the transmit line so as not to clobber any other payload module.
In this system this is achieved by the use of `transmit tokens' handed out 
by the autopilot over the RX line. A `transmit token' is sent to each payload 
module in turn, informing the module that it is clear to transmit data. With 
only one payload module connected to the autopilot, these tokens are sent out 
every 20 ms. |||||| CHECK THIS ||||||||

This two way communication link is used to implement a command interface to 
the autopilot. A payload module can execute commands on the autopilot, 
allowing a variety of useful and interesting possibilities for payload
module design. Of interest to us however, is the ability of the payload
module to set shared memory. Since this shared memory can be accessed 
through the ground station software this allows us to send data through the
autopilot link to the ground station. Shared memory is allocated to 
each payload module, accessible on the ground station using the command:
~\\
~\\
\begin{lstlisting}[caption={Accessing shared memory from ground station}, label=lst:gs_shared_mem_set]
payload[payload_num].mem_bytes[mem_bytes_num]
\end{lstlisting}

Where \emph{payload\_num} is the ID number identifying the payload and 
\emph{mem\_bytes\_num} is the index of the set of shared memory to be accessed.

Each shared memory set is of variable length and can be set from the payload 
module using the following function (code provided by customer):
~\\
~\\
\begin{lstlisting}[language=C, caption={Setting shared memory from payload module}, label=lst:payload_shared_mem_set]
send_set_class_indexed_item_indexed(CLASS_PAYLOAD, module_id, 
CLASS_PAYLOAD_MEM_BYTES, mem_bytes_num, message_to_send,
message_to_send_length)
\end{lstlisting}

Where \emph{CLASS\_PAYLOAD} and \emph{CLASS\_PAYLOAD\_MEM\_BYTES} are constants 
informing the autopilot that the \emph{mem\_bytes} item of the \emph{payload} 
object should be set, \emph{module\_id} is the ID of the payload module,
\emph{mem\_bytes\_num} is the same as used in listing
\ref{lst:gs_shared_mem_set} and \emph{message\_to\_send} is the message to be 
sent (consisting of an array of length \emph{message\_to\_send\_length}, the 
first element of which should be the number of bytes to set in the shared 
memory).

The method through which this shared memory is accessed via the ground station
image viewer is discussed in chapter \ref{chap:implementation_ground_station}.

The ground station software can also send data directly to a payload module 
through the autopilot, where it will be sent on to the RX line of the RS485
bus. The \emph{send\_bytes} command is used to do this, as can be seen in 
listing \ref{lst:ground_station_send_bytes}. 

As discussed in section |||||||| REF |||||||| it was decided that 
our communications protocol would use shared memory and \emph{send\_bytes} 
commands, allowing two way communications between the payload controller and 
ground station software to be established.

\subsection{UAV Camera Communication Protocol}
The Payload Module Interface discussed above (section \ref{sec:autopilot_payload_interface})
allows us to send strings of bytes in both directions. However, in order to 
communicate with the ground station image viewing software some form of 
additional communications protocol is required so that both ends of the link 
are communicating in a mutually understandable manner.

This two way communications is the interface between the payload and the 
ground station software, so some standard protocol was required. It was decided
that a message based system would be used, with the messages from the ground
station to the payload module being sent using \emph{send\_bytes} and the 
messages sent from the payload to the ground station being put into shared
memory. Each message is composed of two elements, one byte for the message ID 
- unique to each type of message - and a variable number of data bytes 
(depending on the message type.) The different message types are detailed 
below:

\subsubsection*{Messages sent from Ground Station To Payload}

\begin{itemize}
\item \textbf{Take Picture}
\begin{itemize}
\item \emph{Data:} None
\item Prompts payload module to capture an image and save it to the SD card.
\end{itemize}

\item \textbf{Image Download Request} 
\begin{itemize}

\item \emph{Data:} Image ID
\item Requests the payload send the image with ID \emph{Image ID} to the 
ground station. This message allows any image stored by the payload module 
to be downloaded over the connection, increasing flexibility. 
\end{itemize}

\item \textbf{Configure Camera}
\begin{itemize}
\item \emph{Data:} Colour Type, Raw Image Resolution, JPEG Image
Resolution
\item Sets the image resolution and colour mode of the camera. Only the 
JPEG mode has been tested so far.
\end{itemize}

\end{itemize}

\subsubsection*{Messages Sent from Payload to Ground Station}

\begin{itemize}

\item \textbf{Picture Taken}

\begin{itemize}
\item \emph{Data:} Image ID

\item Informs the ground station software that an image has been taken and 
saved to the SD card. \emph{Image ID} is the ID of the image that has been 
saved to the SD card.
\end{itemize} 

\item \textbf{Image Download Info}

\begin{itemize}

\item \emph{Data:} Number of Image Packets

\item Sent by the payload after a successful \emph{Image Download Request}
message from the ground station. Informs the ground station how many 
\emph{Image Data} packets to expect.
\end{itemize}

\item \textbf{Image Data} 
\begin{itemize}
\item \emph{Data:} Packet Number, Image Data
\item This message contains an amount of actual image data. Sent after a
\emph{Image Download Info} message which is in turn in response to an 
\emph{Image Download Request} message. The whole image is sent over
\emph{Number of Image Packets} packets (as defined by the \emph{Image Download
Info} message.) \emph{Packet Number} informs the ground station which of these
packets the message is carrying. \emph{Image Data} contains the actual image 
data for this packet and is variable size, with a maximum size of 50 bytes. 
\end{itemize}

\end{itemize}


\subsection{Existing Code}
\label{sec:payload_existing_code}
Our customer had provided us with some payload module communication AVR code
- written for a ATMega168 - for communicating with the autopilot. This code
was the basis on which the payload controllers communication link was built.

The code provided a number of useful utilities:
\begin{itemize}
\item Ability to set shared memory on the autopilot.

\item Ability to receive messages sent from the Ground Station to the 
autopilot.

\item Example code for setting shared memory on the autopilot.
\end{itemize}

This base code was modified slightly after a bug was found in its handling of 
the transmit enable signal. The RS485 communication protocol used for the 
autopilot-payload link (as described in section |||||||| SEC ||||||||) 
requires a `transmit enable' signal to be asserted when the payload is 
transmitting. This signal should be asserted just before data is to be sent 
and cleared just after. However, the original payload base code cleared this 
signal in an interrupt service routine (ISR) which fired after the transmit 
buffer of the UART was ready to accept new data. Since this transmit buffer 
would be ready to accept new data before the data was actually sent over the
physical connection this lead to the transmit enable signal being cleared
before all data had been sent, causing strange behaviour on the RS485 link.
This bug did not seem to cause any problems, and the odd behaviour was only
noticed when testing the system with an oscilloscope. ||||| INC TRACES ||||||
It was considered sensible to fix the bug in case it did cause problems later.

The fix for this problem was reasonably simple: a new ISR was set up which 
fired only when the current transmission had actually completed, and the 
command to clear the transmit enable signal was moved into this ISR.
||||| INC AFTER TRACE ||||||


||||||||| FLOWCHART OF INTERACTIONS |||||||


|||||||| REMINDER: TALK ABOUT VOLATILE VARIABLE PROBLEM

|||||||| REMINDER: AUTOPILOT PROBLEMS

%|||||||| REMINDER: PROBLEMS CHALLENGES
%|||||||| REMINDER: HOW DID WE SOLVE PROBLEMS, DEBUGGING, TESTING, ETC
%|||||||| REMINDER: JOHN COULD TALK ABOUT HOW BROKEN CAMERAS SLOWED DOWN DEVELOPMENT BUT GOOD PLANNING AND CONTINGENCY MINIMISED RISK OR IN MANAGEMENT SECTION
%|||||||| REMINDER: FUTURE WORK
%|||||||| REMINDER: ADD NEWPAGES BETWEEN SECTIONS
