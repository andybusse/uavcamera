\chapter{Glossary}

\section*{Terms} 
\begin{description}
	\item[YCbCr] Colour space used to represent JPEG images. Composed of three values:
		\begin{description}
			\item[Y] luma component.
			\item[Cb] blue-difference chroma component.
			\item[Cr] red-difference chroma component.
		\end{description}
	\item[4:x:y]Chroma subsampling notation.
		\begin{description}
			\item[4] Luma horizontal (and vertical) sampling reference.
			\item[x] Cb and Cr horizontal sampling factor.
			\item[y] Cb and Cr horizontal sampling factor. If 0, indicates 2:1 vertical subsampling for both Cb and Cr.
		\end{description}
\end{description}

\section*{Abbreviations} 
\begin{description}
	\item[AC] Alternating Current 
	\item[DC] Digital Current 
	\item[JPEG] Joint Photographic Experts Group, creators of the jpeg compression method. Used interchangeably with the jpeg image type.
	\item[EXIF] EXchangeable Image File format for digital still cameras.
	\item[MCU] Minimum Coded Unit
	\item[DFT] Discrete Fourier Transform
	\item[FFT] Fast Fourier Transform
	\item[DCT] Discrete Cosine Transform
	\item[WHT] Walsh–Hadamard Transform
	\item[KLT] Karhunen-Lo\`eve Transform
	\item[SOI] Start Of Image
	\item[SOF] Start Of Frame
	\item[DHT] Define Huffman Table(s)
	\item[HT] Huffman Table
	\item[DQT] Define Quantization Table(s)
	\item[QT] Quantization Table
	\item[SOS] Start Of Scan
	\item[EOI] End Of Image
	\item[SD card]
\end{description}
