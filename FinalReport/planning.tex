% --------------------
\chapter{Planning}
% --------------------

\section{Introduction}

\section{Initial Milestones - (mh23g08)}
\label{sec:initial_milestones}
A set of milestones have been devised to help track progress and inform test plans.
Each of the milestones represents a real and measurable progress step towards the 
end goals of this project, the milestones were derived from the tasks listed in the initial
Gantt chart (see appendix |||||||||||). It should be noted that these were the initial
milestones derived for the project and were substantially amended throughout the course of
the project in order to keep them relevant - the final state of the amended milestones
can be seen later in section \ref{sec:amended_milestones}.

To assist comprehension the milestones have been arranged into a number of categories,
relating to the way in which the project work has been broken up.

\subsection{Payload Module - Camera Module Communication Milestones}
sections. 
%\begin{itemize}
	\subsubsection{Milestone: Prototype Camera Module Communication}
	\label{sec:ms_init_prototype_camera_module_comms}
		\begin{itemize}
			\item 	Initial basic communication with a camera module, an image should be
				downloaded from the camera and viewed in some way.
		\end{itemize}

	\subsubsection{Milestone: Payload controller to/from camera module communication}
	\label{sec:ms_init_payload_controller_camera_module}
		\begin{itemize}
			\item 	An image should be downloaded from the camera on to the 
				payload controller hardware.
			\item 	Sub milestones are as follows:
			\begin{itemize}
				\item The payload controller should be capable of changing the
					image resolution of the camera.
				\item The payload controller should be capable of changing the
					colour type of the camera.
			\end{itemize}
		\end{itemize}

%\end{itemize}

\subsection{Payload Module - Image Encoding and Transmission Milestones}
%\begin{itemize}
	\subsubsection{Milestone: Basic Raw Encoding}
		\label{sec:ms_init_basic_raw_encoding}
		\begin{itemize}
			\item 	The payload module should send raw image data in the same form as 
				obtained from the camera via the autopilot link
				to the ground station where it should be saved or viewed in some manner
				to verify the milestone.
		\end{itemize}

	\subsubsection{Milestone: Custom (Progressive/Compressed) Encoding - Matlab Algorithm Prototype}
		\label{sec:ms_init_custom_enc_matlab}
		\begin{itemize}
			\item 	A prototype of progressive encoding prototyped in matlab
				should be implemented. This algorithm should demonstrate how the image
				could be sent progressively and ideally in a compressed form.
		\end{itemize}

	\subsubsection{Milestone: Custom Encoding Algorithm Implementation on Payload Controller}
		\label{sec:ms_init_custom_enc_payload}
		\begin{itemize}
			\item 	Milestone \ref{sec:ms_init_custom_enc_matlab} should be implemented on
				the payload controller module.
		\end{itemize}

	\subsubsection{Milestone: Custom Encoding Data Breakdown and Transmission to Ground Station via Autopilot Link}
		\label{sec:ms_init_custom_breakdown_transmission}
		\begin{itemize}
			\item 	The progressive data produced by milestone
				\ref{sec:ms_init_custom_enc_payload} should be broken down and sent
				via the autopilot link to the ground station.
		\end{itemize}

\subsection{Payload Module Construction Milestones}
Many of the milestones listed above rely upon a working physical implementation of the system at each stage
so these will not be duplicated. However, there are a few milestones which are specific to the physical implementation
of the system.
%Customer Provided `Dummy' Payload Schematics Working

%Camera connection

%Payload communication

	\subsubsection{Milestone: Payload Breadboard Prototype}
		\begin{itemize}
			\item A finalized PCB design ready to be sent for manufacture.
		\end{itemize}

	\subsubsection{Milestone: Payload PCB Design}
		\begin{itemize}
			\item A finalized PCB design ready to be sent for manufacture. 
		\end{itemize}
		
	\subsubsection{Milestone: Complete System on PCB}
		\begin{itemize}
			\item A working system implementation on PCB.
		\end{itemize}

%Final implementation

\subsection{Ground Station Image Viewer Milestones}
	\subsubsection{Milestone: Basic Dummy Server Communications}
		\label{sec:ms_init_basic_dummy_server_comms}
		\begin{itemize}
			\item 	Some prototype programs should be produced to help understand TCP/IP communications.
				One server program serving some test data over a TCP/IP port and one
				client program connecting to this port and sending/receiving test data.
		\end{itemize}

	\subsubsection{Milestone: Communications with Customers Ground Station Sofware}
		\label{sec:ms_init_basestation_comms}
		\begin{itemize}
			\item 	Milestone \ref{sec:ms_init_basic_dummy_server_comms} should be
				completed.
			\item 	Communication with the customers ground station software via TCP/IP console and
				data port. Data should be streamed from some known source and verified correct and
				commands should be sent via the port and verified that they execute in the program.
		\end{itemize}

	\subsubsection{Milestone: Decode Basic Raw Image}
		\label{sec:ms_init_decode_basic_raw_image}
		\begin{itemize}
			\item 	Milestone \ref{sec:ms_init_basestation_comms} should be completed.
			\item 	Data sent from the implementation of \ref{sec:ms_init_basic_raw_encoding} should be
				decoded and saved/displayed on the ground station in some way.
		\end{itemize}

	\subsubsection{Milestone: Custom Decoding - Matlab Implementation}
		\begin{itemize}
			\item 	Milestone \ref{sec:ms_init_custom_enc_matlab} should be completed.
			\item 	Data sent from the implementation of \ref{sec:ms_init_custom_enc_matlab} should be
				decoded and saved/displayed on the ground station in some way. NB: This does not require
				a working payload implementation.
		\end{itemize}

	\subsubsection{Milestone: Custom Decoding - Deployable Implementation}
		\begin{itemize}
			\item 	Milestone \ref{sec:ms_init_decode_basic_raw_image} should be completed.
			\item 	Data sent from the implementation of \ref{sec:ms_init_custom_breakdown_transmission} should be
				decoded and saved/displayed on the ground station in some way.
		\end{itemize}

	\subsubsection{Milestone: Functional User Interface}
		\begin{itemize}
			\item 	A functional user interface that allows users to take images and view them using the implementations
				for the milestones described above.
		\end{itemize}

%\end{itemize}

% This is part of the FinalReport document.
% Copyright (C) 2011 Piyabhum Sornpaisarn, Andrew Busse, Michael Hodgson, John Charlesworth, Paramithi Svastisinha
% See the file COPYING in FinalReport/ for copying conditions.

\section{Risk Management (ms)}

A risk assessment was performed at the beginning of the project to make sure that the appropriate actions could be taken when confronted with a problem. 
The following table shows the risks the group has prepared for, the likelihood and 
impact of the risk which determines the priority of the risk, and the appropriate action in response to the risk.

\begin{center}
	\begin{tabular}{ | p{4cm} | p{2cm} | p{2cm} | p{5cm} | }
	\hline
	\textbf{Risk} & \textbf{Likelihood (1: Low, 5: High)} & 
	\textbf{Impact (1: Low, 5: High)} & \textbf{Action} \\ \hline
	Faulty Components & 4 & 4 & Order spares where feasible.
	Source new/replace faulty components as soon as possible otherwise. \\ \hline
	Team member becoming unavailable & 2 & 4 & At least two people per task.
	Reallocating people to different tasks as required. \\ \hline
	Team member facing difficulties & 5 & 2 & At least two people per task.
	Good team communication. Prioritise necessary tasks first. \\ \hline
	Tasks overrunning & 4 & 3 & Plan redundancies into Gantt Charts.
	Reallocate resources when necessary. Communicate between team members often. \\ \hline
	Loss of files/source code & 1 & 5 & Proper use of source control. Back up all source code. \\ \hline
	Loss of access to facilities & 1 & 4 & Work from home laboratory. 
	Buy missing components if necessary \\
	\hline
	\end{tabular}
\end{center}



\section{Work Allocation}

Working in a group has its advantages, but it also has drawbacks. The group has more man power and time, but the effectiveness of work allocation and other management skills must be achieved. Therefore, to use all the human resources as competent as possible. Although the developers have known about the specification of the project well, it is not so obvious how to implement the entirely approach to the project that is driven by some device which the members never have seen it before such as camera module, and the UAV. The following factor has been considered thoroughly before assigning task to any team members:

\begin{itemize}
\item	Availability of each member.

\item	The resource limitation
 
\item	The clarification of the task

\item   Time allowance for unexpected problems

\item	Each individual skills
\end{itemize}

\begin{figure}[!hbtp]
\begin{center}
\includegraphics[scale=0.6]{figures/spec_block_diagram_2.png} 
\caption{The system final diagram\label{systemBlocD}}
\end{center}
\end{figure}


\subsection{Planned Work Allocation}
 It is very essential that each member other coursework deadline and exams include in the consideration of work allocation. 
The engineer will work effectively if they got assigned  to their interest in the duty and that role clarified well. The work assign to each individual will be based on the posses sufficient skill and knowledge to perform the task. However, the team members must have flexibility in their time in order to support other team members when they have problems. Each task has been assigned a task leader and it will have the another member set to the same task in case the task leader has problems with the work they have done. In order to avoid work over run the time, the project has been set an earlier deadline so that if any unexpected problems happen, there are still a spare time to solve the problem. The specific skills that individuals want to mention are:

\begin{itemize}
\item Andy: Power, PCB, Control, MATLAB
\item Mitch: Image Processing, MATLAB, ASM, Report Writing
\item John: MATLAB, Image Processing, Control, Radio Transmission
\item Peak: Digital Control, Display, MATLAB, Information Theory
\item Michael: Programming, $\mu$Processors, Interfaces
\end{itemize}

Each member has variety of skills. Therefore the task has been located to each person according to this list.

The time line of the task is also important because some task for each individual depend on a completion of another task. This cause problems when one task got stuck and so the productiveness of the time management will be reduced. This can be solved by assigning another task that is not dependant on the other tasks. 

A Resource limitation is one of the factor that limited the group member to do tasks. This is because the customer provide only a single UAV and also only one camera has been purchased, so only one person can work on this device at a time. We solved this problem by having a laboratory space that the UAV will be placed their all the time so any of the team member can access it. To mange the limited resource, the only time that the individual access the UAV is when he wants to test the system. 

To clarify the tasks so the individuals can understand and also to ensure that all the tasks have been assigned is essential. The real development of this project will analyze the requirement from the start of how the project could be ensemble. The project will be a bottom-up design. It will start from each individual completion of a small task and then integrate them into a successful complete project. 



\subsection{Actual Task Allocation}

In the real work environment, the unexpected situation happens at all time. Each members of the group face different problem, some problem is easy to solved but some is very difficult and time consuming. The back up plan and unplanned problem solving skills is needed. Although many problems happens, but all the team members have been flexible to support each other and most main problems have been solved.

Figure\ref{systemBlocD} shows a block diagram of the whole system that the developers has slitted in to blocks of tasks. 
The camera module and other hardware have uncontrollable acquiring time. Some of the group members has problem with the delay of the hardware deliveries. Therefore, the time has to extended longer than the gantt chart. The solution to this is the software has been developed according to the camera sheet of the hardware. And at this stage, the background reading is very necessary so the developers has used this waiting time to plan and develop the software, so when the components arrive, all the task can be implemented.

The task that depend on acquiring the camera are the communication between the payload and the camera,prototype camera module, and image encoding. The problem that the developers have is the camera delivered is a faulty.These tasks has been suspended until the camera have arrived.  Therefore, the new camera has to be purchased and the allocation of work is necessary. There are part that is not depended on the hardware such as encoding/decoding images, and ground station software. So the members who were assigning to work on the camera has been allocate to do the software part first.

The payload controller has the problem with the hardware part of the UAV. The payload did not respond to the UAV signal at first. Therefore, we assign another group member to support this problems. The task leader of this has been assigned to do another software module which have to be implemented. This problem also need a support from the customer to update the UAV firmware. After the problem have been noticed, it has been solved successfully. 

The SD card memory task was considered as a small task in the planned work, but in the real implementation it is very important. Therefore, we assign on of the member responsible to this task. The SD started to implemented after the camera have been arrive and implemented correctly. 

Because there is only one camera purchased, only one member of the team assign to keep the camera. There are problems with the cameras including the delay of deliveries, and camera faulty. The task leader of this has been assign to research on the progressive image on MATLAB in order to make a prototype presentation of a compression image taken. The task has been delayed from the time set to an individual. But after the second working camera arrived, the task has been done beautifully and there is enough time to combine with other tasks.

% This is part of the FinalReport document.
% Copyright (C) 2011 Piyabhum Sornpaisarn, Andrew Busse, Michael Hodgson, John Charlesworth, Paramithi Svastisinha
% See the file COPYING in FinalReport/ for copying conditions.

\section{Team resources (jc)}
\label{sec:team_resources}

This section of the management chapter covers how we controlled our use of physical resources rather than time or personnel resources. This consists of managing our budget and what we had to buy as well as what resources we had already or were provided with.

\subsection{Budget (jc)}

Our overall budget was \pounds 200 and we stayed within that while at the same time not letting it constrain us when we needed to purchase components.

\begin{table}[H]
\begin{tabular}{| l | r || r |}
\hline
\textbf{Component description} & \textbf{Quantity} & \textbf{Combined cost} \\ \hline \hline
RS485 bus transceiver (SOIC14) & 3 & 9.684 \\ \hline
RS485 bus transceiver (DIP14) & 3 & 9.756 \\ \hline
RS485 bus transceiver & 1 & 6.96 \\ \hline
Logic level converter (SOT235) & 3 & 9.684 \\ \hline
RJ45 Socket & 4 & 3.072 \\ \hline
4M flash chip (8DIP) & 4 & 2.592 \\ \hline
Arduino mega protoshield & 1 & 5.46 \\ \hline
Arduino uno protoshield & 1 & 4.38 \\ \hline
microSD breakout board & 1 & 5.99 \\ \hline
Camera module - serial JPEG TTL & 3 & 132 \\ \hline \hline
 & Total cost: & 189.58 \\ \hline
\end{tabular}
\caption{Table of expenditures}
\label{expenditure}
\end{table}

The above table (table \ref{expenditure}) details the way that the project budget was spent. The first column is a brief description of the component, the second column is the number that have been ordered and the third column is the total cost of those components.

The biggest drain on the budget was the camera modules as at \pounds 44 each these were the most expensive components. The first camera breaking was a set-back and some alternative camera options were investigated quickly but it was decided that the uCam was the best option and a new one was ordered, thankfully there was still plenty left of the budget to accommodate this.

\subsection{Electronic Material (jc)}

\subsubsection{Laboratory Equipment}

We were given access to the "Advanced Electronics Lab" on level 3 of the Zepler building. This gave us some bench-space and access to a wide range of lab equipment.

Of the lab equipment available to us the only pieces we actually used were the digital oscilloscope and the bench-top power supply. The oscilloscope was used at various stages through development to verify signals were doing what they should be, where they should be. The power supply was used to power the camera when it was discovered that the combination of SD-card and camera was drawing more current than could be supplied by the arduino.

\subsubsection{From the customer}

We were provided with a skycircuits autopilot \cite{SkyCircuits} as described in \ref{sec:autopilot_research} and some software to interface with it from the ground.

We were also provided with some example microcontroller code and a schematic for a dummy payload module i.e. a payload module that can recognise transmit tokens but that doesn't actually do anything.

\subsubsection{From ourselves}

We also had access to a range of equipment that we owned personally this included a selection of development boards:

\begin{itemize}
	\item Arduino uno, see section \ref{sec:appr_considered_arduino} \cite{arduino_serial_library}
	\item Olimexino-STM32 \cite{olimexino}
	\item mbed LPC1768 Header Board \cite{mbed}
\end{itemize}

This gave us a range of possibilities for investigation into different implementations and initial development.

Another bit of personally owned equipment that proved useful during development and testing was a USB to serial cable which was used to initially test the second camera with the 4D systems software \cite{ucam_test_software}, see section \ref{sec:existing_software_test}. This cable was also then used to attach the debug line for the rest of the development on the arduino platform.

The last bit of personally owned equipment that was used in this project was a micro SD card which was used for storing the images that come off the camera (see section \ref{sec:SD_imp}). The mounting for this used during development was loaned from ECS stores, this was replaced during later stages of development.


\section{Group Communication}
\label{group_comms}
\subsection{Formal Meetings}

For the duration of the project, we held weekly meetings on most Tuesday 
Afternoons from 4pm onwards in the Hartley Library. This allowed us to 
review progress made against progress expected, and to modify our project 
plan and allocation of work for the following week.
\\
Minutes of these meetings are available both in our repository \cite{github} 
(documents/minutes), and as an appendix ?????
\\
The timing of these (the afternoon before our weekly meetings with our 
supervisor) was also quite useful, as it allowed us to present a clear outline 
of new developments and our plans week-by-week.

\subsection{Methods of Communication}
Talk about which methods were used more often, which were useful...

The group has used various methods of communication during the project: 
email, 

\subsection{Source Control}

We decided to use version control software throughout the project, to manage 
everything from minutes to source code and payload schematics. As well as 
being a very useful tool to facilitate group work, it also helps us to meet 
our customer's request to deliver this as an open source project - at the end 
of the project, a repository could easily be gardened and presented, with the 
addition of appropriate open-source licenses at the end of the project.

Initially, we used a central Subversion (SVN) repository hosted on ECS' 
UGForge service. It was decided to use this as most of the group had 
experience using SVN. This worked well for most of the project, but 
unfortunately, something was committed to this repository that could not be 
open sourced (the customer's Ground Control Station software). This presented 
us with a problem, as although it would be trivial to revert this commit with 
the "\$ svn merge" command, this creates a new commit, and the file in 
question would remain in the repository's history and could still be accessed.

In theory, it would be possible to remove this commit from SVN history by 
using the "\$ svnadmin dump" command, filtering the offending commit out of 
the dumpfile, and regenerate the repository with the "\$ svnadmin load" 
command. However, in practice, this was not possible as our repository 
contains binary information above 64kB, therefore the ASCII editor used to 
modify the dumpfile would cut off data in the binary file after the 64kB 
limit, rendering any following data useless.

It would be possible to start a new SVN repository and check in our 
repository up to but not including the offending commit, but we would lose 
all metadata (i.e. committer, commit time, etc.). Therefore, it was decided to 
switch to git, a distributed version control system (which would allow us to 
modify project history). Converting the repository using git-svn was a trivial 
process. Although not all of the group was confident using this tool, git 
lets a committer commit on behalf of another user. Moving to git also 
allowed us to use github, a hosting service, which is free to open-source 
projects (such as ours), which also provides us with some additional 
project statistics. ??? Reference ???? Appendix ???
