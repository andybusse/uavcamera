% This is part of the FinalReport document.
% Copyright (C) 2011 Piyabhum Sornpaisarn, Andrew Busse, Michael Hodgson, John Charlesworth, Paramithi Svastisinha
% See the file COPYING in FinalReport/ for copying conditions.

%% ----------------------------------------------------------------
\section{Encountered Risks (ms)}
\label{encountered risks}
%% ----------------------------------------------------------------

\subsection{Encountered Risk: Faulty Components}
Two major components were found to be faulty very early on. 
The first ordered camera module  
arrived in an unresponsive state. 
Due to the price of the device, 
the group preferred attempting to debug the device before 
ordering a new camera module. 
Eventually however, another device needed to be ordered. 
The second camera module lasted for the majority of 
the development process, but also 
become unresponsive during the final weeks of the project,
after the PCB had been received. 
A third camera was ordered immediately for testing purposes.
Choosing a less fragile camera, or taking special precautions
when dealing with the camera would have mitigated this risk better.

The other faulty component was the autopilot module which 
was given to the group by the customer. 
Due to the nature of the autopilot, being a custom component 
from the customer, it was not possible to simply 
order a new autopilot module. 
The group decided to contact the customer for help, and 
worked with the customer to debug the autopilot software 
as discussed in section \ref{sec:probls_pl}.

\subsection{Encountered Risk: Team Members Unavailable}
Some members of the team were too occupied to attend 
all the weekly group meetings and 
the weekly meetings with the supervisor. 
However, this was not a severe problem and individual 
members were always willing to allocate 
meeting times when asked directly. 
The group members were willing to warn the others 
when they would be unavailable for a large period of time. 
This issue did not cause large set-backs in the project.

\subsection{Encountered Risk: Difficulties}
Whenever difficulties were faced, the group member 
facing the difficulty would inform the other 
group members about the problem. 
If the difficulty concerns a high priority task, 
the individual group members would reassign 
tasks to make sure that all the high priority 
tasks were completed as soon as possible. 
Thanks to the skills audit (see section \ref{sec:work_plan}), this made sure that 
any difficulties faced in the high priority 
tasks were taken care of. 
