%% ----------------------------------------------------------------
\section{Future work (jc and ab)}
%% ----------------------------------------------------------------

This section of the evaluation looks at potential additions to this project, 
which we would have done if time permitted. This starts with parts of the 
specification that we were not able to implement, and continues with some 
more advanced additions that we have considered.

\subsection{Ground Station Software (ab)}

The Ground Station Software which we have submitted remains rather buggy. 
There is a memory leakage issue which causes both our and SkyCircuits' 
software to crash after viewing a certain number of images, we have not had the time to resolve. This is not altogether critical, as the software 
is functional, and once re-started will re-connect to the autopilot. 
As the autopilot is autonomous, can take control of an aircraft from takeoff 
to landing, and is subservient to a manual controller, this is not a critical 
issue, but one that 

It is also not feature-complete (see \ref{sec:incomplete}), something which 
could be resolved with some fairly trivial modifications to our code.

%% ----------------------------------------------------------------
\section{Progressive Image Manipulation}
%% ----------------------------------------------------------------

The progressive image display was not fully implemented
by the end of the project. Below is a list of possible work
which can be undertaken to complete the progressive
image display system.

\subsection{JPEG Header Extractor}

The information obtained by the JPEG header extractor
has not been completely tested. The DQT information 
has yet to take into account multiple quantization tables
existing within a single DQT segment: only one 
quantization table is read. If a loop and struc similar to 
one implemented for the DHT is replicated for the DQT
segment, the extractor will be able to be tested and
implemented on the AVR.

The AVR implementation will require testing with
the ground station image displayer before it
can make it into a final product.

\subsection{Ground Station Software}

Little work was done for a ground station
progressive image displayer. An application would have
to be coded which will read in the information provided by
the extractor and display the image progressively.
This means that the application would need to reverse the
steps used to compress a raw image into JPEG given the 
coefficients of the Huffman tables and the quantization
tables. 

Attempting to reverse the custom compression of 
raw images MATLAB code using DCT can be a good 
starting point. If successful, reversing the quantization 
and Huffman encoding of the JPEG image would be 
required to obtain the information 
necessary to display the image.

\subsection{Video Streaming (ab)}

A rather interesting advancement of this project would be to try and implement 
real-time video streaming from the UAV. This would almost certainly have to use 
a dedicated wireless link. The 38.4 kBaud available in the autopilot modules 
is too restrictive for this sort of application, but the autopilot would still 
be able to control such an application (i.e. able to start/stop transmission, 
toggle power to the module, etc.) but things like 
minimising delay between video capture and video 
display, finding a suitably fast and robust wireless link (which implements 
automatic repeat requests) would make this task a very difficult one. It may 
be preferable to try and hack an existing wireless video module to one's own purposes (if such modules exist and are 
well documented)

\subsection{Using Multiple Cameras (ab)}

Another interesting application would be to mount multiple cameras (of the same 
model purchased for this project) on the UAV, and either take pictures from all 
the cameras at the same time, from some of the cameras at the same time, etc. 
The main issue here will be multiplexing the serial connection to all of the 
cameras, something considered for this project but ultimately not chosen.

\subsection{Using Multiple Types of Camera (ab)}

It would be useful to provide a range of camera modules to the customer, and 
an electronic module that would be able to communicate with all of them. This 
would include all the types of camera we considered (USB, serial and analogue), 
with varying resolutions. This would be rather difficult to implement, as the 
standards used across two types of the same camera interface can be rather 
different, so a driver would need to be written for each type of camera used. 
Also, the budget for such a project would be quite high.

\subsection{SD card filesystem (jc)}

It would be a nice additional bit of functionality if the sd card on the UAV could be accessed as if it were another file system on the computer running the ground station software. This would allow any of the photos stored on the UAV to be viewed easily and would also mean that the UAV could be used as an unusual form of data transfer, carrying files from one place to another.

This would also be useful if the payload was extended to contain other sensors that outputted files of a different type to the jpeg image from the camera.

\subsection{HDR Photography (jc)}

Image processing could be implemented in the image viewer software to enable high dynamic range (HDR) imaging. This could be useful for finding objects within a certain colour/intensity range by expanding the dynamic range over that region.

\subsection{3D photography (jc)}

If multiple cameras are implemented then these could be separated and the images from both combined to form 3 dimensional images, the vertical resolution of the 3d image would be proportional to the separation of the cameras.

An alternate method would be to use a similar method to synthetic aperture radar (SAR) and take one photograph, fly on a bit further and then take another. The speed and/or gps data from the UAV could then be used to work out the distance between the two photographs which could then be fed into the image processing algorithm in order to again generate a 3d image (presuming sufficient overlap within the images).

%\subsection{Camera SD bus}

