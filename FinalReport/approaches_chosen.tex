%% ----------------------------------------------------------------
\chapter{Chosen Design Approaches}
%% ----------------------------------------------------------------


\section{Camera module}
\label{sec:John_chosen_options}

The camera type chosen was a serial camera as described in \ref{sec:Serial_option}. The particular model chosen was uCam (microCam) or the "Camera Module - Serial JPEG TTL" from the "coolcomponents" website.

The feature set of the uCam is as follows:
	\begin{itemize}
		\item It can output images in both RAW and jpeg formats
		\item It can output RAW images at a range of resolutions:
		\begin{itemize}
			\item 80 x 60
			\item 160 x 120
			\item 320 x 240
			\item 640 x 480
			\item 128 x 128
			\item 128 x 96
		\end{itemize}
		\item It can output jpeg images at a range of resolutions:
		\begin{itemize}
			\item 80 x 64
			\item 160 x 128
			\item 320 x 240
			\item 640 x 480
		\end{itemize}
		\item it can output RAW images with a range of colour settings:
		\begin{itemize}
			\item 2bit Gray Scale
			\item 4bit Gray Scale
			\item 8bit Gray Scale
			\item 8bit Colour
			\item 12bit Colour
			\item 16bit Colour
		\end{itemize}
		\item It will auto-detect baud rates from 14400 to 115200
		\item It has selectable baud rates up to 1228800
		\item Small physical size at 32mm x 32mm
		\item Well documented
	\end{itemize}

The camera was chosen because this feature set meets the specification and also allows for additional functionality, such as setting the resolution, if the full feature set is exploited.

\section{Payload/Ground Station Interaction}

\section{Ground Station Image Viewer}

\section{Progressive JPEG Manipulation}

\section{Physical Implementation}

The physical implementation of the payload has taken many forms, reflecting 
the each of the design-prototype-test cycles that our group experienced.

\begin{itemize}
\item Initially, the system was built in two separate parts: The first was 
the modified version of the ATmega168P-based sample schematic provided by 
our customer, which handled communication to the payload port of the 
autopilot. The second was an Arduino Uno R2 with its Serial (UART) line 
connected to the camera and an SPI connection to an SD card. This system 
simply took an image from the camera and wrote it to the SD card. The SD card 
to Arduino connection was via some protoboard and an SD card reader breakout 
board on loan from Zepler Stores.
\item The next design iteration used an "Il Matto" board and daughter board, 
designed for 2011's D4 lab (a two-week Design and Build exercise for second 
year Electronic Engineers). The board itself is rather simple - an ATmega644P 
with all of its ports connected to expansion headers, an SPI line connection 
to an SD card reader, and power taken from a mini-USB connection, stepped 
down to 3V3 by a linear regulator. A daughter board (which slots into the 
expansion headers) has the transceiver to communicate with the autopilot

This implementation of the payload contains all the elements of the final 
schematic, so technically could be test-flown.
\item Whereas the above approach would technically fulfil the specification, 
implementing it would be expensive: the Il Matto boards were one-off boards 
designed specifically for a lab, so a customer would certainly need to 
manufacture and build their own, including the daughter board. Having this as 
a single PCB solution would be a lot neater, smaller, lighter and cheaper.

Therefore, a final PCB was designed (this is elaborated upon in 
\ref{sec:PCB-implementation}), specific to our application, with all the 
features we need. Only through-hole components are used where possible, 
making the task of soldering components to the PCB as easy as possible.
\end{itemize}

\section{System Design Overview}

