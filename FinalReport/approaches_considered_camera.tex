\section{Camera Module}
\label{sec:John_options}

A range of different camera modules were researched and considered for use in the project.

\subsection{Approach: Compact Digital Camera}
\label{sec:Compact_option}
A huge range of compact digital cameras are available for purchase and this was considered as a possible approach.

Advantages:
      \begin{itemize}
         \item High resolution of typically 5+ Megapixels
		 \item Fully encapsulated
		 \item High quality images possible thanks to automatic focus and zoom
     \end{itemize}

Disadvantages:
     \begin{itemize}
        \item Difficult/impossible to communicate with: most modern digital cameras has a USB socket for downloading images off the camera but it is much rarer to be able to send the camera commands (e.g. take photo) and for those cameras where it is possible the command structure is not usually simple.
        \item Requires USB host device in order to properly communicate as only USB host devices provide power via usb and can have USB peripherals connected to them.
	\item Relatively large and heavy compared to the other approaches considered.
	\item Expensive
     \end{itemize}

In order to communicate with a camera via USB a USB host is required. The process is usually fairly complex a driver is normally used. This is extremely difficult to implement on a device compact enough for use in this project.

\subsection{Approach: USB Camera}
\label{sec:USB_option}
Many USB webcams are available for purchase and this was considered as a possible approach.

Advantages:
      \begin{itemize}
         \item Low cost of typically around \pounds 15
         \item High resolution of typically 1 to 3 Megapixels
		 \item Small size
		 \item Cabling included so can be easily positioned in the UAV
     \end{itemize}

Disadvantages:
     \begin{itemize}
        \item Difficult to communicate with - NEED TO EXTEND ON THIS, THIS SHOULD BE REASONBALY DETAILED AND SAY THINGS LIKE DATASHEETS ARE NOT AVAILABLE FOR DEVICE MOST DEVICES (COULD BE A SEPERATE POINT) AND THAT COMPLEX DRIVERS ARE REQUIRED WHICH SIMPLY ARE NOT AVAILABLE ON MOST SMALL/CHEAP EMBEDDED PROCESSORS. GETTING THIS TO WORK COULD BE A PROJECT IN IT'S OWN RIGHT
        \item Requires USB host device - SAME AS ABOVE
	\item High resolution means longer transfer times - MARKED FOR REMOVAL - SEE ROBS COMMENTS
     \end{itemize}

This approach was ruled out because of the difficulty in communicating with this camera type. In order to communicate with a USB camera a USB host is required and as the process is usually fairly complex a driver is normally used. This is extremely difficult to implement on a device compact enough for use in this project. - TAKE OUT THE BIT SAYING IT WAS RULED OUT AND PUT IT IN THE APPROACH CHOSEN: SECTION BELOW

\subsection{Approach: Analogue Camera}
\label{sec:Analog_option}
Camera modules with analogue composite outputs are available, usually for cctv or surveilence purposes, and were also considered as an approach.

Advantages:
      \begin{itemize}
         \item Range of resolutions available
	\item Small size - MARKED FOR REMOVAL - THEY ARE JUST AS SMALL AS OURS
	\item MATURE MARKET, LOTS OF CHOICE
	
     \end{itemize}

Disadvantages:
     \begin{itemize}
        \item Awkward to communicate with - NEEDS MUCH MORE EXPLANATION, NEED TO METION ADC AS WITH ROBS COMMENT, PERPAHS QUICKLY MENTION THAT WOULD PROBABLY USE SOMETHING LIKE COMPONENT OR COMPOSITE VIDEO WHICH ARE COMPLEX AND DIFFICULT TO DECODE
        \item Require case and cabling for placement in UAV
		\item Can be expensive at up to around \pounds 80
		\item Often video rather than still cameras
		\item Often black and white rather than colour
     \end{itemize}

This approach was ruled out because of the additional complexity required in converting from analogue to digital whilst maintaining all the information in the image, that would be required to send an image from a camera of this type through a microprocessor.

\subsection{Approach: Serial Camera Module}
\label{sec:Serial_option}
Some camera modules are available with UART rather than USB serial communications and these to were considered as an approach.

Advantages:
      \begin{itemize}
		 \item Easy to communicate with
         \item Range of resolutions available
		 \item Small size
		 \item Well documented COME WITH DATASHEETS AND DESIGNED SPECIFICALLY FOR ELECTRONICS DEVELOPMENT
     \end{itemize}

Disadvantages:
     \begin{itemize}
        \item Require case and cabling for placement in UAV
	\item moderately expensive at around \pounds 40
        \item Require case and cabling for placement in UAV
     \end{itemize}

This was the approach that was chosen as the ability to communicate via UART combined with the good documentation should allow for control to be established fastest with this camera type.

\section{Approach Chosen: Serial Camera module}
\label{sec:John_chosen_options}

The camera type chosen was a serial camera as described in \ref{sec:Serial_option}. EXPLAIN WHY WE CHOSE THIS AS OPPOSED TO OTHER CHOICES, BUT BRIEFLY

The particular model chosen was uCam (microCam) or the "Camera Module - Serial JPEG TTL" from the "coolcomponents" website.

The feature set of the uCam is as follows:
	\begin{itemize}
		\item It can output images in both RAW and jpeg formats
		\item It can output RAW images at a range of resolutions:
		\begin{itemize}
			\item 80 x 60
			\item 160 x 120
			\item 320 x 240
			\item 640 x 480
			\item 128 x 128
			\item 128 x 96
		\end{itemize}
		\item It can output jpeg images at a range of resolutions:
		\begin{itemize}
			\item 80 x 64
			\item 160 x 128
			\item 320 x 240
			\item 640 x 480
		\end{itemize}
		\item it can output RAW images with a range of colour settings:
		\begin{itemize}
			\item 2bit Gray Scale
			\item 4bit Gray Scale
			\item 8bit Gray Scale
			\item 8bit Colour
			\item 12bit Colour
			\item 16bit Colour
		\end{itemize}
		\item It will auto-detect baud rates from 14400 to 115200
		\item It has selectable baud rates up to 1228800
		\item Small physical size at 32mm x 32mm
		\item Well documented AND SPECIFICALLY DESIGNED FOR ELECTRONICS DEVELOPMENT (MAYBE)
	\end{itemize}

The camera was chosen because this feature set meets the specification and also allows for additional functionality, such as setting the resolution, if the full feature set is exploited.