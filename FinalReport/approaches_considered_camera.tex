\subsection{Camera Module}
\label{sec:John_options}

A range of different camera modules were researched and considered for use in the project. It was decided early on in the project that camera modules would be used rather than attempting to interface with standard digital cameras.

\subsubsection{Approach: USB Camera}
\label{sec:USB_option}
Many USB webcams are available for purchase and this was considered as a possibly approach.

Advantages:
      \begin{itemize}
         \item Low cost of typically around £15
         \item High resolution of typically 1 to 3 Megapixels
		 \item Small size
		 \item Cabling included so can be easily positioned in the UAV
     \end{itemize}

Disadvantages:
     \begin{itemize}
        \item Difficult to communicate with
        \item Requires USB host device
		\item High resolution means longer transfer times
     \end{itemize}

This approach was ruled out because of the difficulty in communicating with this camera type. In order to communicate with a USB camera a USB host is required and as the process is usually fairly complex a driver is normally used. This is extremely difficult to implement on a device compact enough for use in this project.

\subsubsection{Approach: Analogue Camera}
\label{sec:Analog_option}
Camera modules with analogue composite outputs are available, usually for cctv or surveilence purposes, and were also considered as an approach.

Advantages:
      \begin{itemize}
         \item Range of resolutions available
		 \item Small size
     \end{itemize}

Disadvantages:
     \begin{itemize}
        \item Awkward to communicate with
        \item Require case and cabling for placement in UAV
		\item Can be expensive at up to around £80
		\item Often video rather than still cameras
		\item Often black and white rather than colour
     \end{itemize}

This approach was ruled out because of the additional complexity required in converting from analogue to digital whilst maintaining all the information in the image, that would be required to send an image from a camera of this type through a microprocessor.

\subsubsection{Approach: Serial Camera}
\label{sec:Serial_option}
Some camera modules are available with UART rather than USB serial communications and these to were considered as an approach.

Advantages:
      \begin{itemize}
		 \item Easy to communicate with
         \item Range of resolutions available
		 \item Small size
		 \item Well documented
     \end{itemize}

Disadvantages:
     \begin{itemize}
        \item Require case and cabling for placement in UAV
		\item moderately expensive at around £40
     \end{itemize}

This was the approach that was chosen as the ability to communicate via UART combined with the good documentation should allow for control to be established fastest with this camera type.