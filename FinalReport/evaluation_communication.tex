% This is part of the FinalReport document.
% Copyright (C) 2011 Piyabhum Sornpaisarn, Andrew Busse, Michael Hodgson, John Charlesworth, Paramithi Svastisinha
% See the file COPYING in FinalReport/ for copying conditions.

\section{Evaluation on Group Communication (ps)}
\label{evaluation on group communication}

\subsection{Source Control}
As describe in section\ref{source control}, there were problems with subversion where some of the files that could not be put into the public folder were accidentally placed there.
This cause some of the group members tries to fixed this problem.
Although the subversion was effective, we have decided to change it to git.
Git is easy to use and all the members can learn it fast and there are no furthur problem.
This save time of combining report because git warn the user when they are submitting something that other member is working on it.
Therefore, all the work flowed well and we have minimized the submission of work problems.


\subsection{E-mail}
The way of communication by email was effective for our group.
Because all the members know how to use it, there is no time wasting on learning a new way of communication.
However, it is likely that some of the group members misread some of the email or did not react to them immediately.
To make this a more powerful way of communication and to ensure that all the members have read the message, the group should have had a rule that on every email, there must be a reply from all of the group member.

\subsection{Supervisor Meeting}
The formal meeting with supervisor that we have described in section\ref{group_comms} has been done successfully. 
In order to make sure these weekly meetings with the project supervisor were fruitful questions were prepared beforehand, to ensure that the project was on track and the most immediate tasks were made very clear.
The customer also came to the meeting once in every two weeks and would evaluate the progress of the project.
This made the project flow smoothly due to having instant feedback from the creator of the UAV module.
The customer's considerate feedback and presence ensured a successful project.
Nonetheless, the group could have made more out of the meetings by showing some of the progress on the report  throughout the project development period.
This would have saved the time needed to finish the report.


\subsection{Telephone (ms)}

Telephone was used rarely. Generally, it was used in 
emergencies, for example, 
when a group member was unable to attend a meeting
and could not inform the others by e-mail. It was also
used when members were unexpectedly absent from 
certain important meetings. Overall, e-mail and 
telephone communication complemented each other well.

\subsection{Internet Relay Chat (ms)}

As the project advanced during the development phase,
more and more tasks required group members working together
in order to be successfully completed. This is due to 
separate modules needing to communicate with each other
and having different people assigned into developing the different modules
in parallel. As this became a more apparent issue, the group adopted
two IRC channels to keep in constant contact with each other.

The first IRC channel was used for development and keeping the 
different module developers in contact with each other as much as possible. 
The second IRC channel was mainly used for the purpose of writing the project
report, due to the large group work involved in creating one cohesive report
from the efforts of different individuals.

Both IRC channels proved invaluable in the later phases of the project, but
due to the individual nature of the early development phases of the project,
were not necessary to be implemented early on. However, it would have been
useful to prepare an IRC at the beginning of the project, if only to have 
another communication tool to help the group members work together.
