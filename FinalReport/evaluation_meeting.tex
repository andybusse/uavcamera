%% ----------------------------------------------------------------
\section{Group Meetings (ms)}
\label{evaluation communication}
%% ----------------------------------------------------------------

\subsection{Group Meeting Evaluation}

Having meetings on a weekly basis proved to be very efficient
for the group. Because individual group members met with 
each other on a regular basis throughout the project outside
the weekly meetings, the meetings allowed enough time 
between them to keep them from being redundant.

When certain important deadlines were nearing, the group
would naturally meet up on a more frequent basis and
attempt to tackle important challenges together, even when
no meeting was planned. The meetings allowed the group 
to assess their progress and were vital to avoiding tasks
overrunning.

\subsection{Minutes Evaluation}

Minutes were taken during each weekly meeting. 
Some minutes were taken during other important meetings 
as well, including meetings with the customer and the 
supervisor. Although, the group member who took 
minutes was not monitored as well as it could have been, 
this did not have any significant effect on the quality 
and/or consistency of the minutes.

The minutes taken were useful for making sure that
all group members were able to check what tasks
they were assigned and catch up on meetings that they
could not assist. The only major setback with the meetings
was the lack of urgency with which they were distributed
to the other group members. Some group members would
update their minutes immediately after the meeting, whereas 
others would not until after the next meeting. Despite
not seriously setting back the project, this undermined the
usefulness of a nonetheless helpful tool.