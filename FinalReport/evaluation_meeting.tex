%% ----------------------------------------------------------------
\section{Communication Evaluation (ms)}
\label{evaluation communication}
%% ----------------------------------------------------------------

\subsection{Group Meeting Evaluation}

Having meetings on a weekly basis proved to be very efficient
for the group. Because individual group members met with 
each other on a regular basis throughout the project outside
the weekly meetings, the meetings allowed enough time 
between them to keep them from being redundant.
The group has a laboratory room that we spend most time implementing our project.
This motivate the members to work as a group and they can support one another.
However, we could improve our teamwork by setting a time to work 
in the lab together everyday for a certain period of time. 
This can motivate the group members to work harder and more effective.

When certain important deadlines were nearing, the group
would naturally meet up on a more frequent basis and
attempt to tackle important challenges together, even when
no meeting was planned. The meetings allowed the group 
to assess their progress and were vital to avoiding tasks
overrunning.

Minutes of these meetings are available both in our repository \cite{github} 
(documents/minutes), and as an appendix \cite{chap:meeting_minutes}.

\subsection{Minutes Evaluation}

Minutes were taken during each weekly meeting. 
Some minutes were taken during other important meetings 
as well, including meetings with the customer and the 
supervisor. Although, the group member who took 
minutes was not monitored as well as it could have been, 
this did not have any significant effect on the quality 
and/or consistency of the minutes.

The minutes taken were useful for making sure that
all group members were able to check what tasks
they were assigned and catch up on meetings that they
could not assist. The only major setback with the meetings
was the lack of urgency with which they were distributed
to the other group members. Some group members would
update their minutes immediately after the meeting, whereas 
others would not until after the next meeting. Despite
not seriously setting back the project, this undermined the
usefulness of a nonetheless helpful tool.

\subsection{Methods Of Communication Evaluation}

The following methods of communication were used
during the project. One more method of communication
was used in addition to the planned methods.

\subsubsection{E-mail}

For the most part, most of the group members kept track
of their e-mails. On the rare occasion when a group member
became negligent in checking their e-mail, it would either be 
for a brief period of time, or they would be prompted by
telephone to check their e-mail.

\subsubsection{Telephone}

Telephone was used rarely. Generally, it was used in 
emergencies, for example, 
when a group member was unable to attend a meeting
and could not inform the others by e-mail. It was also
used when members were unexpectedly absent from 
certain important meetings. Overall, e-mail and 
telephone communication complemented each other well.

\subsubsection{Internet Relay Chat}

As the project advanced during the development phase,
more and more tasks required group members working together
in order to be successfully completed. This is due to 
separate modules needing to communicate with each other
and having different people assigned into developing the different modules
in parallel. As this became a more apparent issue, the group adopted
two IRC channels to keep in constant contact with each other.

The first IRC channel was used for development and keeping the 
different module developers in contact with each other as much as possible. 
The second IRC channel was mainly used for the purpose of writing the project
report, due to the large group work involved in creating one cohesive report
from the efforts of different individuals.

Both IRC channels proved invaluable in the later phases of the project, but
due to the individual nature of the early development phases of the project,
were not necessary to be implemented early on. However, it would have been
useful to prepare an IRC at the beginning of the project, if only to have 
another communication tool to help the group members work together.