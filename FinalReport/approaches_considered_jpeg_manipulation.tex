\section{Progressive JPEG Manipulation}

\subsection{Approach: JPEG Manipulation}
\label{sec:jpegmanipulation}

This section weighs the decision of implementing a form of custom JPEG image manipulation to 
allow the image to be displayed progressively, as requested by the customer.

\subsubsection*{Custom JPEG Manipulation}

This design approach details the choice to code an algorithm which 
modify the image received from the camera in order to display it progressively. 
Ideally, this would send an extremely low resolution image to the ground station which 
would progressively improve in resolution while more information is sent from the camera.

\paragraph*{Advantages}
\begin{itemize}
	\item Allows an image to be sent manually evaluated sooner
	\item Allows the user to cancel images which show begin to show unwanted information, saving time.
	\item Can possibly allow the image information to be optimized, sending only the information necessary for display.
\end{itemize}

\paragraph*{Disadvantages}
\begin{itemize}
	\item Requires effort and time, including research into the manipulation of the JPEG format.
	\item Benefit is minimal if the sending a simple image is already very quick.
	\item May lose metadata of the JPEG image data (date created, etc.), which can cause software incompatibility issues.
\end{itemize}

\subsubsection*{Standard JPEG Transfer}

This design approach simply displays the JPEG image from the camera as it is received. 

\paragraph*{Advantages}
\begin{itemize}
	\item Requires little to no effort (already done by default).
	\item All the metadata of the image is kept unchanged.
	\item Sends the image at an acceptable speed.
\end{itemize}

\paragraph*{Disadvantages}
\begin{itemize}
	\item Takes time to send all the image data to the ground station.
	\item During the transfer period, state of image is unknown and can be wasted if image is undesireable.
	\item Image sends all the metadata, including that which can be ignored.
\end{itemize}

\subsection{JPEG Data Access}

One of the biggest choices to make concerning the JPEG extractor was how it would 
access the JPEG image data coming from the camera. The two methods applied and tested are studied below:

\subsubsection*{Static Access}

This method of accessing the image data involves storing the JPEG image data into memory and 
having the extractor read the JPEG data as if it were a data array. 
This method would require all of the image data to be stored before beginning to extract the necessary information.

\paragraph*{Advantages}
\begin{itemize}
	\item Easier to provide progressive scanning of the image due to possessing the entire entropy-encoded image data. It is 
		easier to act on the whole image rather than the data coming in at a certain rate.
	\item Much fewer calls to the JPEG byte stream. This allows the extractor to obtain all the information faster from the moment it is called.
	\item Only one instance of memory allocation at the very beginning. 
		This prevents the code from crashing due to insufficient memory in the middle of the information extraction process and wasting time.
\end{itemize}

\paragraph*{Disadvantages}
\begin{itemize}
	\item Very memory intensive: requires storing the entire image in memory before being called.
	\item Delayed by image storage. Although there is only one call to the image data, it 
		must wait until the entire input string is stored before any action can be taken.
	\item A lot of wasted memory. Not all the information from the image is needed for the progressive JPEG manipulation.
\end{itemize}

\subsubsection*{Dynamic Access}

This method of accessing the image data involves using a class to read in one byte from the image data stream at a time.

\paragraph*{Advantages}
\begin{itemize}
	\item More efficient for memory space overall. No need to store the entire image at once.
	\item Allows more efficient use of memory. The extractor only saves the information that it will use.
\end{itemize}

\paragraph*{Disadvantages}
\begin{itemize}
	\item More difficult to use progressive scanning. Entropy-encoded data must be stored fully before progessive scanning can occur.
	\item Delayed by multiple read calls to the JPEG data. A read call must be made for each byte of the JPEG data stream.
	\item Memory allocated dynamically. More vulnerable to insufficient memory crashes during the extraction process.
\end{itemize}

\subsection{Entropy-Encoded Data Storage}

Another major decision to make was the method of storing the Entropy-encoded image data of the JPEG file, 
which is divided into chrominance pixels made up of 3 different components.

\subsubsection*{Store As Chrominance Pixels}

This method stores the chrominance pixels in structures one by one. 
The data is stored in a linked list of chrominance pixels, containing a certain number of Y values and one pair of Cb and Cr values.

\paragraph*{Advantages}
\begin{itemize}
	\item Makes the YCbCr colour modelling method more obvious. 
		This method intuitively partitions the encoded image data using the YCbCr convention.
	\item Requires little data manipulation. Structures are made as input stream is read in, one at a time.
	\item Comprehensive memory allocation. Allocates the appropriate memory for one chrominance pixel as needed.
\end{itemize}

\paragraph*{Disadvantages}
\begin{itemize}
	\item Encoded image data becomes more difficult to apply progressive scans to.
	\item No spatial information is stored. Data is stored as a linear list of chrominance pixels instead of a table representing the image.
	\item Memory must be allocated dynamically. More vulnerable to insufficient memory crashes during the extraction process.
\end{itemize}

\subsection*{Store Components Seperately}

This method of accessing the image data involves storing the three components Y, Cb, and Cr into tables.

\paragraph*{Advantages}
\begin{itemize}
	\item Makes progressive scanning easier. The three components can be selected individually.
	\item Information can be stored in a table to represent the image spatially.
	\item Memory can be allocated either dynamically or prior to the read process.
\end{itemize}

\paragraph*{Disadvantages}
\begin{itemize}
	\item Requires more manipulation of the entropy-encoded image data as it is read in. 
		Components must be monitored to not lose chrominance pixel grouping.
	\item Chrominance information must be stored either unintuitively (a smaller table than the image) or 
		inefficiently (an image sized table with empty values).
	\item Generally more difficult to monitor the information stored.
\end{itemize}