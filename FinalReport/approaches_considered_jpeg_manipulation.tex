\section{Image Manipulation (ms)}
\label{sec:progressiveimagedisplay}

The design of the progressive image
display system to be implemented on the
payload and the ground station required some
important decisions to be made before 
development could begin. This included
the type of image file which would be
manipulated, how the image sent from the camera on 
the UAV would be accessed,
and the method of storing the data
and sending it to the ground station.

\subsection{Image File to Manipulate}

The first design decision was to choose the type of image 
file which would be the most practical to work with given
the properties of the camera and the
capabilities of the payload software.

\subsubsection{Approach: Custom Raw Image Manipulation}

This approach involves using the custom image 
manipulation algorithm as specified 
in section \ref{Custom_comp_research}.

Advantages:
\begin{itemize}
	\item Would be able to handle the images
		given by the camera regardless of the format
		the camera saves the image.
	\item Avoids quantization to allow 
		for lossless compression: No information is
		lost when the image is 
		compressed and decompressed.
	\item Uses a simple, reliable, and easy to 
		understand image compression technique
		(Discrete Cosine Transform)
	\item Can more easily create a progressive
		display of the image due to having less
		information to decode.
\end{itemize}

Disadvantages:
\begin{itemize}
	\item Restricted to decoding raw images; 
		cannot deal with JPEG images.
	\item Does not compress the image as 
		efficiently as more advanced 
		image compression techniques.
\end{itemize}

\subsubsection{Approach: JPEG File Manipulation}

This approach involves manipulating JPEG standard image 
files instead of raw image files. This requires the
microcontroller to work around the compression offered
by the JPEG file standard (see section \ref{jpeg_image_compression}).

Advantages:
\begin{itemize}
	\item Compatible with most standard
		camera outputs.
	\item Takes advantage of the reliable and
		efficient JPEG compression algorithm.
	\item Uses quantization to allow for a greater
		degree of image compression.
\end{itemize}

Disadvantages:
\begin{itemize}
	\item Compression is lossy: image loses 
		more and more information 
		each time it is compressed 
		(saved and opened).
	\item Requires more sophisticated and complex
		decompression algorithms.
	\item Requires more background research to
		understand and manipulate.
\end{itemize}

\subsubsection{Chosen Approach: JPEG Manipulation}

Before the camera was agreed upon, work began on the
custom raw image compression algorithm. This
allowed us to get a good idea of how the progressive
display of the image would work in the final product.
However, when the camera was chosen, it's output
was restricted to JPEG image files. Therefore, JPEG 
file manipulation was chosen in order to adapt to
the hardware given.

\subsection{JPEG File Manipulation}
\label{sec:jpegmanipulation}

This section weighs the decision of implementing a 
form of custom JPEG image manipulation to allow the 
image to be displayed progressively, as requested by the customer.

\subsubsection{Approach: Custom JPEG Manipulation}

This design approach details the choice to code an algorithm which 
would modify the image received from the camera in order to 
display it progressively. Ideally, this would send an extremely low 
detail image to the ground station which would progressively improve 
while more information is sent from the camera by implementing 
the higher frequency components of the image.

Advantages:
\begin{itemize}
	\item Allows an image to be sent and 
                     be manually evaluated sooner.
	\item Allows the user to cancel images which begin to show 
		unwanted information, saving processing time.
	\item Can possibly allow the image information to be 
		optimized, sending only the 
		information necessary for display.
\end{itemize}

Disadvantages:
\begin{itemize}
	\item Requires effort and time, including research into 
		the manipulation of the JPEG format.
	\item Benefit is minimal if the sending a simple image 
		is already very quick.
\end{itemize}

\subsubsection{Approach: Standard JPEG Transfer}

This design approach simply displays the JPEG image from 
the camera as it is received. 

Advantages:
\begin{itemize}
	\item Requires little to no effort 
		(already done by default).
	\item All the metadata of the image is kept unchanged.
\end{itemize}

Disadvantages:
\begin{itemize}
	\item Takes time to send all the image data to the ground station.
	\item During the transfer period, state of image is unknown and 
		can be wasted if image is undesirable.
	\item Image sends all the metadata, including 
		metadata which can be ignored.
\end{itemize}

\subsubsection{Chosen Approach: Custom JPEG Manipulation}

Prior to knowing the exact speed it would take to send 
an image from the camera to the ground station, 
it was decided that custom JPEG manipulation would optimize 
the image further and help reach the target time specified (3 minutes). 
Therefore, tasks concerning the custom manipulation of 
the JPEG image were planned and worked on 
at the beginning of the project. 
However, when a working module was constructed, the time needed to 
send the image was already well within the time specified. 
Because this was the primary advantage of the 
custom JPEG manipulation process, 
the task's priority was dropped accordingly.

\subsection{JPEG Data Access}

One of the biggest choices to make concerning the 
image manipulation software to be implemented on the AVR was 
how it would  access the JPEG image data coming from the camera. 
The two methods considered are studied below:

\subsubsection{Approach: Static Access}

This method of accessing the image data involves storing 
the JPEG image data into memory and 
having the AVR read the JPEG data as if it were a data array. 

Advantages:
\begin{itemize}
	\item Easier to provide progressive scanning of the 
		image due to possessing the entire 
		entropy-encoded image data. It is easier to 
		read the whole image rather than 
		have the data coming in at a certain rate on the AVR.
		This would allow the AVR to have more options on
		the order of the image data to send.
	\item Much fewer calls to the JPEG byte stream. 
		This allows the extractor to obtain all the 
		information faster from the moment it is called.
	\item Only one instance of memory allocation at the very beginning. 
		This prevents the code from crashing due to insufficient 
		memory in the middle of the information 
		extraction process and wasting time.
\end{itemize}

Disadvantages:
\begin{itemize}
	\item Very memory intensive: requires 
		storing the entire image in memory 
		before being called due to the variable
		size of the JPEG file and its components.
	\item Delayed by image storage.
		Although there is only one call to the 
		image data, it must wait until the entire 
		input string is stored before 
		any action can be taken safely.
	\item A lot of wasted memory. 
		Not all the information 
		from the image is needed for the 
		progressive JPEG manipulation.
\end{itemize}

\subsubsection{Approach: Dynamic Access}

This method of accessing the image data 
involves using a class to read in one byte 
from the image data stream at a time.

Advantages:
\begin{itemize}
	\item More efficient for memory 
		space overall. No need to store 
		the entire image at once.
	\item Allows more efficient use of memory. 
		The AVR only saves and sends the 
		information that the ground station will use.
\end{itemize}

Disadvantages:
\begin{itemize}
	\item More difficult to use progressive scanning. 
		Entropy-encoded data must be stored fully 
		before progressive scanning can occur.
	\item Delayed by multiple read calls to the JPEG data. A 
		read call must be made for each 
		byte of the JPEG data stream.
	\item Memory allocated dynamically. 
		More vulnerable to insufficient memory 
		crashes during the extraction process.
\end{itemize}

\subsubsection{Chosen Approach: Dynamic Access}

The dynamic form of data accessing was preferred 
due to the memory constraints imposed on the device.

\subsection{Entropy-Encoded Data Storage}

Another major decision to make was the method of 
storing the entropy-encoded image data of the JPEG file, 
which is divided into chrominance pixels made up of 
3 different components.(See section 
\ref{sec:colour_space_conversion} for
more information on chrominance pixels.)

\subsubsection{Approach: Store As Chrominance Pixels}

This method stores the chrominance pixels in 
the image one by one. 
The data is stored in a linked list of chrominance pixels, 
containing a certain number of Y values and 
one pair of Cb and Cr values.

Advantages:
\begin{itemize}
	\item Makes the YCbCr colour modelling method 
		more obvious. 
		This method intuitively partitions the 
		encoded image data using the YCbCr convention.
	\item Requires little data manipulation. 
		Structures are made as input stream is read in, 
		one at a time.
	\item Comprehensive memory allocation. 
		Allocates the appropriate memory for 
		one chrominance pixel as needed.
\end{itemize}

Disadvantages:
\begin{itemize}
	\item Encoded image data becomes more 
		difficult to apply progressive scans to.
	\item No spatial information is stored. 
		Data is stored as a linear list of chrominance 
		pixels instead of a table representing the image.
	\item Memory must be allocated dynamically. 
		More vulnerable to insufficient memory 
		crashes during the extraction process.
\end{itemize}

\subsubsection{Approach: Store Components Seperately}

This method of accessing the image data involves 
storing the three components Y, Cb, and Cr into tables.

Advantages:
\begin{itemize}
	\item Makes progressive scanning easier. 
		The three components can be selected individually.
	\item Information can be stored in a table to 
		represent the image spatially.
	\item Memory can be allocated either 
		dynamically or prior to the read process.
\end{itemize}

Disadvantages:
\begin{itemize}
	\item Requires more manipulation of the 
		entropy-encoded image data as it is read in. 
		Components must be monitored to not 
		lose chrominance pixel grouping.
	\item Chrominance information must be stored either 
		unintuitive (a smaller table than the image) or 
		inefficiently (an image sized table with empty values).
	\item Generally more difficult to monitor the information stored.
\end{itemize}

\subsubsection{Chosen Approach: Store As Chrominance Pixels}

The entropy-encoded data will be stored as chrominance 
pixels for the sake of clarity. 
It is also a method which could be coded more easily 
than separate component storage. 
This allowed the AVR to work with the progressive 
JPEG encoder sooner to avoid the task from overrunning.